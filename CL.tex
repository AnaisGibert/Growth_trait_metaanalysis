\documentclass{letter}
\usepackage{hyperref}
\signature{Anais Gibert on behalf of all authors}
\address{Biological Sciences\\ Macquarie University \\ NSW 2109, Australia}
\begin{document}

\begin{letter}{Editor-in-Chief Journal of Ecology}
\opening{Dear Dr David J. Gibson,}


May we please submit for Journal of Ecology a manuscript titled ``On the link between functional traits and growth rate: meta-analysis shows effects change with plant size, as predicted``. We would be grateful if you would consider it for publication.

May we please drawn your attention on the literature cited? We strongly believe that authors whose data publications enter into meta-analyses deserve to get credit via citation. In the current draft, these citations are in the main list of Literature Cited --as starred references-- increasing substantially it length. We hope it may no affect the review process. Maybe do you have another solution to situations of this kind?

\textbf{Novelty statement:}
In a nutshell, this paper sorts out the differences across plant species in potential growth rates. We believe it's a very substantial advance. Growth rates are obviously really important for ecology. They have been seen as a core element of plant ecological strategies going back to Grime and Hunt 1975. Over the past 20 years much ecological strategy research has shifted into "trait ecology", using traits such as SLA and wood density and seed mass as indicators of ecological strategy. Yet empirical reports of the correlations between traits and growth rates have been inconsistent. Our manuscript first spells out reasons why traits should influence growth differently depending what size the individual plant has achieved -- in other words, traits should influence growth trajectories rather than simply making species fast-growing versus slow-growing throughout life. Then second, we provide meta-analysis of correlations that have been reported between traits and growth rate. To the extent data are available, these correlations are shown to shift depending on plant size consistently with the reasons explained and the predictions made in the first part of the paper.  


We are excited by this work and look forward to your evaluation of the paper.


\closing{Sincerely}


\end{letter}
\end{document}