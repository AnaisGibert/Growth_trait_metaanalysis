\documentclass[a4paper]{article}\usepackage[]{graphicx}\usepackage[]{color}
\usepackage{alltt}
\usepackage{amssymb,amsmath}
\usepackage{palatino}
\usepackage{framed}
\usepackage{parskip}
\usepackage{graphicx}
\usepackage{fixltx2e}
\usepackage[default,osfigures,scale=0.95]{opensans}
\usepackage{geometry}
\usepackage{longtable}
\usepackage{authblk}
\usepackage[colorinlistoftodos]{todonotes}
\geometry{verbose,tmargin=2cm,bmargin=2cm,lmargin=2cm,rmargin=2.5cm}
\linespread{2}
\usepackage{color}

\usepackage[title,titletoc,toc]{appendix}


\usepackage{etoolbox}
\newbool{MyRefNumbers}
\booltrue{MyRefNumbers}

\usepackage{natbib}

\bibliographystyle{ecol_let}
\setcitestyle{authoryear,open={(},close={)}}
\newcommand{\bibstar}{* }

\begin{document}


\clearpage

\title{On the link between functional traits and growth rate: meta-analysis shows effects change with plant size, as predicted}

\author[1]{Anais Gibert*}
\author[1]{Emma F. Gray}
\author[1]{Mark Westoby}
\author[1]{Ian J. Wright}
\author[1]{Daniel S. Falster}
\affil[1]{Biological Sciences, Macquarie University NSW 2109, Australia}

\maketitle

\textit{Running title (36/45 characters)}Influence of traits on growth changes with size

\textit{Corresponding author} Anais Gibert, anais.gibert@gmail.com, Biological Sciences, Macquarie University NSW 2109, Australia, tel: +61 2 9850 8194


\clearpage

\section*{Abstract (195/350 words)}\label{abstract}


\begin{enumerate}

\item
Growth rates (GR) have often been seen as central elements of plant ecological strategies, and as linked to species traits. Yet, the literature is inconsistent about empirical correlations between functional traits and plant growth rate, casting doubt on the capacity of traits to predict growth.

\item
Traits should influence growth in a way that depend on individual plants size. We outline mechanisms and hypotheses based on new theory, and test these predictions in tree species for five traits using a meta-analysis of 112 studies ($>$ 500 correlations). As a first approximation of plants size, we analysed the shift in the correlations between growth and traits across three plant stages (i.e. juvenile, sapling and adult).

\item
Results were consistent with predictions. Specific leaf area was correlated with GR in juvenile but not in adult plants. Correlations of GR with wood density and assimilation rate were not affected by stage. Maximum height and seed mass were correlated with GR only in one plant stage category.

\item
Synthesis: We show that traits-GR correlations change in a predictable way as a function of plant size. Our understanding of plant strategies should shift away from attributing slow vs fast growth to species throughout life, in favour of attributing growth trajectories.

\end{enumerate}

\textit{Keywords (8/10)} asymptotic maximum height, maximum photosynthetic rate, plant ontogenetic stage, plant functional traits, plant growth strategy, seed mass, specific leaf area, wood density


\clearpage

\section*{Introduction}\label{introduction}

The phrase ``plant ecological strategy'' refers to how species face common challenges of acquiring sufficient water, nutrients and light for growth, and then ultimately to replace themselves with offspring. An idea that species strategy differences might involve fast versus slow growth goes back at least to \citet{Grime:1975gr}, who measured a wide spread of relative growth rates for seedlings during early exponential growth under favourable conditions. Seedling potential relative growth rate (RGR) has continued to be used as a major strategy indicator within the CSR (Competitor - Stress tolerator - Ruderal) classification (e.g. \citealt{grime1979plant, Grime:1997wm}). Meanwhile partitioning of seedling RGR into different components showed that specific leaf area (SLA = 1/leaf mass per area LMA) was typically the strongest source of variation between species \citep{Poorter:1989tx}. From the 1990s a new style of ``trait ecology'' arose whereby measurable traits including SLA were used directly as strategy axes \citep{Westoby:2002ft}. An approach via measurable traits made broad comparisons possible across continents and latitudes and thousands of species. Since then, traits have been seen as constructive approaches to understanding plant growth strategies \citep{Grime:1977kc,Chapin:1980gz}, community composition \citep{Lavorel:2002ff,Shipley:2006ie}, global vegetation dynamics \citep{Scheiter:2013ed} and ecosystem processes \citep{Lavorel:2002ff}.

Currently, two main spectra of variation are widely cited as underpinning differences between species in vegetative growth rates. The ``leaf economic spectrum'' \citep{Wright:2004jb} and the ``wood economic spectrum'' \citep{Chave:2009iy} reflect tissue construction costs (for leaf area and for wood volume respectively), trading off with rates of tissue turnover or mortality. It has been widely expected that species with low tissue construction cost will typically have fast growth rates, at least in favourable physical environments \citep[e.g.][]{MullerLandau:2004dc,Wright:2004jb,Poorter:2008iu,Chave:2009iy,Larjavaara:2010bn,Iida:2012jb,Paine:2015df}. Yet, a growing number of empirical studies indicate the correlation between traits and growth is not as consistent as that. Consider SLA as an illustration. Species with high SLA can deploy more canopy area for a given dry mass investment in leaves \citep{Poorter:1999wd, Reich:1992wm}. In addition, SLA is correlated with high photosynthetic rates and high nitrogen (N) or phosphorus concentrations in the leaf \citep{Wright:2004jb}. For seedling relative growth rates under favourable conditions, high SLA has repeatedly been found to be a strong predictor regardless of vegetation type or growth form \citep{Lambers:1992bj,Reich:1992wm,Grime:1997wm,Poorter:1999wd,Wright:1999ds}. Yet for adult plants, the correlation is mostly much weaker or absent \citep{coomes_comparison_1998,Poorter:2008iu,Aiba:2009ft,Easdale:2009gv,Wright:2010tp}. The growing discrepancy between theoretical expectation and empirical results has led some to question whether traits are even useful for understanding growth \citep{Wright:2010tp, Paine:2015df}.

Meanwhile the possibility has been raised that trait-growth correlations may change systematically with size \citep{Falster:2011ii, Ruger:2012jv, Iida:2014ep, Iida:2014hq}. Recent theory suggests that the absence of a correlation between SLA and growth at larger sizes may be expected, when potential influences of traits on plant function are considered \citep{Falster:2011ii}. At larger sizes the low-cost deployment benefits of high SLA may be offset or even outweighed by the high costs of rapid leaf turnover (see section ``Theoretical expectations''). While recent empirical studies supported the idea of a variation in the relationships between traits and growth rates across plant sizes (\citealt{Iida:2014ep, Iida:2014hq}), the trend and mechanisms behind this variation remain unclear. Moreover, the question whether the influence of traits on growth changes with plant size in a predictable and consistent way is very much open.

In this paper, we used a meta-analytic approach on five prominent traits (specific leaf area = SLA, wood or stem tissue density = WD, maximum leaf photosynthetic capacity per unit area = Aarea, asymptotic or maximum height = Hmax, and seed mass = SM) to investigate:
\begin{enumerate}
  \item How general is the tendency for correlations between traits and growth rate to vary with plant size?
  \item  Do the observed relationships conform to predictions (outlined in section ``Theoretical expectations'') derived from a recently developed mechanistic model \citep{Falster:2011ii}?
\end{enumerate}
We gathered correlation coefficients between trait and growth from studies that gave explicit information about the size or stage of the plants. While most studies on traits have not been concerned with effects of plant size, they have generally reported trait-growth correlations for identifiable stage classes. We used plant stage (i.e. juvenile, sapling and adult) as a first approximation of plant size. We focused on tree and woody species since the mechanistic model used to assess our predictions is most applicable to these growth forms.

\section*{Theoretical expectations}\label{theory}

Recent theoretical work \citep{Falster:2011ii} allows us to formulate hypotheses about relationships between traits and growth, based on a mathematical decomposition of plant growth rate. Specific predictions arise from considering the influence of traits on elements of these equations and how these effects may vary with size (see Appendix \ref{sec:growth} for mathematical explanations). This model extends earlier efforts to model growth as a function of traits \citep{Lambers:1992bj,cornelissen_seedling_1996,Wright:2000kw,Enquist:2007ek}. However, whereas earlier models focused solely on mass-based relative growth rate and did not predict a change in effect of traits with size, the revised equations do allow for these effects, while also making predictions for both absolute and relative growth rates in any of height, diameter or mass.

\textbf{H1: The relationship between SLA and growth rate is expected to shift from positive in seedlings to non-significant or negative in adult plants}

Plants with high SLA can deploy more leaf area per unit of mass invested. However, high-SLA species also have shorter leaf lifespan and faster leaf turnover rate \citep{Wright:2004jb}. Whether plant growth increases with SLA depends on the relative magnitude of the leaf construction and leaf turnover effects of SLA (Table \ref{tab:trade-offs}). During early life, when leaf comprises a large part of the plant, decreasing leaf construction cost has a positive effect on growth rate. As plants increase in size and leaf mass fraction decreases, the benefit of cheaper leaf construction diminishes; thus the cost of replacing leaf turnover becomes more important.

\textbf{H2: The relationship between WD and growth rate is expected to be negative across all plant size}

Low WD decreases the cost of building stem tissue and thus the cost of deploying additional leaf area (Table \ref{tab:trade-offs}), which should result in faster growth. Although cheaper wood carries a risk of increased mortality, this cost is not reflected in growth outcomes. Thus decreasing WD is predicted to increase growth across the entire size range.

\textbf{H3: The relationship between Hmax and growth rate is expected to be absent in seedlings but to strengthen with increasing plant size.}

As plants approach maximum height they increasingly divert energy away from growth and into reproduction (see \citealt{Thomas:1996do,Thomas-2011, Wenk:2014jz}). For seedlings, where all individuals are investing solely in growth, we expect no influence of Hmax on growth. As plants become larger, species with smaller Hmax are expected to be allocating more to reproduction and less to growth, because at any given size they are closer to their Hmax (Table \ref{tab:trade-offs}).

\textbf{H4: The relationship between SM and relative growth rate is negative in seedlings, but weakens with increasing size.}

The main effect of increased seed size is to produce a larger overall seedling. Generally relative growth rate decreases with size, and as such larger seedlings are predicted to have a lower relative growth rate during the earliest phases of growth \citep{Turnbull:2012ew}. However, among larger plants we expect the effect of this initial size to be minimal, and thus the correlation between seed size and growth rate should disappear.

\textbf{H5: The relationship between Aarea and growth rate is expected to be positive across all plant sizes, at least under high-light conditions}

We expect greater photosynthetic capacity to translate into faster growth rates by increasing the rate of biomass production (Table \ref{tab:trade-offs}), at least at high light where Aarea is measured. It is uncertain how much this might translate to faster growth in low light.

\textbf{Effect on absolute and relative growth rates}
Tree growth in diameter, height or mass can be expressed in either absolute (AGR) or relative terms (RGR). According to the growth model, the predicted relationship between growth and either SLA or SM in seedlings revert if absolute AGR$_\textrm{mass}$ is used instead of RGR$_\textrm{mass}$ (see Appendix \ref{sec:growth}). For SLA, the correlation with RGR$_\textrm{mass}$ is positive but becomes negative with AGR$_\textrm{mass}$. For SM the correlation with RGR$_\textrm{mass}$ is negative but becomes positive with AGR$_\textrm{mass}$. For the others traits (WD, Hmax and Aarea), the predicted relationships are not affected by the growth measurement used.


\section*{Materials and methods}\label{material-and-methods}

\subsection*{Dataset construction}

We aimed to identify studies that measured the correlation between five prominent functional trait (SLA, WD, Aarea, Hmax and SM) and a growth measurement across species. Our literature search was conducted in two phases.

\subsubsection*{Phase 1 - Literature snapshot}\label{literature-snapshot}

We first developed a snapshot of the literature, to assess potential to recover useufl data from the wide corpus of literature available. This snapshot was obtained via Web of Science database (Thomson Reuters), between March and June 2014, using a range of Boolean search terms in English. Search terms were as follows: (``trait'' AND (``growth rate'' OR ``relative growth rate'')) AND (``tree*'' OR ``woody*''), where ``trait'' was allowed to be any of the trait names listed in Table \ref{tab:lit_search}. This snapshot indicated that 82\% of the reported correlations linking growth to any of our five core traits would be recovered by searching only on SLA and WD, allowing us to narrow our search considerably.

\subsubsection*{Phase 2 - Systematic search}\label{systematic-search}

While Web of Science guarantees a relatively stable and reproducible search environment, with clearly defined lists of indexed journals, its searches are restricted to meta-data, such as title, keywords and abstracts \citep{Beckmann:2012hn}. This limitation does not apply to Google Scholar, where, theoretically, all text that is electronically available on the Internet will be indexed. Thus, we used Google Scholar to perform a more systematic search of the literature for SLA and WD.

The Google Scholar database was searched using a range of Boolean search terms for SLA and WD: ((``SLA'' OR ``LMA'' OR ``specific leaf area'' OR ``leaf mass area'' OR ``wood density'' OR ``WD'' OR ``wood specific gravity'' OR ``stem density'' OR ``stem specific density'') AND (``growth rate'' OR ``relative growth rate'')) AND (``tree*'' OR ``woody*''). This keyword search in Google Scholar returned approx. 18900 papers, from which we took the first 750 items when ranked by ``relevance''.
These papers were examined in addition to those discovered in Phase 1. In addition, we gleaned data from other sources, using our own knowledge as researchers in the field (this included adding data from an as-yet unpublished study by Gleason et al).  The complete list of studies where we inspected the full text is given in Appendix \ref{app:supp_info_files} in Supplementary material.

Each study had to meet some basic criteria to be included in our analysis. We restricted our meta-analysis to documents reporting: i) the relationship between growth and trait as a correlation coefficient, ii) the sample size and the number of species used to perform the correlation, and iii) the size/stage of plants. Ultimately data was included from 112 different articles (starred references in bibliography). Where available, we also recorded the geographic coordinates, number of sites, experiment type (greenhouse, field experiment, forest, shade-house etc.), growth measure (relative or absolute growth rate measured using plant diameter, height or mass), vegetation type, life form (tree, woody or mixed with herbaceous species etc.), biological scale (interspecific, intraspecific or intragenus), and experimental treatments (e.g. fertilization, shade, salt stress). To minimise publication bias via selective reporting of significant results within the published articles (e.g. a particular case of a ``failure to publish bias'' see \citealt{Jennions:2013ta}), we asked those authors reporting a non-significant correlation for additional information about the specific correlation coefficient when these were not provided.

Overall, we recorded coefficient correlations for a list of 20 traits (Fig. \ref{fig:fig1}): the five for which we had prior expectations, plus 15 others commonly used in the trait literature. SLA, leaf area, leaf nitrogen and phosphorus concentration, net (mass) assimilation rates (NAR), and Aarea are traits involved in species leaf economic strategy \citep{Wright:2004jb,Wright:2010tp}. SM and Hmax capture information about reproductive strategy \citep{Falster:2005bw,Moles:2006ft}. WD and sapwood area per leaf area (SA/LA), vessel size and density, and stem specific conductance are all involved in species wood economic strategy  \citep{Chave:2009iy}. Here WD and SA/LA are also considered as indicators of stem construction cost (\citep{Falster:2011ii}, whereas SLA and leaf thickness capture crucial information about leaf construction cost \citep{Wright:2004jb}. Finally, the leaf area ratio and leaf mass ratio are indicators of resource allocation to leaf \citep{Lambers:1992bj}. According to the snapshot of the literature on Web of Science, 62\% of the literature for these 20 traits were captured by a search on SLA and WD.

\subsubsection*{Plant stage categories}\label{plant-stage-categories}

We were unable to obtain a consistent and continuous measure of plant size across all studies, thus we relied on categorical groupings by ``stage'' to test our hypotheses. The authors of the 112 articles used in the meta-analyses, reported their plants as belonging to any of fix different stages: ``seedling'', ``juvenile'', ``sapling'', ``adult'', ``mixed''. While these categories have been established according to the expertise of the authors, by compiling data across multiple studies and vegetation types, we were concerned about the risk of assigning plants of similar size to different categories. This risk turned out to be negligible: overall, we found that the original thresholds in size/age used by the authors to distinguish different stages was relatively compatible (Fig. \ref{fig:fig2}). In particular, there was a clear break between seedlings and saplings in both age (at 1 yr) and height (at 0.5 m); there was also a clear break between saplings and adults in diameter (at 10 cm). According to simple allometric relationships between size variables plotted using data from the Biomass And Allometry Database \citep{Falster:2015}, these thresholds also align relatively well with one another (Fig. \ref{fig:figA1}). Thus we adopted the above values as cut-offs for defining stage boundaries. However, the groups ``juvenile'' and ``sapling'' were largely overlapping, causing us to merge these two.

Using the identified boundaries among stages we were able to assign all studies into one of three classes: ``juvenile'', ``sapling'', or ``adult''.  We preferred the term ``juvenile'' to ``seedling'', because for a portion of the scientific community ``seedling'' also refers to a particular experimental condition. Using the above cut-offs we were also able to assign a stage category to studies for which no stage have been distinctly mentioned by the authors (5\% of the data, noted as ``mix'' in Fig. \ref{fig:fig2}). Only 5 of 112 studies were ambiguous on their plant stage attribution. Half of these equivocal cases involved the juvenile category. We attributed a study to a new stage category only when 95\% of the recorded range of sizes for that study lay within a new stage category. The others equivocal cases were caused by an ambiguity in the definition of ``adult'' used by the authors. Some plants with a large stature are not assessed as adult by the authors because they are still non reproductive (see ID 52 and 50). Here, we attributed a study to the adult stage when the authors identified the plants as reproductive, and to sapling stage when they identified the plants as non-reproductive, regardless the plant size or age. Results using the categories established by the authors are similar than the one using the standardized categories and are available in Fig. \ref{fig:figA5} model.1. In the main text, we report only results using the standardized categories.

\subsection*{Statistical analyses}\label{statistical-analyses}

\subsubsection*{Effect size calculation}

As a measure of effect size, we used the cross-species Pearson product-moment correlation coefficient $r$ between growth rate and trait, weighted by the number of species contributing to the relationship. For studies that reported another coefficient -- such as coefficient of determination, Spearman's rho rank correlation, or Kendall's tau rank correlation -- we converted these to correlation coefficients following \citet{Lajeunesse:2013tm}. As the magnitude of Pearson's correlation coefficient approaches $\pm$ 1.0, its distribution becomes skewed. Accordingly we transformed the correlation $r$ into a metric $z$ with more-desirable statistical properties. The transformation was achieved via Fisher's z-transformation,
\[ z = \frac{1}{2} \ln \frac{1+r}{1-r}, \]
which has a variance estimate of
\[ v_z= \frac{1}{n-3}, \]
where $n$ is the number of species. We weighted effect size by the number of species used to calculation the correlation, instead of the raw number of individuals, because our questions primarily concerned the correlation between growth and traits across species and not within species.

\subsubsection*{Stage effect}

We tested stage effect on traits for which $>$ 15 correlation coefficients across at least two stages were recorded. We used a mixed-effect model of meta-analysis, in which it is assumed that studies within a stage share a common mean effect but that there is also random variation among studies in a stage, in addition to sampling variation. This random effect allows us to account for the heterogeneity among studies (e.g. experimental type, vegetation type, error, etc.). For each trait, we established if stage could explain variance in $z$ by comparing models with the stage variable to the null model using log-likelihood ratios \citep{Zuur:2009cfa}. If the log-likelihood test returned a significant p-value, the stage model fitted the data better than the null model. Estimates of the effect size for each stage were considered significant if the 95\% confidence intervals did not overlap with zero.

\subsubsection*{Sensitivity analysis}

We tested how robust are our results were to the particular decisions made in the analysis, in the criteria used for including a study in the analysis. The number of species used to perform the correlation, the experimental conditions, and the concordance between stage used to measure trait and growth are clear criteria for assessing the quality of the data for addressing our hypotheses. Thus we stratified the data set into two subsets: one including only studies under unstressed conditions ('ideal' dataset), and the other set including studies under all conditions ('complete' dataset). We excluded from the ``ideal'' dataset studies where growth and traits were measured on plants from different stages, unless a particular stage is by definition required to measure a trait such as for SM (i.e. seed stage), or Hmax (i.e. adult stage).
We also tested the sensitivity of our results to inclusion of the few studies that measured a large number of responses for the same trait. In the case of multiple comparisons within a study, we used the method proposed by \citet{Borenstein:2009um}: we calculated the mean correlation coefficient by trait and by study, and then re-calculated the effect size. This produced a reduced dataset, where multiple responses for any given trait were condensed into a study-level average.

\subsubsection*{Quality criteria for meta-analysis}

\cite{Koricheva:2014ku} outlined 14 methodological quality criteria to ensure robust results from meta-analyses. Table \ref{tab:meta} outlines how we address each of these.  In particular, we quantified heterogeneity in effects size, we tested for publication bias and we explored temporal change in effect size (see Appendix \ref{app:supp_info_analyses}).

\subsection*{Availability of code and data}\label{code}

All analyses were conducted with R software \citep{Ralanguageanden:2014wf}. The code and data for producing all figures and results in this paper is available at github.com/AnaisGibert/Growth\_trait\_metaanalysis.

\section*{Results}\label{results}

\subsection*{Main effects}

For both SLA and SM, the correlation between growth rate and the trait changed significantly with plant size in the manner expected. Model fits improved when stage was incorporated (see log likelihood ratio tests, Fig. \ref{fig:fig3} a and d; average values for the coefficients ($\bar{r}$) in Fig. \ref{fig:figA4}). The correlation between growth and SLA shifted from strongly positive at juvenile stage $\bar{r}=0.6 \pm 0.08$, Mean $\pm$ SD) to not significantly different from zero at adult stage ($\bar{r}= -0.01 \pm 0.07$, CI 95\% overlapping zero in Fig. \ref{fig:fig3} a). The correlation between growth and SM shifted from negative at juvenile stage ($\bar{r}= -0.59 \pm 0.07$) to not significantly different from zero at adult stage ($\bar{r}= -0.09 \pm 0.02$, , CI 95\% overlapping zero in Fig. \ref{fig:fig3} d). For both traits, correlation coefficients recorded at sapling stage were intermediate between juvenile and adult stages.

For WD, Hmax and Aarea, correlations with growth were also consistent with expectations, although with limitations arising from very unequal numbers of observations between growth stages for some traits. For Hmax, the expectation was for a correlation with growth to be absent for small plants but present at intermediate and larger sizes, the mechanism being increased relative allocation to reproduction as Hmax is approached. Indeed Hmax was positively correlated with growth at adult stage ($\bar{r}= 0.38 \pm 0.05$, $n=18$). At sapling and juvenile stages, the correlation between Hmax and growth was not significantly different from that observed at adult stage (model fit not improved when stage was incorporated, see log likelihood ratio tests in Fig. \ref{fig:fig3} c), but also was consistent with no correlation (Fig \ref{fig:fig3} c, confidence intervals overlapping zero), due to much smaller numbers of replicate studies. Similarly for WD and for Aarea, results were consistent with negative (WD) and positive (Aarea) correlations with growth across all plant sizes (Fig. \ref{fig:fig3} b and e, model fit not improved when stage was incorporated). However, at the same time confidence intervals of r often spanned zero especially where there were small numbers of studies for particular growth stages.

\subsection*{Coverage of dataset}

Given our basic conclusion that reported correlations responded to plant size in a manner consistent with expectations from theory, we assess now the coverage of the available data, to determine if biased sampling could have erroneously generated the above results.

In total, we found 112 studies published between between 1983 and 2014 that met our requirements. Climate zones spanned from temperate regions to  tropics (map in Fig. \ref{fig:figA2}). Different experiment types (laboratory, field, plantation or greenhouse), growth measurements and growth form (tree, woody or across growth form) were included. Number of species used for each correlation ranged from 2 to 300 species with average 31. Given this variety, it was not surprising to find wide ranges of $r$ reported for all traits. For instance, $r$ varied from -0.98 to 0.99 for SLA, and from -0.74 to 0.74 for WD (Fig. \ref{fig:figA3}). The growth form, the experiment types and the growth measurement were not markedly best predictors of the data (results shown in the Fig. \ref{fig:figA5}).

The results reported above were confined to correlations across species, and under non-stressed conditions ($n = 176$). A ``complete'' dataset ($n = 216$) also included correlations measured under stressed conditions, and spanning different plant stages between the growth and trait measurements. This ``complete'' dataset generated similar effect sizes to those reported above from the ``ideal'' dataset, and led to similar conclusions (Fig. \ref{fig:fig3} grey points). Differences primarily concerned the confidence intervals around particular mean effect sizes in Fig. \ref{fig:fig3}. These differences in CI were not easily attributed to any single cause. For instance, the correlation between growth and WD at juvenile stage shifted from negative to non significant (i.e CI 95\% included zero) due to 2 positive correlation coefficients from \citet{Augspurger:1984ct};  these correlations unlike others used values of WD from adult plants together with growth from juveniles. For Aarea, the comparison between complete and ideal data showed no signal of an effect of stress (here mainly low light) on the correlation with growth at seedling stage; the correlation was still positive between Aarea and growth even when 7 correlations measured under low light conditions were added. Finally, our results were unlikely to be biased by dominance of one or a few studies that measured a large number of responses, since when multiple responses were condensed into a single average within a trait and within a study, outcomes were very similar (Fig. \ref{fig:figA5}).

Overall our study met all the quality criteria required for meta-analysis in ecology (Table \ref{tab:meta}). We showed that the impact of publication bias is probably trivial in our meta-analyses (Appendix D, Fig. \ref{fig:figA7}). The extent of the heterogeneity was high in our meta-analysis,  but substantially explained by the plant stage effect for SLA and SM, as expected (Appendix D). Finally, there was no indication of a temporal change in the effect size being reported within stages (Appendix D, Fig. \ref{fig:figA9}).


\section*{Discussion}\label{discussion}

%TODO please insert refs below
The meta-analysis reported here demonstrates that for several key traits the influence of traits on growth rates changes as plants grow from small to large. While this point has been made recently in smaller scale studies (refs??), it is the first time that mechanistic hypotheses are confirmed across a large number of studies already available. Meta-analysis allowed us to generalize and thus to reach a consensus about the effect of traits on growth. Growth rates are known to change with size through ontogeny \citep{Condit:1993hd, Clark:1999ed, Herault:2011dd}, \textit{i.e. plants follow a growth trajectory}; species traits should be understood as influencing this trajectory and potentially re-ranking growth rates with size. This is a shift in view from the current approach in plant strategy schemes, whereby traits are thought to render species fast-growing or slow-growing throughout life.

\subsection*{Why does the effect of traits on growth vary with size?}

Further, for five prominent traits, the patterns observed are consistent with specific predictions from a mechanistic growth model. This indicates provisionally that the mechanisms through which traits affect growth are correctly understood in the model. For SLA and for SM the meta-analysis confirms quite strongly the predicted influence of plant size on the trait-growth correlation. For WD, Aarea and Hmax results were consistent with predictions, but coverage was limited and additional studies would be helpful for some trait-growth-stage relationships.

It has previously been noted that the correlation of SLA with growth disappears among saplings and larger trees (\citealt{Wright:2010tp} for Barro Colorado Island forest, Panama), and this meta-analysis generalises that result. Several authors have suggested that the disappearance of SLA's influence on growth might be due to a disproportionate accumulation of leaf area by low SLA species \citep{Reich:1992wm,Poorter:2008iu,Wright:2010tp}, yet it remains unclear whether such a mechanism works as intended. More importantly, the assumed empirical relationship between SLA and total leaf area does not hold true across large number of species (Duursma and Falster, in review). In the model described here, the mechanism is slightly different: high-SLA species incur faster leaf turnover, the cost of which outweighs the benefits of cheap leaf construction at larger sizes.

For seedlings our meta-analyses showed, as expected, a negative relationship between SM and relative growth rate (in mass or height or volume) for small plants, that dissipated as plants grew larger. This negative relationship arises because seedlings from larger seeds are begin life larger themselves, as has been extensively discussed in the literature \citep[reviewed by][]{Turnbull:2012ew}. moreover, the prediction applies only to RGR not to AGR (Appendix Fig. \ref{fig:figA8} d).

For WD the meta-analysis showed negative correlation with growth across all plant stages, as previously reported from individual studies \citep{Wright:2010tp,Ruger:2012jv}. It should be noted however that there are few tests of this proposition at seedling stage. In our model (Appendix \ref{sec:growth}) cheaper stem construction is responsible for the predicted negative correlation. Lower xylem conductance and associated lower photosynthetic potential have also been suggested \citep[reviewed by][]{Chave:2009iy}.

As far as we are aware, our study is the first to compared correlations between Aarea and growth among small and large plants. In the growth model (Appendix \ref{sec:growth}), a consistent correlation irrespective of size arises from a direct effect of Aarea on plant growth rate. In juvenile stage, where most of the data were, we noted no signal of an effect of low light on the positive correlation between Aarea and growth rate.

Meta-analysis showed a positive correlation between Hmax and growth for adult stage, but not for saplings and juveniles. The trend for tall stature species to grow faster at a given tree size than small stature species has been noted across several tree size classes and forest types (e.g. \citealt{Poorter:2008iu,Wright:2010tp,Herault:2011dd,Ruger:2012jv,Iida:2014ep}). In \citet{Wright:2010tp}, Hmax was considered as the weakest predictor of growth rates of saplings; however it was the strongest predictor of growth rates of large trees. \citet{Poorter:2008iu} speculated that the Hmax-RGR relationship among large trees might arise from including small tree species with maximum size not much larger than the  10cm dbh lower threshold used to define large trees. In our growth equations, Hmax affects growth via the fraction of energy diverted to reproduction. At a given tree size, species closer to their Hmax are expected to be allocating relatively less to vegetative growth. However this applies only at sizes where some of the species have begun to reproduce, and should not affect sapling and juvenile growth rates.

\subsection*{No signal of cross-correlations among different traits}

It is notable that the expected correlations between traits and growth do not appear to be much overridden by cross-correlations among different traits. A recurring issue in trait ecology is how many of the observed correlations arise directly from mechanisms, versus how many from secondary cross-correlation. For example, several papers have suggested that correlations between Hmax and growth rate may arise via cross-correlation with other traits such as WD \citep{Thomas:1996do, Poorter:2006vb, Wright:2010tp}. Similarly, effects of SM on growth have been suspected to be mediated via other traits \citep{Shipley:1990js,Westoby:2002ft,Poorter:2006vb}. Secondary cross-correlation might also potentially have the effect of suppressing a correlation that would otherwise be present. While such effects may still be observed within individual studies, across the range of studies reported we found that the available data were consistent with predictions arising directly from mechanisms.

\subsection*{Future research directions}

Light and water micro-environments can affect growth rates, and might modify correlations with traits in relation to plant size. For example, instead of a positive relationship that disappeared with plant size, \citet{Iida:2014ep} found SLA was negatively correlated with RGR at small size and positively correlated with RGR at larger sizes. Similarly, instead of a negative correlation between WD and growth across all plant stages, \citep{Iida:2014ep,Iida:2014hq} found a negative correlation at small size that disappeared when trees were larger. \citet{Iida:2014hq} explained the switch in directionality by changes in light and water condition up the vertical profile within a forest: larger trees reached a position in the canopy where they were released from light limitations, other resources such as water may have become limiting, and the performance of trees may then have been regulated by different traits. Note that in \citet{Iida:2014hq}, the shift in the correlation between growth and WD occurred for trees larger than 22-26 cm dbh. This points to one of the possible limitations in our meta-analysis, where trees with height above 5 m were pooled as ``adult''.

Clearly there is much scope for strengthening our understanding of correlations between traits and growth through (1) adding further empirical studies for traits and size-classes with weaker representation, (2) accounting more satisfactorily for variables such as light, and (3) expressing plant sizes on continuous scales of mass or height rather than through the approximate and partly subjective categories unavoidably adopted here. Additional data are specifically required in small plants (juvenile and sapling) for WD and Hmax, and in large plants (adult and sapling) for Aarea and more broadly for all traits requiring a measure of whole plant biomass (e.g. LMR, LAR, NAR). Nevertheless the existing published evidence, even with all its limitations, is consistent with predictions from a simple growth model, and this suggests that species traits can shed light on their growth trajectories, and hence on the changing growth rate rankings across species as individuals become larger.

\section*{Acknowledgments}\label{Acknowledgment}

We thank Andrea Stephens for helpful comments on the meta-analyses, and Sean Gleason et al for pre-publication access to their data. Charles Warren and Jordi Martìnez-Vilalta kindly provided additional information about their published results. This work was funded by the Australian Research Council through a fellowship to Westoby and a discovery grant to Falster.

\clearpage
\linespread{1}

\nocite{*}

\bibliography{output/refs}\label{references}


\clearpage
\section*{Figures}

\begin{figure}[h!]
\centering
\includegraphics{output/Fig1.pdf}
\caption{Number of recorded correlations between trait and growth rate in our dataset, byt size and trait. Results shown here are for the ``complete'' dataset (recorded both across and within species, under unstressed and stressed conditions). Traits are: Aarea (or Amass) = rate of CO$_{2}$ assimilation per unit leaf area (or leaf mass) = maximum photosynthetic rate, Hmax = asymptotic maximum height,  Ks = stem hydraulic conductance, LA = leaf area, LMR = leaf mass ratio, NARarea (or NARmass) = net mass assimilation rate per unit of leaf area (or leaf mass), Nmass (or Narea) = nitrogen content per unit of leaf mass (or leaf area), Pmass = phosphorus content per unit of leaf mass, SA/LA = sapwood area per leaf area, SLA = specific leaf area, thickness = leaf thickness, Vessel size = stem conduit size, vessel density = stem conduit density, WD = wood density.}
\label{fig:fig1}
\end{figure}

\begin{figure}[h!]
\centering
\includegraphics{output/Fig2.pdf}
\caption{The range of plant sizes and stage recoded in original publications, compared to standardised thresholds used in the analysis. Panels are for
\textbf{a)} age, \textbf{b)} height and \textbf{c)} diameter. Line segments indicate the size range (max., min.) for the plants used to measure growth. Article IDs correspond to the reference number in our dataset. Colours denote stage information given by the authors, while line styles denote growth form. The vertical lines indicate the thresholds used in our analysis. Results shown here are the ``complete'' dataset (see methods for details).}
\label{fig:fig2}
\end{figure}

\begin{figure}[h!]
\centering
\includegraphics{output/Fig3.pdf}
\caption{The relationship between growth rates and traits changes with plant stage and trait. Plots show the effect size (+ 95\%CI) for the average correlation across species, by plant stage. Effect size is a standardized measure of the deviation of correlation coefficient from zero (Fisher's z-transformed). Effects are significant if confidence intervals (95\%CI) for a given stage do not overlap with zero. Results shown are from both the ``complete'' (black)  and ``ideal'' (grey) datasets. Trait names as in Fig. \ref{fig:fig1}. $n$ indicates number of correlations contributing to each effect. Likelihood Ratio Tests (LRT, for ``ideal'' dataset only) compared maximum-likelihood fits between a pooled model and one separated by stage. A significant p-value means the model including stage effect fitted the data better than the null model.
}
\label{fig:fig3}
\end{figure}

\clearpage
\section*{Table}
\setcounter{table}{0}
\begin{table}[h!]
\centering
\caption{\textbf{Hypothesised effects of traits on key elements of plant function determining growth rate.} Arrows indicate the effect an increase in trait value would have on the main elements of growth eqs. \ref{eq:eq0}-\ref{eq:eq4} in Appendix \ref{sec:growth}. Traits are SLA: Specific leaf area, WD: Wood density, Hmax: Asymptotic maximum height, SM: seed mass, Aarea: Maximum photosynthetic rate. For further details, see  Appendix \ref{sec:growth}.}
\vspace{1cm}
  \begin{tabular}{cccc}
   & Leaf deployment per mass & Allocation to vegetative growth & Net live biomass production \\
   Trait
    & ($\frac{\textrm{d}a_\textrm{l}}{\textrm{d}m_\textrm{t}}$)
    & ($\frac{\textrm{d}m_\textrm{t}}{\textrm{d}b}$)
    & ($\frac{\textrm{d}b}{\textrm{d}t}$) \\ \hline
  SLA &$\nearrow$ &$\_$ & $\searrow$  \\
  WD & $\searrow$ &$\_$  &$\_$ \\
  Hmax &$\_$ &$\nearrow$ &$\_$ \\
  SM &$\_$&$\_$&$\_$ \\
  Aarea &$\_$ &$\_$ & $\nearrow$ \\
\hline
  \end{tabular}
\label{tab:trade-offs}
  \end{table}


%TODO: I suggest keeping this table, but merging the two right hand columns. The text in the explanation column mostly just repeats material that is presented elsewhere, so best to just say briefly where to find it.
\clearpage
\begin{table}[h!]
\centering
\caption{\textbf{Check-list of quality criteria for meta-analysis in ecology} developed by Koricheva and Gurevitch (2014) }
\vspace{0.5cm}
\begin{tabular}{p{3cm} p{1cm} p{10cm} p{3cm}}
  \hline
  Quality criteria & Check & Explanations & Addressed in \\
  \hline
  1. Reporting full details of bibliographic searches & yes	& Google Scholar database was searched using the range of Boolean search terms ((``SLA'' OR ``LMA'' OR ``specific leaf area'' OR ``leaf    mass area'' OR ``wood density'' OR ``WD'' OR ``wood specific gravity'' OR ``stem density'' OR ``stem specific density'') AND (``growth rate'' OR ``relative growth rate'')) AND (``tree*'' OR ``woody*''), in English. & ``Systematic search'', Table \ref{tab:lit_search} \\
  2. Reporting inclusion/exclusion criteria & yes & Studies included report i) correlation coefficient between growth and trait, ii) the sample size/ number of species, and iii) the size/stage of plants. Studies excluded report correlation within species	& ``Systematic search'' \\
  3. Weighting effect sizes by study precision & yes & Effects sizes were weighted by number of species. The question addressed in our study concerns the correlation between growth and trait across species, thus the number of species is a better proxy of a study quality than sample size. &	 ``Systematic search'' and  ``Effect size calculation'' \\
 4. Specifying meta-analytical model &	yes	& Mixed-effect model with a study ID as a random variation, and stage as a fixed effect, m1 ~lmer (stage + (1|ID)) &	``Statistical analyses'', Fig3 \\
 5. Quantifying heterogeneity in effect sizes &	yes &	We estimated the extend of the heterogeneity using the statistics $I^{2}$ (Santos and Nakagawa, 2012) for each trait. &	Appendix D \\
 6. Exploring causes of heterogeneity &	yes &	We explored two causes of variability: the stage and the year of publication & Appendix D, Fig3 \\
 7. Multifactorial analysis of explanatory variable	& yes & We focused on the ``plant stage'' as the main exploratory variable. Yet, the stage variable may be confounded with other explanatory moderators (i.e. growth measurement, experiment type or growth form). Thus, we checked if no one of them would not consistenly be a strongest predictor of the data. We did not test the multiple variables in a hierarchical fashion, because of the limited size of our dataset. & FigA4, FigA6  \\
  8. Testing for publication bias &	yes	& We used funnel plots for a visual assessment. We used the ``trim and fill'' method to estimate the number of studies missing (Duval and Tweedie 2000). We also estimated the Rosenberg’s fail-safe number (Rosenberg 2005). 	&  Appendix D, FigA7\\
  9. Sensitivity analysis &	yes &	We tested how robust are our results to the decision made in the analysis. We calculated effect size using 3 differents dataset: i) an  ``ideal'' dataset, focusing on unstressed conditions exclusively, ii) a ``complete'' dataset reporting all conditions, and iii) a ``reduced dataset'' where multiple responses were condensed into a single average within a trait and within a study. &	Fig3, FigA5 \\
  10.Exploring temporal changes in effect size &	yes	& There may be a confounding effect between the publication year and the stage. Thus, we explored the publication effect on the effect size within each stage for traits where a stage effect had been highlighted (SLA and SM). & Appendix D, FigA9 \\
  11. Controlling for phylogeny &	no &	It was not possible to control by the phylogeny since the number of species used to measured each coefficient of correlation ranged from 2 to 300 species. &	\\
  12. Specifying the software used &	yes	& R software, code available on Github. & ``Availability of code and data'' \\
 13. Providing reference list of primary studies included in the analysis &	yes &	Available on Github. The Bibtex file `meta-analyses.bib` corresponds to all the articles included in the meta-analyses, `read.bib` corresponds to all the articles (included and excluded) inspected for the meta-analyses.	& ``Availability of code and data'' \\
  14. Providing the data set used for meta-analysis  &	yes &	A copy of the complete data set is available on Github.	& ``Availability of code and data'' \\
   \hline
\end{tabular}
\label{tab:meta}
\end{table}


\clearpage
\begin{appendices}\label{sec:appendices}
\renewcommand{\thefigure}{S\arabic{figure}}
\renewcommand{\thetable}{S\arabic{table}}

\setcounter{figure}{0}
\setcounter{table}{0}

\section{Derivation of hypotheses from growth equations}\label{sec:growth}

We adapted growth equations from \cite{Falster:2011ii} to generate hypotheses on both absolute and relative growth rates in mass (eqn \ref{eq:eq1}), height (eqn \ref{eq:eq2}) and diameter (eqn \ref{eq:eq4}).

The rate at which \textbf{live biomass} $b$ is produced by a plant is given by:

\begin{equation}\label{eq:eq0}
\underbrace{\frac{db}{dt}}_{\text{net live biomass production}} = \underbrace{y}_{\text{yield}}  \big(\underbrace{a_{l} A_{\textrm{area}}}_{\text{photosynthesis}} - \underbrace{\sum_{i = l,s,r,f} m_{i} r_{i}}_{\text{respiration}}\big) - \underbrace{ \sum_{i = l,s,r,f} m_{i} k_{i}}_{\text{turnover}},
\end{equation}

where $y$ is the ratio of carbon fixed in mass per carbon assimilated,  and  $A_{\textrm{area}}$ is the assimilation rate of CO$_{2}$ per leaf area.

Here, and in the following equations, $m_i$ denotes the \textbf{live mass on the plant}, $r_i$ the mass-specific \textbf{respiration rate}, and $k_i$ the \textbf{turnover rate} of different plant tissues; the subscripts $i = l,s,r,t,f$ referring to leaves, sapwood and bark, roots, total of these vegetative tissues, and to the reproductive tissues (fruits, flowers etc.) (note that the live biomass $b = m_t +m_f$). Similarly $a_i$ denotes \textbf{areas}, of leaves ($i = l$) and of cross-sections of total stem and sapwood ($i = st,ss$) respectively.


The \textbf{growth rate in mass}, being the increment in vegetative live mass ($m_t$), is then:

\begin{equation}\label{eq:eq1}
\underbrace{{\frac{dm_t}{dt}}}_{\text{growth rate in mass}} = \underbrace{\frac{dm_t}{db}}_{\text{allocation to vegetative growth}}  \times \underbrace{\frac{db}{dt}}_{\text{net live biomass production}},
\end{equation}

where allocation to growth is the fraction of net live biomass increase that goes to vegetative growth rather than to reproductive tissue production.

The \textbf{growth rate in height} can then be expressed as:

\begin{equation}\label{eq:eq2}
\underbrace{\frac{dH}{dt}}_{\text{growth rate in height}} = \underbrace{\frac{dH}{da_{l}}}_{\text{architecture layout}} \times \underbrace{{\frac{da_{l}}{dm_{t}}}}_{\text{leaf deployment per mass}} \times {\frac{dm_t}{db}} \times {\frac{db}{dt}}.
\end{equation}

Similarly \textbf{growth rate in basal area} can be expressed as:

\begin{equation}\label{eq:eq3}
\underbrace{{\frac{da_{st}}{dt}}}_{\text{basal area growth rate}} = \underbrace{\frac{da_{ss}}{da_{l}}}_{\text{sapwood area per leaf area}} \times {\frac{da_{l}}{dm_{t}}} \times  {\frac{dm_t}{db}} \times {\frac{db}{dt}} +  \underbrace{ \frac{da_{st}}{dt}}_{\text{sapwood turnover}}.
\end{equation}

The \textbf{growth rate in diameter} is then given by the geometric relationship between diameter and area applied to stem:
\begin{equation}\label{eq:eq4}
\underbrace{\frac{dD}{dt}}_{\text{growth rate in diameter}} = \sqrt[]{\frac{\Pi}{a_{st}}} \times {\frac{da_{st}}{dt}}.
\end{equation}

For the case of seedlings, eqs. \ref{eq:eq0}-\ref{eq:eq1} can be simplified considerably. Turnover of all tissues and allocation to reproduction have not yet begun. It is also common to assume all respiration rates are proportional to leaf area. Biomass growth rate in seedling then becomes a linear function of leaf area:

\begin{equation}\label{eq:eq1_seedlings}
{\frac{dm_t}{dt}}  \approx  a_{l} \, y \, (A_{\textrm{area}} - r_{l}).
\end{equation}

The \textbf{relative growth rate in mass} $(\textrm{RGR}_{\textrm{mass}})$ in seedlings can be obtained by dividing both sides of this simplified eq \ref{eq:eq1_seedlings} by the total mass of living tissues $m_{t}$. The outcome is equivalent to the classical partitioning of RGR \citep{Lambers:1992bj, Cornelissen:1998ta}:
\begin{equation}\label{eq:eq1_seedlings_RGR}
\textrm{RGR}_{\textrm{mass}}  \approx \textrm{NAR} \times  \textrm{SLA} \times  \textrm{LMF},
\end{equation}
 where net assimilation rate NAR = $y \, (A_{\textrm{area}} - r)$, SLA = $a_{l}/ m_{l}$ and leaf mass fraction LMF = $m_{l}/ m_{t}$.


The predicted effects of different traits on growth can then be understood by considering the influence of trait variation on elements of equations  \ref{eq:eq0}-\ref{eq:eq1_seedlings_RGR} (see Table \ref{tab:trade-offs}) and how these may interact with plant size.


\textbf{Specific leaf area (SLA):} For seedlings (eq. \ref{eq:eq1_seedlings_RGR}), cheaper leaf construction (high SLA) should translate directly into faster relative growth rate in mass. For larger plants (eq. \ref{eq:eq1}-\ref{eq:eq4}), SLA is expected to affect growth rate positively through more cost-efficient deployment of leaf area (via $da_l/dm_t$; the same effect as in seedlings), but also negatively via increased leaf turnover (eqn 1; Table \ref{tab:trade-offs}). The effect of SLA on $da_l/dm_t$ decreases as plant size increases and increasing amounts of the plant are wood. Consequently the effect of turnover costs in slowing growth comes to dominate at larger sizes.

\textbf{Wood density (WD):} Cheaper stem construction (low WD or stem density) increases the marginal rate of leaf deployment per total mass increment ($da_l/dm_t$), and consequently increases growth rate in leaf area, height and diameter (eqs. \ref{eq:eq2}-\ref{eq:eq4}; Table \ref{tab:trade-offs}). Disadvantages associated with cheap stem construction take the form of higher mortality and so are not reflected in growth equations. Under this scenario, wood density is negatively correlated to growth rate across all plant stages. Because the relative expenditure on stem compared to leaf is increases as plants grow larger, the relationship between WD and growth may also strengthen in larger plants.

\textbf{Maximum height (Hmax):} Hmax affects growth via the fraction of energy allocated to vegetative growth versus to seed production ($dm_t/db$ in eq. \ref{eq:eq1}; Table \ref{tab:trade-offs}). For juveniles and seedlings this fraction is 1 for all species and so no effect is expected. But once some species begin to produce seed, then species closer to their Hmax value are expected to be allocating more to seed production and less to vegetative growth. Hence when comparing across plants at a given size, species with smaller Hmax are expected to show slower vegetative growth rates.

\textbf{Seed mass (SM):} The $\textrm{RGR}_{\textrm{mass}}$ slows as plants grow larger, and large-seeded species begin larger as seedlings. For larger plants, SM is not expected to have any distinct influence on growth rates  (Table \ref{tab:trade-offs}).

\textbf{Potential photosynthesis rate (Aarea):} The effects of benefits (higher assimilation) and costs (higher respiration) of increased Area occur within the same element of eq. \ref{eq:eq0}, and assimilation must outweigh respiration for there to be growth, so the net effect will be positive (Table \ref{tab:trade-offs}).

Note that the predicted relationship between growth and either SLA or SM in seedlings differs if absolute growth rate in mass is used instead of RGR$_\textrm{mass}$ (i.e. becomes negative for SLA, positive for SM). In our study, all data recorded for these trait-stage combinations have been measured on a RGR base for mass (see Fig S9).


\clearpage

\section{Supporting files}\label{app:supp_info_files}

The code and data for producing all figures and results in this paper is available at github.com/AnaisGibert/Growth\_trait\_metaanalysis.  Upon publication, a copy of the complete data set is also available from the Dryad Digital Repository.

\section{Supporting tables}\label{app:supp_info_tables}

\linespread{1}

\begin{table}[h!]
\centering
\caption{\textbf{List of the twenty traits targeted}. Abbreviation,
definition and search term used in the meta-analysis}
\label{tab:lit_search}
\vspace{0.5cm}
\begin{tabular}{p{3cm}p{3cm}p{8cm}}
  \hline
Trait & Definition & Search terms \\
  \hline
Aarea & Photosynthetic rate per area & ``photosynthetic rate per area'' OR ``rate of CO2 assimilation per area'' OR ``Aarea'' OR ``Amax'' OR ``photosynthetic capacity per area'' \\
  Amass & Photosynthetic rate per mass & ``photosynthetic rate per mass'' OR ``rate of CO2 assimilation per mass'' OR ``Amass'' OR ``Amax'' \\
  Hmax & Maximum height & ``potential height'' OR ``maximum height'' OR ``Hmax'' \\
  Ks & Stem specific conductance & ``Ks'' OR ``stem specific conductance'' OR ``stem hydraulic conductance'' OR ``sapwood conductance'' \\
  LA & Leaf lamina area & ``leaf length'' OR ``leaf lamina area'' OR ``leaf width'' OR ``leaf size'' \\
  LAR & Leaf area ratio & ``leaf area ratio'' \\
  LMR & Leaf mass ratio & ``leaf mass ratio'' OR ``leaf mass fraction'' \\
  NARarea & Net assimilation rate per area &  ``net assimilation rate per area'' OR ``rate of dry mass increament per area'' OR ``NARarea'' OR ``NAR'' \\
  Narea & Leaf nitrogen content per area & ``Narea'' OR ``nitrogen per area'' OR ``N/area'' OR ``leaf nitrogen content per area'' OR ``LNCarea'' OR ``leaf nitrogen content" \\
  NARmass & Net assimilation rate per mass &  ``net assimilation rate per mass'' OR ``rate of dry mass increament per mass'' OR ``NARmass'' OR ``NAR'' \\
  Nmass & Leaf nitrogen content per mass & ``Nmass'' OR ``nitrogen per mass'' OR ``N/mass'' OR ``leaf nitrogen content per mass'' OR ``LNCmass'' OR ``leaf nitrogen content" \\
  Parea & Leaf phosphorus content per area & ``Parea'' OR ``leaf P'' OR  ``leaf phosphorus content per area''  OR ``leaf phosphorus content'' \\
  Pmass & Leaf phosphorus content per mass & ``Pmass'' OR ``leaf P'' OR  ``leaf phosphorus content per mass'' OR ``leaf phosphorus content'' \\
  SA/LA & Sapwood area per leaf area & ``sapwood area per leaf area'' OR ``huber value'' OR ``leaf area per sapwood area'' \\
  Seed mass & Seed mass & ``seed mass'' OR ``seed size'' OR ``seed volume'' \\
  SLA & Specific Leaf Area & ``leaf mass per area'' OR ``specific leaf area'' OR ``leaf construction cost'' \\
  Thickness & Leaf thickness & ``leaf thickness''  OR ``leaf tissue density'' \\
  Vessel density & Vessel density & ``vessel density'' OR ``stem conduit density'' \\
  Vessel size & Vessel area or diameter & ``conduit area'' OR ``stem conduit area'' OR ``vessel diameter'' OR ``vessel size'' OR ``conduit size''   \\
  WD & Wood density & ``wood density'' OR``WD'' OR ``wood specific gravity'' OR ``stem density'' OR ``stem specific density'' OR ``SSD'' \\
   \hline
\end{tabular}
\end{table}


\clearpage
\section{Supporting figures}\label{app:supp_info_figures}


\begin{figure}[h!]
\centering
\includegraphics{output/FigA1.pdf}
\caption{Allometric relationships between a) age and diameter (m), b) age and height (m), and c) diameter and height in trees according to the BAAD database. Green and red lines are the cutoffs between stage categories used in the literature. Green lines corresponds to limits between juveniles and saplings, red lines to limits between saplings and adults.}
\label{fig:figA1}
\end{figure}


\begin{figure}[h!]
\centering
\includegraphics{output/FigA2.pdf}
\caption{Geographical distribution of site locations from studies performed in the field used in the meta-analysis. Results shown here are from the ``complete'' dataset (see methods for details).}
\label{fig:figA2}
\end{figure}


\begin{figure}[h!]
\centering
\includegraphics{output/FigA3.pdf}
\caption{Range of correlations (r) between
growth and trait reported in the literature (112 articles). Plots shown the coefficients of correlation (r)
between the plant growth and \textbf{a)} Specific leaf area (SLA, n =
111), \textbf{b)} Wood density (WD, n = 64), \textbf{c)} Asymptotic
maximum height (Hmax, n = 23), \textbf{d)} Seed mass (Seedmass, n = 36 and
\textbf{e)} Rate of CO2 assimilation (Aarea, n = 24) reported from 112
articles. The x abcissa represents the articles ID sorted by the average coefficient value r reported.
Results shown here are from the ``complete'' dataset (recorded both across and within species
and under unstressed and stressed conditions). Each bubble represents a r value, the diameter of the bubble
is proportional to its weights in the meta-analysis (i.e. number of species used). The colors
correspond to the stage for which the r values have been recorded: red
points corresponds adult stage, orange points to saplings, and grey
points to juveniles.}
\label{fig:figA3}
\end{figure}

\begin{figure}[h!]
\centering
\includegraphics{output/FigA4.pdf}
\end{figure}

\begin{figure}[h!]
\centering
\includegraphics{output/FigA4b.pdf}
\caption{\textbf{Mean coefficient correlation r (+SD) between growth rate and traits, established from 112 articles.} The mean coefficients are weighted by sample size (i.e number of species used to perform the correlation), and calculated both across all plant stages and for each stage. Each panel corresponds to a functional trait: \textbf{a)} Specific leaf area (SLA), \textbf{b)} Wood density (WD), \textbf{c)} Net assimilation rate (NARarea), \textbf{d)} Seed mass, \textbf{e)} Asymptotic height (Hmax) and \textbf{f)} Assimilation rate of CO2 (Aarea). Results shown here are from the ``unstressed dataset'' (recorded across species and under unstressed conditions).}
\label{fig:figA4}
\end{figure}

\begin{figure}[h!]
\centering
\includegraphics{output/FigA5.pdf}
\caption{\textbf{Effect size (+ 95\%CI) of plant stage on the correlations between growth rate and five functional traits across species, with a data set averaging multiple comparisons by trait and by study.} Effects size are calculated from a reduced dataset; in the case of multiple comparisons within a study, we calculated the mean correlation coefficient by trait and by study. a) Specific leaf area (SLA), b) Wood density (WD), c) Net assimilation rate (NARarea), d) Seed mass, e) Asymptotic height (Hmax) and f) Assimilation rate of CO2 (Aarea). Effect size is a standardized measure of the magnitude of the relationship between a particular stage and the correlation coefficient z (Fisher's z-transformed). Effects are significant if confidence intervals (+CI95\%) do not overlap with zero. LRT: likelihood Ratio Test compared the fit between null model and stage model, both models are fitted by Maximum Likelihood. Black point: ideal data (under unstressed conditions), grey points: all the data. ``n'' indicate the number of independent comparisons for each effect.}
\label{fig:figA5}
\end{figure}


\begin{figure}[h!]
\centering
\includegraphics{output/FigA6a.pdf}
\end{figure}

\begin{figure}[h!]
\centering
\includegraphics{output/FigA6b.pdf}
\end{figure}

\begin{figure}[h!]
\centering
\includegraphics{output/FigA6c.pdf}
\caption{\textbf{Effects size on the correlation between growth and traits of plant stage (mod1), vegetation type (mod2), growth measurement (mod3) and experiment type (mod4).} The effect size (z + 95\%CI) is reported for each class, and models are fitted by restricted max likelihood (REML). n = number of correlation r reported. Abbreviation: across veg = more than 2 vegetation types, boreal\&temp = boreal and temperate deciduous forest, med = Mediterranean,  temp\&med = temperate deciduous and Mediterranean forests, temp = temperate deciduous, temp rain = temperate rain, trop rain = tropical rain, trop seas = tropical seasonal forests, GR = growth rate and RGR = relative growth rate, measured in diameter (D), height (H), mass(M), cross section area (CSA). AIC: Akaike information criterion corresponds to the quality of each model, and allow a comparison between models. Smallest is the AIC better is the model. Results shown here are from ``ideal'' dataset (recorded at interspecific and under unstressed condition) and ``complete'' dataset (recorded at both interspecific and intraspecific level and under unstressed and stressed conditions).}
\label{fig:figA6}
\end{figure}

\begin{figure}[h!]
\centering
\includegraphics{output/FigA7.pdf}
\caption{\textbf{Funnel plots with filled in data based on the trim and fill method, showing transformed correlation coefficient z between growth and the traits a) Specific leaf area (SLA), b) Wood density (WD), c) Asymptotic height (Hmax), d) Seed mass (SM) and e) Assimilation rate of CO2 (Aarea)}. Results shown here are from the ``ideal'' dataset (recorded across species and under unstressed conditions).  Note that in these plots the y axis is the 1/number of species included in the correlation, not the replication for the correlation. The number of species is sometimes as low as 1, and these are within-species correlations. The number of replicates used for the correlation coefficient was often much greater than the number of species. Nevertheless, the characteristic shape of funnel plots is observed, with correlations involving few species showing much wider variation even though within-species replication was sometimes high. The open points correspond to the estimated missing studies. The number of missing values is 9 for SLA,1 for WD, 0 for Hmax, 6 for SM and 0 for Aarea}
\label{fig:figA7}
\end{figure}

\begin{figure}[h!]
\centering
\includegraphics{output/FigA8.pdf}
\caption{\textbf{Growth measurement used by traits and by size.} Results shown here are from the ``complete'' dataset (recorded across species and under both unstressed and stressed conditions). The colours correspond to the stage for which the r values have been recorded: red bars corresponds adult stage, orange bars to saplings, and grey bars to juveniles.}
\label{fig:figA8}
\end{figure}

\begin{figure}[h!]
\centering
\includegraphics{output/FigA9.pdf}
\caption{\textbf{Temporal changes in coefficient of correlation between traits and growth. a) Specific leaf area (SLA), b) Wood density (WD), c) Asymptotic height (Hmax), d) Seed mass (SM) and e) Assimilation rate of CO2 (Aarea)} Results shown here are from the ``ideal'' dataset (recorded across species and under  unstressed conditions). The colours correspond to the stage for which the r values have been recorded: red points corresponds adult stage, orange points to saplings, and grey points to juveniles.}
\label{fig:figA9}
\end{figure}

\clearpage
\section{Supporting Analyses}\label{app:supp_info_analyses}
\subsection{Quantifying and exploring the cause of heterogeneity in effect sizes}

\textbf{Goal:} In our meta-analysis, studies reported differed in design as well as in vegetation types, biome or species studied. Such diversity is commonly referred as methodological heterogeneity, and may or may not be responsible for observed discrepancies in the results of the studies. The extent of heterogeneity might influence the results of the meta-analysis, and then induces some difficulty in drawing overall conclusions (See \citealt{Higgins:2002iq}). It is therefore important to be able to quantify the extent of heterogeneity among a collection of studies.

\textbf{Methods:} A common way of addressing the extent of heterogeneity is the statistic $I^{2}$ (\citealt{Santos:2012gt}, originally defined by \citealt{Higgins:2002iq}). $I^{2}$ estimates the consistency of the results obtained from published studies used in our meta-analysis; it describes the percentage of variability in point estimates that is due to heterogeneity rather than sampling error. It can be interpreted as a ratio (0\% $<$ $I^{2}$ $<$ 100\%) not affected by the number of studies or the metric of the effect size in the analysis. It also has the advantage of being analogous to indices used for regression (where R2 is the proportion of the total variance that can be explained by the covariates) a common statistics in the trait literature.

An $I^{2}$ near zero indicates that almost all of the dispersion will be attributed to random error, and any attempt to explain the variance is an attempt to explain something that is (by definition) random. By contrast, as $I^{2}$ moves away from zero we know that some of the variance is real and can potentially be explained by subgroup analysis or meta-regression.
No universal rule could cover definition for ‘low, ‘moderate’ or ‘high heterogeneity. \citealt{Higgins:2003hz} suggested some benchmarks for $I^{2}$ based on the survey of meta-analyses of clinical trials. They suggested that values on the order of 25\%, 50\% and 75\% might be considered as ``low'', ``moderate'' and ``high'', respectively. These suggestions are tentative; the interpretation of heterogeneity in a systematic review will depend critically on the size and direction of treatment effects, as well as on considerations of methodological diversity in the studies (see \citealt{Borenstein:2009um}).
Here, we estimated $I^{2}$ as following equation from \citealt{Higgins:2002iq}., then we calculated how part of the heterogeneity may be due to the influence of moderators (stage and publication year) and finally we asked if the residual heterogeneity is statistically significant. In R, we used the package metafor (code available on guithub).


\textbf{Results:} We observed high levels of heterogeneity for all traits (SLA :$I^{2}$ 85.4\% CI: 80.9 to 92.1, WD :$I^{2}$ 61.8\% CI: 42.8 to 84.1, Hmax: $I^{2}$ 81.2\% CI: 69.8 to 93.4, SM: $I^{2}$ 91.2\%, 85.9 to 96.1, Aarea: $I^{2}$ 68.3\% CI: 33.7 to 90.1) which suggests that the correlations coefficient r between growth and traits are not uniform across the range of studies investigated. This result suggested that a large part of the variance is not explained by sample error.

For SLA and SM, 50.1\% and 20\% respectively of the total amount of heterogeneity were accounted by including the stage as a moderator. For the other traits, the stage explained few or no part of the heterogeneity observed across studies ($<$5\%). These results were consistent with our expectations about a shift in the relationship between growth and traits along plant size for SLA, SM and the absence of effect for WD and Aarea (see Theoretical expectation section). The result of Hmax was more ambiguous since we expected the relationship between Hmax and growth rate to be absent in seedlings but to strengthen with increasing plant size (at adult stage). As discussed in the result section of the article, the smaller numbers of replicate studies for sapling and seedling stage limited the generality of our conclusion for this trait.

The test for residual heterogeneity was significant for all traits, possibly indicating that others moderators not considered in the model were influencing the correlation between growth and trait. This result is not surprising for the trait literature. While the trait and growth measurements are standardized \citep{Cornelissen:2003gw}, studies differed largely in their design as well as in vegetation types, biome or species studied. Indeed, a goal in the trait literature is to establish general pattern about correlation between growth and traits and to validate them across a large variation of conditions. Here, we did not explore the influence of other moderators, since our goal was to test clear hypotheses about the influence of plant size/stage, not to establish the best predictor of the correlation between growth and trait. Finally, since between study variation is significant, as indicated by heterogeneity analyses, using the random effect model was more appropriate in our meta-analysis.

\subsection{Testing for publication bias}
\textbf{Goal:} Studies with significant results are more likely to be published. As a consequence, the studies in a meta-analysis may overestimate the true effect size because they are based on a biased sample of the target population of studies. A publication bias occurs when the probability of publication depends on the statistical significance of the effect. Here we tested if there is an evidence of any bias, and if the effect size observed for each trait is an artefact of this bias.

\textbf{Method:} We first checked the dataset for publication bias with a visual assessment of funnel plots. A publication bias toward significant results creates an asymmetric funnel with small studies with non-significant effects missing from the mouth of the funnel on the side opposite to the true effect. While funnel plot are recommended to aid interpretation, there is a high risk failing to detect actual publication bias (see (\citealt{Koricheva:2013tz}) by using exclusively a visual assessment. In addition, funnel plots are not effective when the number of studies is small ($<30$) such as for Aarea and SM.
We also used the ``trim and fill'' method to estimate the number of studies missing from our meta-analyse. This method adjusts for funnel plot asymmetry \citep{Duval:2000dg}.
Finally, we estimated the number of studies needed to overturn a result, using the Rosenberg’s fail-safe number \citep{Rosenberg:2005hk}. The fail-safe N is not based on funnel plot asymmetry. The Rosenberg method calculates the number of studies averaging null results that would have to be added to the given set of observed outcomes to reduce significance level (p-value) of the weighted average effect \citep{Rosenberg:2005hk}. A significant meta-analytic result is robust if the fail-safe N is greater than 5k+10, where k is the number or studies already in the meta-analysis \citep{Rosenthal:1979do}.
In R, we used the package `metafor` (code available on guithub).

\textbf{Result:} As expected the correlation coefficient varied less widely as sample size increased (funnel plots in Fig. \ref{fig:figA9}). Funnel plots exhibited a typical funnel shape for all five functional traits analyzed, showing no evidence for publication bias (Fig. \ref{fig:figA9}).
The ‘trim and fill’ method detected some missing values in our data. The estimated number of missing value ranges from 0 to 9 depending on the trait considered (Fig. \ref{fig:figA9}). The overall estimates measured with the fill in data changed from $0.5 \pm 0.07$  to $0.4 \pm 0.07$  for SLA and from $-0.48 \pm 0.11$  to $-0.32 \pm 0.12$  for the seed mass, suggesting a lower correlation coefficient for SLA, and SM. Even if the estimated correlation were closest from zero with the missing studied filled in, the results still indicate that the effect is statistically significant.
The Rosenberg’s Fail-safe N are 7086, 1823, 990, 574, 291 for SLA, WD, Hmax, Seedmass and A area, suggesting that there would need to be over a large number of studies with a mean correlation coefficient of 0 added to the analysis before the cumulative correlation would become non significant. N is greater than 5k+10 for all traits.
All together these result suggest that the impact of bias is probably trivial in our meta-analyses.

\subsection{Exploring temporal changes in effect size}
\textbf{Goal:} Our meta-analysis combines data from studies published between 1983 and 2014. It has been shown that the magnitude of the effect size may changes over time due to a publication bias (i.e. a lag to publish negative results), a change in the methodology or biological changes in the magnitude of the effect (see \citealt{Koricheva:2013hy}). Therefore, detecting such temporal changes may be important to the interpretation of the results of a meta-analysis.

\textbf{Method:} In the trait literature the year of publication may be confounded with a methodological change. Indeed, in the early 90’s studies mainly focused on seedlings, whereas today a broader range of plant size/stage are studied (Fig. \ref{fig:figA9}). As a consequence, we used publication year as a moderator in our analyses \citep{Zvereva:2008jm}p across and within stages for all traits.

\textbf{Results:} There was a change in the effect size with publication year for SLA ($p<0.0001$, estimates: $-0.04 \pm 0.009$ ), SM ($p = 0.011$, estimates: $0.041 \pm 0.016$ ) and Aarea ($p = 0.0102$, estimates: $-0.0453 \pm 0.0176$ ). The Fig. \ref{fig:figA9} showed that for these 3 traits the publication years may be confounded with plant stage; the correlation coefficients have been reported from 1993, but studies started to report data on adult stage only 10 years later (around 2000-2007).

For SLA and SM, a publication year effect was not observed within seedling and sapling stages, but was identified for the adult stage (SLA $p = 0.003$, SM $p = 0.06$). For both SLA and SM, the magnitude of the effect size decreased with publication year (Fig. \ref{fig:figA9} a and d), suggesting we overestimated the magnitude of the effect size for these two traits.  In reality, the main effect size obtained would be more close to zero or more negative for SLA and more positive for SM. These results are consistent with our hypotheses; we expected a negative or non-significant effect of SLA on growth in adult stage, and a non-significant or positive effect of SM on plant growth (see theoretical expectation).

For Aarea, publication year explained a part of the variation across stage but not within stage, and the magnitude of the effect size decreased with publication year. This pattern is classic in meta-analyses; \citealt{Koricheva:2013hy} showed that a majority of studies reported a decrease rather than an increase in the magnitude of effect size with publication year.  Yet, here this effect did not lead to a loss of the statistical significance of the main effect size or to a change of sign. While overestimated, the main effect size for the correlation between growth and Aarea stay positive  (Fig. \ref{fig:figA9} e).

The temporal change were confounded with a plant stage effect in our meta-analyse, but they did not jeopardizing the stability of our conclusions. The conclusions of a meta-analyses conducted in different year may not differ from the one reported here.


\end{appendices}
\end{document}