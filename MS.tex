\documentclass[a4paper,11pt]{article}
\usepackage[osf]{mathpazo}
\usepackage{ms}
\usepackage[]{natbib}
\usepackage{floatpag}
\floatpagestyle{empty}

%\raggedright

\usepackage{graphicx}

% Enable cross referencing to Evidence document
\usepackage{xr}
\externaldocument{ms-suppinfo}


\definecolor{grey}{rgb}{0.5, 0.5, 0.5}
\newcommand{\smurl}[1]{\url{#1}}
\newcommand{\ud}{\ensuremath{\rm{d}}}
\newcommand{\tabitem}{~~\llap{\textbullet}~~}
\newcommand{\email}[1]{\href{mailto:#1}{\texttt{#1}}}

\title{On the link between functional traits and growth rate: meta-analysis shows effects change with plant size, as predicted}
\author{Ana{\"i}s Gibert\textasteriskcentered, Emma F. Gray, Mark Westoby, Ian J. Wright, Daniel S. Falster}
\affiliation{Department of Biological Sciences, Macquarie University,
  Sydney, Australia \\
\textasteriskcentered Email for correspondence: \texttt{anais.gibert@gmail.com}
}
\runninghead{Influence of traits on growth changes with size}


\date{}

\bibliographystyle{mee}
\mstype{Research Article}
\keywords{maximum height, photosynthetic rate, ontogenetic stage, growth strategy, seed mass, specific leaf area, wood density}

\begin{document}
\mstitlepage
\noindent
\parindent=1.5em
\addtolength{\parskip}{.3em}
\doublespacing
\linenumbers
\section{Summary 309/350}\label{abstract}
\begin{enumerate}
\def\labelenumi{\arabic{enumi}.}
\itemsep1pt\parskip0pt\parsep0pt
\item A plant's growth rate is seen as a central element of its ecological strategy, and as determined by its traits. Yet the literature is inconsistent about the empirical correlation between functional traits and growth, casting doubt on the capacity of some prominent traits to influence growth rate.

\item We propose that traits should influence growth in a way that depends on the size of individual plants. We outline mechanisms and hypotheses based on new theoretical work, and test these predictions in tree species using a meta-analysis of 103 studies ($>$ 500 correlations) for five traits (specific leaf area, wood density, maximum height, seed mass and maximum assimilation rate). We also recorded data for 14 other traits commonly used in the trait literature. To capture the effects of plant size, we tested for a shift in the direction of correlation between growth rates and each trait across three ontogenetic stages: seedling, sapling and adult.

\item Results were consistent with predictions, although there were some limitations arising from unequal numbers of observation across ontogenetic stages. Specific leaf area was correlated with relative growth rate in seedlings but not in adult plants. Correlations of growth with wood density were not affected by ontogenetic stage. Seed mass, assimilation rate and maximum height were correlated with relative growth rate only in one ontogenetic stage category: seedlings, seedlings and adults, respectively.

\item Although we were able to confirm several of our theoretical predictions, major knowledge gaps still exist in the trait literature. For example, for one-third of the traits considered, the majority ($>$ 75\%) of reported correlations with growth came from the same ontogenetic stage.

\item Synthesis: We show for some traits, how trait-growth correlations change in a predictable way with plant size. Our understanding of plant strategies should shift away from categorizing species as having slow or fast growth throughout their life, in favour of attributing growth trajectories.
\end{enumerate}

\section*{Introduction}\label{introduction}

The phrase ``plant ecological strategy'' refers to how species face common challenges of acquiring sufficient water, nutrients, and light for growth; and then ultimately replacing themselves with offspring. The idea that strategy differences among species might involve fast versus slow growth goes back at least to \citet{Grime:1975gr}, who observed a wide spread of relative growth rates among seedlings growing under favourable conditions. Seedling potential relative growth rate has continued to be used as a major strategy indicator within Grime's CSR (Competitor -- Stress tolerator -- Ruderal) classification (e.g. \citealt{grime1979plant, Grime:1997wm}). Meanwhile partitioning of seedling relative growth rate into different components showed that specific leaf area (SLA) was typically the strongest source of variation between species \citep{Poorter:1989tx, Rees:2010gk}. During the 1990's a new style of ``trait ecology'' arose whereby measurable traits, including SLA, were used directly as strategy axes \citep{Westoby:2002ft}. This approach, using measurable traits, made broad comparisons possible across continents, latitudes, and thousands of species. Since then, functional traits have been seen as constructive tools to understanding plant growth strategies \citep{Westoby:2002ft}, community composition \citep{Lavorel:2002ff,Shipley:2006ie}, global vegetation dynamics \citep{Scheiter:2013ed} and ecosystem processes \citep{Lavorel:2002ff}.

Currently, two main spectra of trait variation are widely cited as underpinning differences between species in vegetative growth rates. The ``leaf economic spectrum'' \citep{Wright:2004jb} and the ``wood economic spectrum'' \citep{Chave:2009iy} reflect tissue construction costs (for leaf area and for wood volume respectively), trading off with rates of tissue turnover or mortality. It has been widely expected that species with low tissue construction cost will typically have fast growth rates, at least in favourable physical environments \citep[e.g.][]{MullerLandau:2004dc,Poorter:2008iu,Chave:2009iy,Larjavaara:2010bn,Iida:2012jb,Paine:2015df}. Yet, recently a number of empirical studies indicate the correlation between traits and growth is not consistent. Consider SLA as an example. By definition, species with high SLA can deploy more leaf area for a given dry mass investment in leaves \citep{Poorter:1999wd, Reich:1992wm}. In seedlings growing under favourable conditions, SLA has repeatedly been found to be a strong predictor of relative growth rate, regardless of vegetation type or growth form \citep{Lambers:1992bj,Reich:1992wm,Grime:1997wm,Poorter:1999wd,Wright:1999ds}. Yet for adult plants, the correlation has mostly been much weaker or absent \citep{coomes_comparison_1998,Poorter:2008iu,Aiba:2009ft,Easdale:2009gv,Wright:2010tp}. The discrepancy between theoretical expectation and empirical results has led some to question whether traits are even useful for understanding growth \citep{Wright:2010tp, Paine:2015df}.

Meanwhile the possibility has been raised that trait-growth correlations may change systematically with size \citep{King-1999, Falster:2011ii, Ruger:2012jv, Iida:2014ep, Iida:2014hq}. Size has long been known to play a central role in organism design and function \citep{Vogel:1988ux, Vogel:2003wb}, including that of trees \citep{Farnsworth:1995im, Givnish:1995ta, King-2011}. As plants grow larger, a greater amount of energy is devoted towards building and maintaining non-photosynthetic tissues \citep{Givnish:1995ta,King-2011}. A recent model by \citet{Falster:2011ii} suggests that increased size shifts the benefits and costs of some trait-based trade-offs, such that the net effect of the trait on growth changes with size. Thus the absence of a correlation between SLA and growth at larger sizes may in fact be expected, when the potential influences of traits on plant function are properly considered \citep{Falster:2011ii}. While some recent empirical studies support the idea that the relationship between traits and growth rates may change with size (\citealt{Iida:2014ep, Iida:2014hq}), the generality of the trend and mechanisms behind this variation remain unclear.

In this paper, we consider five prominent traits (specific leaf area = SLA, wood or stem tissue density, maximum leaf photosynthetic capacity per unit area = Aarea, maximum or asymptotic height, and seed mass) and we investigate:
\begin{enumerate}
  \item How general is the tendency for correlations between traits and growth rate to vary with plant size?
  \item Do the observed relationships conform to predictions derived from a recently developed mechanistic model \citep{Falster:2011ii} (see ``Theoretical expectations'')?
\end{enumerate}
To address these questions we used a meta-analytic approach, compiling data from studies reporting correlation coefficients between traits and growth. While most studies investigating the link between traits and growth have not primarily been concerned with the effects of plant size, they have generally reported trait-growth correlations for identifiable ontogenetic stages (i.e. seedling, sapling, and adult). We therefore used ontogenetic stage as a surrogate for plant size. We focused on tree and woody species since the mechanistic model used to assess our predictions is most applicable to these growth forms.
Using a meta-analysis approach, we were able to assess our hypotheses across a much broader number of species and vegetation types than would have been possible in an experimental study. We anticipated a sizeable literature on traits and growth to be available but we were uncertain about the representation of all trait-stage combinations. An additional goal of our study was therefore to identify gaps in the knowledge needed to improve our understanding of the link between functional traits and plant growth.

\section*{Theoretical expectations}\label{theory}

Recent theoretical work \citep{Falster:2011ii} allows us to formulate hypotheses about the expected relationship between traits and growth, based on a mathematical decomposition of plant growth rate. Specific predictions arise from considering the influence of traits on elements of these equations and how these effects may vary with size. This model extends earlier efforts to model mass-based growth of seedlings as a function of traits \citep{Lambers:1992bj,Cornelissen:1996hf,Wright:2000kw,Enquist:2007ek}. However, whereas earlier models focused solely on mass-based relative growth rate and did not predict a change in effect of traits with size, the revised equations do allow for these effects, while also making predictions for both absolute and relative growth in any of mass (eq. \ref{eq:eq1}), height (eq. \ref{eq:eq2}), or diameter (eq. \ref{eq:eq4}).

We first outline equations for the new growth model, adapted from \citet{Falster:2011ii}. We then use this model to generate qualitative predictions for how the effect of traits on growth rate changes with plant size. These hypotheses are summarised in Table \ref{tab:trade-offs}.

\subsection*{A generic model for growth rate in plant mass, height and stem diameter}\label{sec:growth}

The rate at which \textbf{live biomass} $B$ is produced by a plant is given by
\begin{equation}\label{eq:eq0}
\underbrace{\frac{\ud B}{\ud t}}_{\text{net live biomass production}} = \underbrace{\alpha_{\rm y}}_{\text{yield}}  \big(\underbrace{A_{\rm l} p}_{\text{photosynthesis}} - \underbrace{\sum_{i = {\rm l,s,r}} M_{i} r_{i}}_{\text{respiration}}\big) - \underbrace{ \sum_{i = \rm{l,s,r}} M_{i} k_{i}}_{\text{turnover}} \, ,
\end{equation}
where $\alpha_{\rm y}$ is the ratio of carbon fixed in mass per carbon assimilated, and $p$ is the assimilation rate of CO$_{2}$ per leaf area. Here, and in the following equations, $M_i$ denotes the \textbf{live mass on the plant}, $r_i$ the mass-specific \textbf{respiration rate}, and $k_i$ the \textbf{turnover rate} of different plant tissues; the subscripts $i = \rm{l,s,r,a,f}$ referring to leaves, sapwood and bark, roots, total of these vegetative tissues, and to the reproductive tissues (fruits, flowers etc.)  Similarly $A_i$ denotes \textbf{areas}, of leaves ($i = l$) and of cross-sections of total stem and sapwood ($i =\rm{st,ss}$) respectively. Note that the live biomass $B = M_{\rm a} + M_{\rm f}$.  Applying the chain rule of differential calculus, the \textbf{growth rate in plant mass}, ($M_{\rm a} $), can be decomposed as
\begin{equation}\label{eq:eq1}
\underbrace{{\frac{\ud M_{\rm a} }{\ud t}}}_{\text{growth rate in mass}} = \underbrace{\frac{\ud M_{\rm a} }{\ud B}}_{\text{allocation to vegetative growth}}  \times \underbrace{\frac{\ud B}{\ud t}}_{\text{net live biomass production}},
\end{equation}
where $\frac{\ud M_{\rm a} }{\ud B}$ is the fraction of net live biomass increase that goes to vegetative growth rather than to production of reproductive tissues.

The \textbf{growth rate in height} can likewise be decomposed as the product of several meaningful terms:
\begin{equation}\label{eq:eq2}
\underbrace{\frac{\ud H}{\ud t}}_{\text{growth rate in height}} = \underbrace{\frac{\ud H}{\ud A_{\rm l}}}_{\text{architecture layout}} \times \underbrace{{\frac{\ud A_{\rm l}}{\ud M_{\rm a} }}}_{\text{leaf deployment per mass}} \times {\frac{\ud M_{\rm a} }{\ud B}} \times {\frac{\ud B}{\ud t}},
\end{equation}
where $\ud H / \ud A_{\rm l}$ is the growth in plant height
per unit growth in total leaf area. This term expresses the architectural
strategy of the plant. We assume it is unrelated to the traits considered in the current paper, in other words that it does not automatically change in response to changes in the traits. The term $\ud A_{\rm l} / \ud M_{\rm a} $
accounts for the marginal cost of deploying an additional unit of leaf
area, including construction of the leaf itself and various support
structures.  $\ud A_{\rm l} / \ud M_{\rm a} $
can itself be expressed as a sum of costs per unit leaf
area for constructing various support tissues,
\begin{equation}\label{eq:daldmt}
\frac{\ud A_{\rm l}}{\ud M_{\rm a} }
= \left(\frac{\ud M_{\rm l}}{\ud A_{\rm l}}
 +  \frac{\ud M_{\rm s}}{\ud A_{\rm l}} + \frac{\ud M_{\rm r}}{\ud A_{\rm l}}\right)^{-1}.
\end{equation}

Similarly \textbf{growth rate in basal area} can be expressed as the sum of growth in sapwood and heartwood, with the former term again decomposed as a product:
\begin{equation}\label{eq:eq3}
\begin{array}{ccccc}
\underbrace{{\frac{\ud A_{\rm st}}{\ud t}}}_{\text{basal area growth rate}} &=& \frac{\ud A_{\rm ss}}{\ud t} &+& \frac{\ud A_{\rm sh}}{\ud t} \\
&=& \underbrace{\frac{\ud A_{\rm ss}}{\ud A_{\rm l}}  \times {\frac{\ud A_{\rm l}}{\ud M_{\rm a} }} \times  {\frac{\ud M_{\rm a} }{\ud B}} \times {\frac{\ud B}{\ud t}}}_{\text{increase in total sapwood}} &+&  \underbrace{k_{\rm ss} \times \frac{\ud A_{\rm ss}}{\ud t}}_{\text{sapwood turnover}}.\\
\end{array}
\end{equation}
Here $\frac{\ud A_{\rm ss}}{\ud A_{\rm l}} $ is the  sapwood area per leaf area. The \textbf{growth rate in diameter} is then obtained via the geometric relationship between stem diameter and stem area,
\begin{equation}\label{eq:eq4}
\underbrace{\frac{\ud D}{\ud t}}_{\text{growth rate in diameter}} = \left(\pi \, A_{\rm st}\right)^{-0.5} \times {\frac{\ud A_{\rm st}}{\ud t}}.
\end{equation}

\subsection*{Predicted effects of traits on growth}

The above equations are general, i.e. they do not depend on detailed assumptions about plant construction. To make qualitative predictions with respect to traits such as specific leaf area and wood density we require one additional assumption: that the mass of sapwood tissue ($M_{\rm ss}$) increases disproportionately with respect to the plant's leaf area ($A_{\rm l}$). This assumption is included in nearly every model of plant growth that has been developed, on the basis that as stems grow longer the total mass of sapwood associated with each leaf must be increasing \citep[e.g.][]{Falster:2011ii,King-1999,Enquist:2007ek}. It has the effect of changing the relative importance of different components within equations for $\frac{\ud B}{\ud t}$ (net live biomass production) and $\frac{\ud A_{\rm l}}{\ud M_{\rm a} }$ (marginal cost of deploying additional leaf area) as a plant increases in size. Specifically, in eq. \ref{eq:eq0}, the costs of stem respiration and turnover increase faster than leaf photosynthesis, respiration or turnover, as the plant grows. Similarly, in eq. \ref{eq:daldmt}, the costs of constructing sapwood to support a given unit of leaf area, $\frac{\ud M_{\rm s}}{\ud A_{\rm l}}$, increases faster than the cost of producing a unit of leaf area, $\frac{\ud M_{\rm l}}{\ud A_{\rm l}}$ which is simply the inverse of specific leaf area. Whether plant growth rate increases with a specific trait, therefore depends on how the trait affects the different elements of \ref{eq:eq0}-\ref{eq:eq4} and how those effects change with size (Table \ref{tab:trade-offs}).

Based on the above model and this single assumption about growth, we outline five hypotheses that arise by considering the influence of trait variation on elements of equations  \ref{eq:eq0}-\ref{eq:eq4} and how these may interact with plant size (see Table \ref{tab:trade-offs}). It is possible that other correlations may arise indirectly. Nevertheless, for the purposes of this paper we adopt as working hypotheses only those arising from direct mechanistic causation. We consider predictions for each of height, diameter, and mass-based growth, and how they  may vary depending whether growth is expressed in absolute or relative terms (see Table \ref{tab:trade-offs}). For many trait and growth combinations, the predicted effect of the trait on growth is similar when expressed in relative and absolute terms. However, this is not true for all traits (details given below).
\subsubsection*{H1: The relationship between SLA and growth rate is expected to shift from positive in seedlings to non-significant or negative in adult plants}

By definition, plants with high SLA can deploy a unit of leaf area more cheaply. However, high-SLA species also have shorter leaf lifespan, and must use more energy replacing leaves lost via turnover \citep{Wright:2004jb}. These two effects occur within the terms  $\frac{\ud B}{\ud t}$ (net live biomass production) and $\frac{\ud A_{\rm l}}{\ud M_{\rm a} }$ (marginal cost of deploying additional leaf area), appearing on the right side in the equations for growth rate (\ref{eq:eq1}, \ref{eq:eq2}, \ref{eq:eq3}). Whether plant growth rate increases with SLA depends on how the relative magnitude of the benefit (cheap leaf construction, via $\frac{\ud A_{\rm l}}{\ud M_{\rm a} }$ ) and cost (higher leaf turnover, via $\frac{\ud B}{\ud t}$) of increased SLA changes with size (Table \ref{tab:trade-offs}).
During early life, when leaves comprise a large part of the plant, increasing SLA has an overwhelmingly positive effect on height and diameter growth rate, because the effect of reducing $\frac{\ud M_{\rm l}}{\ud A_{\rm l}}$ (the cost of producing a unit of leaf area) has a large effect compared to the other terms in eq. \ref{eq:daldmt}. As plants increase in size, however, the cost of building sapwood ($\frac{\ud M_{\rm s}}{\ud A_{\rm l}}$) increases disproportionately, decreasing the benefit of cheaper leaf construction in eq.  \ref{eq:daldmt}. Consequently, the effect of leaf turnover becomes relatively stronger at larger sizes, and as such, the effect of SLA on height, diameter and, mass growth shifts from positive to negative as plants increase in size (Table \ref{tab:trade-offs}). Note in the case of mass growth, the effect of cheaper leaf construction is realised not in the $\frac{\ud A_{\rm l}}{\ud M_{\rm a} }$ term , but as a greater total leaf area at a given mass, i.e. the accumulated effect of past reductions in  $\frac{\ud A_{\rm l}}{\ud M_{\rm a} }$.

\subsection*{H2: The relationship between wood density and growth rate is expected to be negative across all plant sizes}

Assuming plants maintain a similar cross section of conducting tissue per unit leaf area, cheaper stem construction (low wood density or stem density) will increase the marginal rate of leaf deployment per total mass increment ($\frac{\ud A_{\rm l}}{\ud M_{\rm a} }$), and consequently growth rate in height and diameter (eqs. \ref{eq:eq2}-\ref{eq:eq4}; Table \ref{tab:trade-offs}). Disadvantages associated with cheap stem construction take the form of higher mortality and so are not reflected in growth equations. Under this scenario, wood density is negatively correlated to growth rate across all ontogenetic stages. Because the relative expenditure on stem compared to leaf increases as plants grow larger, the relationship between wood density and growth may also strengthen in larger plants. A similar shift is expected for growth in mass, in either absolute or relative terms. Note in the case of mass growth, the effect of cheaper stem construction is realised not in the $\frac{\ud A_{\rm l}}{\ud M_{\rm a} }$ term , but via greater total leaf area at a given mass, i.e. the accumulated effect of past reductions in  $\frac{\ud A_{\rm l}}{\ud M_{\rm a} }$.

\subsection*{H3: The relationship between maximum height and growth rate is expected to be absent in seedlings but to strengthen with increasing plant size}

As plants approach maximum height they increasingly divert energy away from growth and into reproduction (see \citealt{Thomas:1996do,Thomas-2011, Wenk:2015jz}). This effect occurs in the term $\frac{\ud M_{\rm a} }{\ud B}$ (the allocation to vegetative growth) in eqs. \ref{eq:eq2}-\ref{eq:eq3}. For seedlings, where individuals are investing solely in growth, we expect no influence of maximum height on growth (i.e. $\frac{\ud M_{\rm a} }{\ud B} =1$). But once some species begin to produce seed, then species closer to their maximum height value are expected to be allocating more to seed production and less to vegetative growth. Hence when comparing across plants at a given size, species with smaller maximum height are expected to show slower vegetative growth rates. Similar correlations are expected no matter what measure of growth is used (Table \ref{tab:trade-offs}).

\subsection*{H4: The relationship between seed mass and relative growth rate is negative in seedlings, but weakens with increasing size}

The only effect of increased seed size in the above model is to produce a larger overall seedling. Under similar light conditions, larger seedlings are predicted to have faster absolute growth rates (in diameter, height, and mass), because of their greater total leaf area. At the same time, relative growth rate is predicted to decrease with size, because the ratio of leaf area to support mass decreases with plant size \citep[see also][]{Turnbull:2012ew}. We therefore expect a positive correlation between seed size and absolute growth rate and a negative correlation between seed size and relative growth rate, at the seedling stage (Table \ref{tab:trade-offs}). As plants grow, differences in initial mass will decrease in importance, relative to other factors influencing growth through the life-cycle. We therefore expect the correlations between seed size and growth rate observed for seedlings to disappear among larger plants (Table \ref{tab:trade-offs}).

\subsection*{H5: The relationship between Aarea and growth rate is expected to be positive across all plant sizes, particularly under high-light conditions}

We expect greater photosynthetic capacity to translate into faster growth rates by increasing the rate of biomass production (Table \ref{tab:trade-offs}), at least at high light where Aarea is typically measured. This effect occurs by increasing  $\frac{\ud B}{\ud t} $ in eqs. (\ref{eq:eq1}-\ref{eq:eq3}). The positive effect of increasing Aarea (higher assimilation) may be offset by the costs (higher respiration), however, assuming both increase in proportion, and because assimilation is typically higher than respiration in high light, the net effect will be positive (Table \ref{tab:trade-offs}). It is uncertain how much this effect might translate to faster growth in low light.

\section*{Materials and methods}\label{material-and-methods}

\subsection*{Dataset construction}\label{data-construction}

We aimed to identify studies reporting a correlation across species, between traits and growth rates (e.g. relative or absolute growth, in diameter, height or mass, see Table \ref{tab:trade-offs}). Systematic searches were conducted across the two most comprehensive bibliographic platforms: Google scholar and Web of Science (Thomson Reuters). While Web of Science guarantees a relatively stable and reproducible search environment, with clearly defined lists of indexed journals, the searches are limited to meta-data (i.e. title, keywords, and abstracts, \citealt{Beckmann:2012hn}). This limitation does not apply to Google Scholar. We supplemented these records with results from our own bibliographic libraries and suggestions from other researchers.

A big challenge when searching literature databases is to identify suitable studies from the vast corpus available. Initially we were interested to know whether data existed for a variety of traits; however the number of potential search terms made systemic searching for all these terms in Google Scholar impractical. A preliminary investigation of search results for a variety of traits on the Web of Science platform suggested that searching only on SLA and wood density -- two of the most widely measured traits -- would capture $\sim$80\% of search records that would be returned by searching for a wider range of traits (see Appendix \ref{app:literature-snapshot}). We therefore narrowed our searching to only cover these two main traits.

Both Web of Science and Google Scholar databases were searched for records linking SLA and wood density to growth rate. On Web of Science, we searched using  search terms: (``trait'' AND (``growth rate'' OR ``relative growth rate'')) AND (``tree*'' OR ``woody*''), where ``trait'' was either (``leaf mass per area'' OR ``specific leaf area'' OR ``leaf construction cost'') or (``wood density'' OR``WD'' OR ``wood specific gravity'' OR ``stem density'' OR ``stem specific density'' OR ``SSD'' ). On Google scholar we searched using terms: ((``SLA'' OR ``LMA'' OR ``specific leaf area'' OR ``leaf mass area'' OR ``wood density'' OR ``WD'' OR ``wood specific gravity'' OR ``stem density'' OR ``stem specific density'') AND (``growth rate'' OR ``relative growth rate'')) AND (``tree*'' OR ``woody*''). This keyword search in Google Scholar returned approximately 18900 papers. We examined all records returned from the Web of Science search (701 articles) and the first 750 items from Google Scholar when ranked by ``relevance''. All searches were conducted between March and December 2014.

In total we inspected the full text of 583 articles (see Appendix \ref{app:literature-list} for list). Individual studies had to report the following information to be included in our dataset: i) the relationship between growth and trait as a correlation coefficient, ii) the sample size and the number of species used to perform the correlation, and iii) the size or stage of plants. Despite searching only on SLA and wood density, we recorded coefficient correlations for a list of 19 traits in total: the five for which we had prior expectations, plus any of 14 others commonly used in the trait literature. Where available, we also recorded the geographic coordinates, number of sites, experiment type (greenhouse, field experiment, forest, shade-house etc.), growth measure (relative or absolute growth rate measured using plant diameter, height or mass), vegetation type, life form (tree, woody or mixed with herbaceous species etc.), biological scale (interspecific, intraspecific or intragenus), and experimental treatments (e.g. fertilization, shade, salt stress). In some cases, we asked the authors for additional information about the specific correlation coefficient value when this was not already provided.

The 19 traits recorded (Fig. \ref{fig:fig1}) capture different components of plant ecological strategies. SLA, leaf area, leaf nitrogen and phosphorus concentration, net (mass) assimilation rates (NAR), and Aarea are traits involved in species leaf economic strategy \citep{Wright:2004jb,Wright:2010tp}. Seed mass and maximum height capture information about reproductive strategy \citep{Falster:2005bw,Moles:2006ft}. Wood density and sapwood area per leaf area (SA/LA), vessel size and density, and stem specific conductance are all involved in species wood economic strategy  \citep{Chave:2009iy}. Here wood density and SA/LA are also considered as indicators of stem construction cost \citep{Falster:2011ii}, whereas SLA and leaf thickness capture crucial information about leaf construction cost \citep{Wright:2004jb}. Finally, the leaf area ratio and leaf mass ratio are indicators of resource allocation to leaf versus other tissue types \citep{Lambers:1992bj}.

Ultimately, data were recorded from 103 different articles \citep{Aiba:1997ho, Aiba:2009ft, Antunez:2001gk, Atkin:1998in, augspurger_light_1984, baltzer_physiological_2007, baraloto_differential_2006, baraloto_seed_2005, Bloor:2003fa,  Broncano:1998ck, Brown:1996ks, Bruhn:2000jz, Cai:2007ie, CamaraZapata:2003df, CastroDiez:1998gz, CastroDiez:2003jd, Cernusak:2008ds, Chao:2008hg, Chaturvedi:2011jx, Chaturvedi:2014ie, Chave:2009iy,  coomes_comparison_1998, coomes_greater_2009, Cornelissen:1996hf, Cornelissen:1997fd, Cornelissen:1998ta, Dalling:2004gs, DeBell:1994ix, Easdale:2009gv, fan_hydraulic_2012, Fayolle:2012bx, Fujimoto:2006fg, Galmes:2005fa, Gleason:2016, gleeson_plant_1994, grotkopp_toward_2002, Grubb:1996kr, Herault:2011dd, Hoffmann:2003ga, Huante:1995fd, Huante:1995tl, Huante:1998bk, HUNT:1997ge, Iida:2014ep, Iida:2014hq, king_influence_1994, king_tree_2005,  King:2006he, Kitajima:1994ez, Kohyama:2003dk, Kruger:2006fq, Lamers:2006dl, LopezIglesias:2014dk, Loveys:2002hj, Lusk:1997ga, Lusk:2002fma, Lusk:2013bq, Lusk:2013hz, MartinezVilalta:2010iq, McCormack:2012ch, Moya:2008ep, Muller-landau_interspecific_2004, Nascimento:2005kl, Osone:2008dr, Osunkoya:1994ip, Osunkoya:2010ef, Paz:2005kx, Poorter:1999fp, Poorter:2006vb, Poorter:2008iu, Poorter:2010co, Popma:1988tv, Prior:2004cv, Quero:2008jg, Read:2011du, Reich:1998ir, Reich:1998ja, Rossatto:2009gq, RuizRobleto:2005hc, Sack:2004hm, saldana-acosta_seedling_2009, SalgadoLuarte:2012ci, Saverimuttu:1996ih, Shen:2014fw, Shipley:2002fs, Stratton:2001ck, Thomas:1996do, Tomlinson:2014im, Veneklaas:2002dt, Villagra:2013do, Villar:2006cf, walters_are_1996, Walters:1993eu, Walters:1993hf, Wang:1998cs, Warren:2005bc, Westbrook:2011ja, Wright:1999ds, Wright:2000kw, Wright:2001gb, Wright:2003kb, Wright:2010tp, Zhang:1995bh}.

\subsubsection*{Ontogenetic stage categories}

We used categorical groupings by ontogenetic ``stage'' to test our hypotheses regarding the effects of traits on growth rates. The authors of the 103 articles used in this analysis reported their plants as belonging to any of five different stages: ``seedling'', ``juvenile'', ``sapling'', ``adult'', ``mixed''. Because these categories were established according to the expertise of the authors, we were concerned about the risk of assigning plants of similar size to different categories when compiling data across multiple studies and vegetation types. This risk turned out to be small: overall, we found that the original thresholds in size/age used by the authors to distinguish different stages were relatively compatible with thresholds we defined ourselves (Fig. \ref{fig:fig2}). In particular, there was a clear break between seedlings and saplings in both age (at 1 yr) and height (at 0.5 m). There was also a clear break between saplings and adults in stem diameter (at 10 cm). Comparing these values to the bivariate relationships observed between variables (age, height, diameter) in the ``Biomass And Allometry Database'' \citep{Falster:2015}, demonstrated that the thresholds identified above for each stage transition also aligned relatively well with one another (Fig. \ref{fig:figA1}). Thus we adopted the above values as cut-offs for defining stage boundaries.

Using the identified boundaries among stages we were able to assign all studies into one of three classes: ``seedling'', ``sapling'', or ``adult''. The group ``juvenile'' was largely overlapping with the three others, and so we merged this category with one or the other on a case by case basis. Using the above cut-offs, we were also able to assign a stage category to studies for which no stage was distinctly mentioned by the authors (5\% of the articles, noted as ``mix'' in Fig. \ref{fig:fig2}). Only 7 of 103 articles were ambiguous on their ontogenetic stage attribution. Half of these equivocal cases involved the juvenile category. We attributed a study to a new stage category only when 95\% of the recorded range of sizes for that study lay within a particular stage category. Other equivocal cases were caused by ambiguity in the usage of ``adult'' by the authors. Some plants with a large stature were not described as adult by authors because they were still non reproductive \citep{King:2006he}. Conversely plants with small stature -- because stunted by wind for example \citep{Stratton:2001ck} -- were sometimes assessed as adult when reproductive. For purposes of the present paper, we attributed a study to the adult stage when the authors identified the plants as reproductive, and to sapling stage when they identified the plants as non-reproductive, regardless of the plant's size or age. We found that results using the original categories established by the authors were similar to those using the standardized categories (see Fig. \ref{fig:figA6}). Thus in the main text, we report only results using the standardized categories.

\subsubsection*{Analyses of relative and absolute growth rates under conservative and relaxed criteria}\label{conservative-vs-entire-dataset}

The dataset included studies differing in the growth measurement used (relative or absolute), the number of species, the experimental conditions (stressed vs unstressed), and the concordance between stages used to measure trait and growth rate. These differences provide good criteria for assessing the quality of the data for addressing our hypotheses. We adopted a three-stage approach where at first we were conservative by focusing exclusively on relative growth rate, across studies that included \underline{\textgreater} 10 species, under unstressed conditions, and where the trait and growth were measured at the same stage (expect for maximum height and seed mass, where a particular stage is by definition required to measure the trait). We then compared these results with those from the subset of the dataset using relative growth rate, but with all other criteria relaxed. Finally, we compared results from both relative and absolute growth rates, from the entire dataset. 


\subsection*{Statistical analysis}\label{statistical-analyses}

\subsubsection*{Effect size calculation}\label{effect-size}

To measure effect size, we used the cross-species Pearson product-moment correlation coefficient $r$ between growth rate and trait, weighted by the number of species contributing to the relationship. Results from studies that reported another coefficient -- such as coefficient of determination, Spearman's rho rank correlation, or Kendall's tau rank correlation -- were converted to correlation coefficients following \citet{Lajeunesse:2013tm}. As the magnitude of Pearson's correlation coefficient approaches $\pm$ 1.0, its distribution becomes skewed. Accordingly, we transformed the correlation $r$ into a metric $z$ with more-desirable statistical properties. The transformation was achieved via Fisher's z-transformation,
\[ z = \frac{1}{2} \ln \frac{1+r}{1-r},\]
which has a variance estimate of
\[ v_z= \frac{1}{n-3},\]
where $n$ is the number of species. We weighted effect size by the number of species used to calculate the correlation, instead of the raw number of individuals, because our questions primarily concerned the correlation between growth and traits across species and not within species. Throughout we report means and 95\% confidence intervals (CI) on effect size.

\subsubsection*{Ontogenetic stage effect}\label{stage-effect}

We tested an effect of ontogenetic stage on the relationship between growth rate and traits using a mixed-effect model. In this model, the response variable corresponds to the z-transformed coefficients of correlation calculated above, and it was assumed that studies within a stage shared a common mean effect, but that there was also random variation among studies, in addition to sampling variation \citep{Zuur:2009cfa}. Inclusion of this random effect allowed us to account for heterogeneity among studies, for example in experimental type, and vegetation type. For each trait, we ran a separate model and established whether stage could explain additional variance in $z$ by comparing models with and without a stage effect using log-likelihood ratios \citep{Zuur:2009cfa}. If the log-likelihood test returned a significant p-value, the stage model fitted the data better than the model without. Estimates of the effect size for each stage were considered significant if the CIs did not overlap with zero. Non-overlapping CIs across stages highlighted significant differences in effect sizes.

\subsubsection*{Quality criteria for meta-analyses}

\citet{Koricheva:2014ku} outlined 14 methodological criteria to ensure robust results from meta-analyses. Table \ref{tab:meta} summarises how we addressed each of these. In particular, we i) quantified heterogeneity in effect size, ii) tested for publication bias, and iii) explored temporal change in effect size (see Appendix \ref{app:supp_info_analyses}).
We also tested how robust our results were to the decisions made in the analysis. Specifically, we assessed the sensitivity of our results to inclusion of studies reporting multiple trait-growth correlations across different conditions (e.g. vegetation type, sites, environmental treatment). In the case of multiple comparisons within a study, we used the method proposed by \citet{Borenstein:2009um}: we calculated the mean correlation coefficient by trait and by study, and then re-calculated the effect size. This produced a reduced dataset, where multiple responses for any given trait were condensed into a study-level average.

\subsubsection*{Software}

All analyses were conducted with \texttt{R} software \citep{Ralanguageanden:2014wf} using packages \texttt{lme4} \citep{Bates-2014}, \texttt{metafor} \citep{Viechtbauer-2010}, and \texttt{gpplot2} \citep{Wickham:2009}.


\section*{Results}\label{results}

\subsection*{Representation of ontogenetic stages among 19 traits}
In total, correlations had been reported for at least two different stages across 17 traits out of the 19 targeted (Fig. \ref{fig:fig1}). We had thought it likely that for some traits, correlations would be reported only for one growth stage. But this limitation applied only for NARmass and vessel density, where studies appeared confined to seedling and adult stages respectively. However, for a majority of traits the number of observations across stage was highly unbalanced; for one-third of the 19 traits $>$ 75\% of correlations with growth reported come from the same stage (Fig. \ref{fig:fig1}).

\subsection*{Main effect for relative growth rates under conservative criteria}

We identified only 55 studies containing 137 correlations reporting data measured under conservative criteria, i.e. unstressed growth, growth measured as relative growth rate, traits and growth recorded in same stage, and correlation coefficient calculated across \underline{\textgreater} 10 species. Overall, results were consistent with predictions, although with limitations arising for some traits due to very unequal (and limited) numbers of observations between growth stages.

For SLA, the correlation between relative growth rate and the trait changed significantly with plant size in the manner expected (Fig. \ref{fig:fig3}a). The correlation between relative growth rate and SLA shifted from strongly positive at seedling stage ($\bar{r}=0.60 \pm 0.08$, Mean $\pm$ SD) to not significantly different from zero at adult stage ($\bar{r}= -0.01 \pm 0.07$, CI 95\% overlapping zero). The correlation coefficients recorded at sapling stage were intermediate between seedling and adult stages. Model fit improved significantly when stage was incorporated (see log likelihood ratio tests, Fig. \ref{fig:fig3}a; average values for the coefficients ($\bar{r}$) in Fig. \ref{fig:figA4}).

For wood density and seed mass, the available results are consistent with our theoretical expectations, but it would be desirable to have substantially more data. Indeed, while it is possible to conduct a meta-analysis even with few observations, strong conclusions are generally precluded by very large CIs of the effect size (i.e highly uncertain estimates and CI often spanned zero).

For wood density, results were consistent with the prediction for negative correlations with relative growth rate across all plant sizes (Fig. \ref{fig:fig3}b). Accordingly, model fits were marginally improved when stage was incorporated (p-value = 0.049). Note that the CIs did not cross zero for seedling and sapling stage, despite the small sample sizes (n= 1 and 4 respectively), meaning a strong correlation and/or a good consistent effect in the coefficient reported for each stage across studies.

As expected, for seed mass the correlation with relative growth rate shifted from negative at seedling stage ($\bar{r}= -0.5 \pm 0.07$) to not significantly different from zero at adult and sapling stages (Fig \ref{fig:fig3}c, CIs overlapping zero). However, the correlations between seed mass and relative growth rate at adult and sapling stages were also not significantly different from that observed at seedling stage (see log likelihood ratio tests and CIs overlapping in Fig. \ref{fig:fig3}c). We could not distinguish between these alternatives. But the number of coefficients reported for seed mass was almost similar between adult (n= 7) and seedling (n= 10) stages, suggesting that CIs overlapping zero at adult stage were due to variation in correlation coefficients reported rather than to a much smaller numbers of replicates.

For maximum height and Aarea, correlations with relative growth rate were also consistent with expectations, but our ability to test all hypotheses was at times limited due to low or no sample size.

For maximum height, the expectation was for a correlation with growth to be present for large plants but absent for smaller sizes, the mechanism being increased relative allocation to reproduction as maximum height is approached. Here, maximum height was positively correlated with growth at adult stage ($\bar{r}= 0.38 \pm 0.05$), not significantly correlated at sapling stage (Fig \ref{fig:fig3}c, CIs overlapping zero) and no data were recorded for seedling stage. While these results were consistent with our expectation that maximum height would be correlated with growth at adult stage, they did not differ significantly across ontogenetic stages (model fit not improved when stage was incorporated, see log likelihood ratio tests in Fig. \ref{fig:fig3}c, and CIs overlapping between sapling and adult stages). Due to small numbers of replicate studies for sapling stage and no replicates for seedling stage, we could not conclude on the absence of correlation between maximum height and growth at small plant size.

For Aarea, results were consistent with a positive correlation with relative growth rate at seedling stage. However, no conclusion can be drawn about a change along ontogenetic stages, as only four correlation coefficients were reported for this trait, and three of these were measured at the seedling stage (Fig. \ref{fig:fig3}e).

\subsection*{Main effects for relative growth rate under relaxed criteria}

We now report results for relative growth rate from the entire dataset. Our aim here was to determine if additional data alters or reinforces the above conclusions, by increasing replication. For this analysis our dataset included correlations concerning relative growth rate exclusively (Fig. \ref{fig:fig3}), measured across at least two species (but using several individuals), and did not control for the source of trait or growth (i.e. traits could be measured on a different ontogenetic stage to growth data) or environmental conditions. The relaxed criteria increased the number of replicates, from $n = 137$ to $n = 167$ correlations across 93 articles. However, the additional replicates were unequally distributed across trait-stage combinations. Some trait-stage combinations that already had a lot of data got even more support (e.g. from 25 to 59 correlation coefficients reported for SLA at seedling stage), while others still had few replicates. On the 13 traits-stage combinations studied in total, six had $\le$ five correlation coefficients: wood density-seedling, maximum height-sapling, maximum height-seedling, seed mass-sapling, Aarea-adult and Aarea-sapling.

Results from the entire dataset reinforced our conclusions for both SLA and wood density, and did not alter them for the others traits. For SLA and wood density, although the number of replicates doubled by using the entire dataset, effect sizes and overall results were similar to those reported above using the conservative criteria (Fig. \ref{fig:fig3}). For seed mass and Aarea, the additional replicates in the entire dataset allowed us to be more confident of the effect size for seedling stage (Fig \ref{fig:fig3}d, e). Comparison of Aarea effect between the conservative and relaxed criteria (Fig. \ref{fig:fig3}d), indicated little effect of stress (here mainly low light) on the correlation with relative growth rate at seedling stage as the correlation was still positive when 8 correlations measured under low light conditions were added. For maximum height, there were no additional replicates in the entire dataset; we still cannot conclude undoubtedly that the correlation is absent for smaller sizes.

\subsection*{Comparisons between relative and absolute growth rates under relaxed criteria}

The only correlation that we hypothesized would change if absolute growth rate was used instead of relative growth rate, was for seed mass. Overall, the model fits did not improve when information about growth measurement (relative vs absolute growth rates) was incorporated in the model (see log likelihood ratio tests in Fig. \ref{fig:fig4}), thus supporting this claim. For a majority of traits, effect sizes were generally consistent between relative growth rate and absolute growth rate when both measurements were available (e.g. effect size for relative and absolute growth rate of Hmax and wood density at adult stage in Fig. \ref{fig:fig4} b and c). However, for seed mass and wood density at seedling stage, the direction of effect sizes changed when correlations were measured with absolute versus relative growth rate. For seed mass, effect size at seedling stage with relative growth rate was negative and shifted to non-significant with absolute growth rate (Fig \ref{fig:fig4}d; CI overlapping zero). While these results were consistent with expectations, our ability to draw any firm conclusions was limited due to low sample size for correlation with absolute growth rate (i.e CIs overlapping, Fig. \ref{fig:fig4} d). For wood density, some differences were observed between studies reporting relative and absolute growth rates: in particular, the correlation at juvenile stage shifted from negative with relative growth rate to non-significant with absolute growth rate (i.e CI included zero, Fig. \ref{fig:fig4}b). There may be several reasons for this difference, however, we note that the only available correlation coefficients recorded for absolute growth rate came from \citet{Augspurger:1984ct}, and unlike most other studies, they used wood density values estimated on adult plants together with growth data taken from seedlings. This suggest possible confounding between the type of growth measurement and origin of the trait data.

\subsection*{Coverage of the data and assessment of publication bias}

Given our basic conclusion that reported correlations shifted with plant size in a manner consistent with expectations from theory, we assessed the coverage of the available data, to determine if biased sampling could have erroneously generated the above results.

In total, 103 articles published between 1983 and 2014, and containing 551 correlations linking growth rate to any of the 19 traits across a range of ontogenetic stages were identified. Data reported here were variable in multiple ways. Climate zones spanned from temperate regions to tropics (map in Fig. \ref{fig:figA2}). Different experiment types (laboratory, field, plantation or greenhouse), growth measurements and growth form (tree, woody or across growth form) were included. The number of species used for each correlation ranged from 2 to 300 species, with an average of 31. Given this variety, it was not surprising to find a wide range of $r$ reported for each trait. For instance, $r$ varied from -0.98 to 0.99 for SLA, and from -0.74 to 0.74 for wood density (Fig. \ref{fig:figA3}). Growth form, experiment type and growth measurement type were not strong predictors in general (Fig. \ref{fig:figA6}); suggesting that these factors are not like explanations for the  major results reported.

Overall our study met quality criteria expected for meta-analysis in ecology (Table \ref{tab:meta}). The impact of publication bias was seemingly trivial (See Appendix \ref{app:heterogeneity}, Fig. \ref{fig:figA7}). The extent of heterogeneity was high, but was substantially explained by our core hypothesis of an effect of ontogenetic stage for SLA and seed mass (Appendix \ref{app:supp_info_analyses}). There was no indication of a temporal change in the effect size being reported within stages (Appendix \ref{app:supp_info_analyses}, Fig. \ref{fig:figA9}). Finally, our results were unlikely to be biased by dominance of one or a few studies that measured a large number of responses, since when multiple responses were condensed into a single average within a trait and within a study, outcomes were very similar (Fig. \ref{fig:figA5}).


\section*{Discussion}\label{discussion}

The meta-analysis reported here demonstrates that for several key traits, the influence of the trait on growth rate changes as plants grow from small to large. While the possibility has been raised recently that trait-growth correlations may change systematically with size \citep{Falster:2011ii, Ruger:2012jv, Iida:2014ep, Iida:2014hq}, this is the first time that specific mechanistic hypotheses have been corroborated against a wide array of existing studies. Meta-analysis allowed us to generalize and thus to reach a consensus about the effect of traits on growth, even if our conclusions were at times biased toward a particular growth stage, due to uneven coverage of the empirical literature across trait-growth-stage combinations. Growth rates were already known to change with size \citep{Condit:1993hd, Clark:1999ed, Herault:2011dd}, in other words individual plants follow a growth trajectory. Our results suggest species traits should be understood as influencing this trajectory and potentially re-ranking growth rates between individuals from different species as their size increases. This is a shift in outlook compared to much present thinking, whereby traits are seen as rendering species fast-growing or slow-growing throughout life.

\subsection*{Why does the effect of traits on growth vary with size?}

For five prominent traits, the patterns observed were consistent with specific predictions from a mechanistic growth model, although there were some limitations arising from unequal numbers of observation across ontogenetic stages. This suggests the mechanisms through which traits affect growth are correctly represented in the model. For SLA the meta-analysis strongly confirmed the predicted influence of plant size on the trait-growth correlation. For wood density, Aarea, seed mass and maximum height results were consistent with predictions, but data was limited and additional studies would be helpful for these trait-growth-stage relationships.

A single feature of plant growth underpins to the patterns associated with SLA, wood density, and seed size: the decrease in the ratio of leaf area to support tissue biomass with increasing plant size. Size-dependent requirements for mechanical and hydraulic support result in a substantial decline in the ratio of leaf biomass to total biomass with increasing plant size. Hence, the ratio of biomass production to total biomass and therefore relative growth rate must decline with increasing plant size \citep{Givnish:1995ta, Enquist:2007ek}, resulting in lower relative growth rates for large seedlings of large-seeded species than for small seedlings of small-seeded species \citep[reviewed by][]{Turnbull:2012ew}. For seedlings our study showed, as expected, a negative relationship between seed mass and relative growth rate (in mass, height, or volume) for small plants, that dissipated as plants grew larger.

Increasing support costs also devalue the benefits of cheap leaf construction (low SLA), because as plant's grow larger, an increasing amount of total biomass production is devoted to building and maintaining support tissues \citep{Givnish:1995ta,King-2011}. The net result is that the growth advantage experienced by high SLA species at small sizes, dissipates by the time large plante size are attained. While it has previously been noted that the correlation of SLA with growth rate disappears among saplings and larger trees (\citealt{Wright:2010tp} for Barro Colorado Island forest, Panama), our meta-analysis generalises that result. Several authors have suggested that the disappearance of SLA's influence on growth might be due to a disproportionate accumulation of leaf area by individuals with low SLA \citep{Reich:1992wm, Poorter:2008iu,Wright:2010tp}, yet it remains unclear whether such a mechanism works as intended. Further, the assumed empirical relationship between SLA and total leaf area does not hold true as a generalisation across large number of species \citep{Duursma-2015}. In another modelling study, the SLA that lead to optimal growth rate was found to decrease with plant size, because of the interplay with the degree of self-shading in the crown and the fact that support costs are greater in larger plants \citep{King-1999}. In the model described here, the mechanism is slightly different: high-SLA species incur faster leaf turnover, the cost of which outweighs the benefits of cheap leaf construction at larger sizes.

For wood density the meta-analysis showed negative correlations with relative growth rate at all ontogenetic stages, as previously reported from individual studies \citep{Wright:2010tp,Ruger:2012jv}. It should be noted however that there are few data available to test this proposition at seedling stage (n=1-4). In our model, cheaper stem construction is responsible for the predicted negative correlation. Lower xylem conductance in higher density sapwood and the associated lower photosynthetic potential, has also been suggested \citep[reviewed by][]{Chave:2009iy}.
For seedlings a negative correlation arose regardless of whether wood density was inferred from measurements made on adult or measured in fact in seedlings (e.g. stem density in  \citealt{CastroDiez:1998gz}), suggesting that both measurements could be used.

As far as we are aware, our study is the first to compare Aarea and growth correlations between small and large plants. In the growth model, a consistent correlation irrespective of size arises from a direct effect of Aarea on plant growth rate. Meta-analyses showed a positive correlation for seedling but not for saplings and adult (n=1). In the seedling stage, where most data were available, we noted no signal of an effect of low light on the positive correlation between Aarea and growth rate.

Meta-analysis showed a positive correlation between maximum height and relative growth rate for adult stage. The trend for tall stature species to grow faster at a given tree size than small stature species has been noted across several tree size classes and forest types (e.g. \citealt{Thomas:1996do,Poorter:2008iu,Wright:2010tp,Herault:2011dd,Ruger:2012jv,Iida:2014ep}). In \citet{Wright:2010tp}, maximum height was considered a weak predictor of growth rates of saplings; however it was the strongest predictor of growth rates of large trees. \citet{Poorter:2008iu} speculated that the maximum height-relative growth rate relationship among large trees might arise from including small tree species with maximum size not much larger than the 10cm dbh lower threshold used to define large trees. A possible mechanism for Poorter et al's observation is captured by our growth equations: maximum height affects growth via the fraction of energy diverted to reproduction. At a given tree size, species closer to their maximum height are expected to be allocating relatively less to vegetative growth. However this applies only at sizes where some of the species have begun to reproduce, and should not affect sapling and seedling growth rates. While the results of the meta-analysis not countered with the expected pattern for maximum height, we noted that a limited number of replicates were available for both saplings (n=4) and seedling stages (n=1). Consequently, subsequent research is welcomed to demonstrate this result on small plant sizes.

\subsection*{No signal of cross-correlations among different traits}

It is notable that the expected correlations between traits and growth do not appear to be much overridden by cross-correlations among different traits. A recurring issue in trait ecology is how many of the observed correlations arise directly from mechanisms, versus how many from secondary cross-correlation. For example, several papers have suggested that correlations between maximum height and growth rate may arise via cross-correlation with other traits such as wood density \citep{Thomas:1996do, Poorter:2006vb, Wright:2010tp}. These studies also suggest species towards one end of a trait spectrum may generally be found in higher light. Similarly, effects of seed mass on growth have been suspected to be mediated via other traits \citep{Shipley:1990js,Westoby:2002ft,Poorter:2006vb}. Secondary cross-correlation might also potentially have the effect of suppressing a correlation that would otherwise be present. While such effects may still be observed within individual studies, across the range of studies reported we found that the available data were consistent with predictions arising directly from mechanisms.

\subsection*{Future research directions}

Many biotic and abiotic conditions such as light and water micro-environments can affect growth rates, and might modify correlations with traits in relation to plant size. For example, for SLA, instead of a positive relationship with growth rates that disappeared with plant size, \citet{Iida:2014ep} found SLA was negatively correlated with relative growth rate at small size and positively correlated with relative growth rate at larger sizes. Similarly, instead of a negative correlation between wood density and growth across all ontogenetic stages, some studies found a negative correlation at small size that disappeared when trees were larger \citep{Iida:2014ep,Iida:2014hq}. The authors explained the switch in directionality by changes in light and water condition up the vertical profile within a forest: larger trees reached a position in the canopy where they were released from light limitations, other resources such as water may have become limiting, and the performance of trees may then have been regulated by different traits. Note that in \citet{Iida:2014hq}, the shift in the correlation between growth and wood density occurred for trees larger than 22-26 cm dbh. This points to one of the possible limitations in our meta-analysis, where trees with diameter above 10 cm were pooled as ``adult''.

Clearly there is much scope for strengthening our understanding of correlations between traits and growth through (1) adding further empirical studies for traits and size-classes with weaker representation, (2) accounting more satisfactorily for variables such as light, (3) expressing plant sizes on continuous scales of mass or height, rather than through the approximate and partly subjective categories unavoidably adopted here, and (4) moving beyond the qualitative model predictions evaluated here towards quantitative predictions, e.g. via the recently published \texttt{plant} model \citep{Falster-2016}. Additional data are specifically required in small plants (seedling and sapling) for wood density and maximum height, and in large plants (adult and sapling) for Aarea and more broadly for all traits requiring a measure of whole plant biomass (e.g. LMR, LAR). However, the existing published evidence is consistent with several qualitative predictions of a simple model that assesses the effects of plant size on growth. Thus, the linkage of growth to the mechanisms outlined here can shed light on tree growth trajectories and the size-dependent effects of functional traits on the changing species growth rate rankings as individuals grow larger.

\section*{Acknowledgements}\label{Acknowledgment}

We thank Andrea Stephens for helpful comments on the meta-analyses, and Sean Gleason et al for pre-publication access to their data. Charles Warren and Jordi Mart{\'{i}}nez-Vilalta kindly provided additional information about their published results. Elizabeth Wenk and Freya Thomas for their friendly review of our manuscript. This work was funded by the Australian Research Council through a fellowship to M.W, a discovery grant to D.S.F and a Macquarie University Research Excellence Scholarship and L'Oreal UNESCO FWIS Fellowship to E.F.G.

\subsection*{Availability of code and data}\label{code}

The code and data for producing all figures and results in this paper is available at \smurl{github.com/AnaisGibert/Growth\_trait\_metaanalysis}. Prior to publication, the data set will also be made available in the Dryad Digital Repository.

\linespread{1}
\bibliography{output/refs-main}\label{references}

\clearpage
\section*{Figures}

\begin{figure}[h!]
\centering
\includegraphics[width=15cm,height=20cm,keepaspectratio]{output/Fig1}
\caption{Number of recorded correlations between trait and growth rate in our dataset, by stage and trait. Traits are: Aarea (or Amass) = rate of CO$_{2}$ assimilation per unit leaf area (or leaf mass) = maximum photosynthetic rate, Hmax = asymptotic or maximum height, Ks = stem hydraulic conductance, LA = leaf area, LMR = leaf mass ratio, NARarea = net mass assimilation rate per unit of leaf area, Nmass (or Narea) = nitrogen content per unit of leaf mass (or leaf area), Pmass = phosphorus content per unit of leaf mass, SA/LA = sapwood area per leaf area, SLA = specific leaf area, Thickness = leaf thickness, Vessel size = stem conduit size, Vessel density = stem conduit density, WD = wood density. Results shown are from the entire dataset (see methods for details).}
\label{fig:fig1}
\end{figure}


\begin{figure}[h!]
\centering
\includegraphics[width=15cm,height=20cm,keepaspectratio]{output/Fig2}
\caption{The range of plant sizes and stages recorded in original publications (horizontal coloured lines), compared to standardised thresholds used in the analysis (vertical black lines). Horizontal line segments indicate the size range (max., min.) of the plants used to measure growth. Horizontal line colours refer to the original stage attributed by the authors .The solid vertical lines indicate the thresholds used to classify plants in one of three stages in our meta-analyses. Studies marked with a ``*'' were those where stage was changed or assigned, compared to the original publication.}
\label{fig:fig2}
\end{figure}



\begin{figure}[h!]
\centering
\includegraphics[width=15cm,height=20cm,keepaspectratio]{output/Fig3}
\caption{The relationship between relative growth rates (RGR) and traits changes with ontogenetic stage and trait. Plots show the effect size (+ 95\% confidence intervals; CI) for the average correlation by ontogenetic stage. Effect size is a standardized measure of the deviation of correlation coefficient from zero (Fisher's z-transformed). Effects are significant if CIs for a given stage do not overlap with zero. Trait names as in Fig. \ref{fig:fig1}. $n$ indicates number of correlations contributing to each effect. Results shown are for the "conservative dataset" (i.e. correlations that included \underline{\textgreater} 10 species, under unstressed conditions, and where the trait and growth were measured at the same stage) and for the entire dataset (i.e. with all criteria relaxed, see methods for details). Likelihood Ratio Tests (LRT) compared maximum-likelihood fits between a pooled model and one separated by stage for the conservative dataset. A significant p-value means adding a stage effect to the model significantly improved model fit.}
\label{fig:fig3}
\end{figure}


\begin{figure}[h!]
\centering
\includegraphics[width=15cm,height=20cm,keepaspectratio]{output/Fig4}
\caption{Relationships between relative (RGR) and/or absolute (AGR) growth rates and traits along ontogenetic stage. The legend is identical to Fig 3, with two exceptions. i) Results shown are from the entire dataset (see methods for details). ii) Likelihood Ratio Tests compared maximum-likelihood fits between a model separated by stage and one separared by stage and growth measurement (RGR vs AGR).}
\label{fig:fig4}
\end{figure}

\clearpage
\section*{Table}
\setcounter{table}{0}

\newcommand{\sepp}{{\color{grey}/}}

\begin{table}[h!]
\centering
\caption{Hypothesised effects of traits on key elements of plant function determining growth rate, as described in eq. \ref{eq:eq0} -- \ref{eq:eq4}. Arrows indicate the effect an increase in trait value would have on each element of the equations, with dashes indicating no effect. Traits are: specific leaf area (= SLA), wood density (= WD), Maximum height (= Hmax), Seed mass, and Maximum photosynthesis per unit leaf area (= Area) For further details, see main text.}
{\footnotesize
\vspace{1cm}
  \begin{tabular}{lllcccccc}
  \hline
  & & & {\bf Symbol} & {\bf SLA} & {\bf WD} & {\bf Hmax} & {\bf Seed mass} & {\bf Aarea} \\ \hline
  \\
  \multicolumn{9}{ l } {\textbf{Elements of equations \ref{eq:eq0} -- \ref{eq:eq4}}}  \\
  & \multicolumn{2}{ l } {Leaf deployment per mass}  & $\ud A_{\rm l} / \ud M_{\rm a} $ & \\
    & &     \tabitem leaf  &  & $\uparrow$ & - & - & - & - \\
    & &     \tabitem sapwood & & - & $\downarrow$ & - & - & - \\
    & &     \tabitem root & & - & - & - & - & - \\
  & \multicolumn{2}{ c } {Allocation to vegetative growth} & $\ud M_{\rm a} / \ud B$ & - & - & $\uparrow$ & - & - \\
  & & & & & & \\
  & \multicolumn{2}{ l } {Net live biomass production} & $\ud B / \ud t$ & & & & & \\
  & &     \tabitem photosynthetic rate & & - & - & - & - & $\uparrow$  \\
  & &     \tabitem respiration rate  & & - & - & - & - & $\uparrow$  \\
  & &     \tabitem leaf turnover rate & & $\downarrow$ & - & - & - & - \\
  & \multicolumn{2}{ l } {Architecture} & $\ud H / \ud A_{\rm l}$ & - & - & - & - & - \\
  & \multicolumn{2}{ l } {Sapwood area per leaf} & $ \ud A_{\rm ss} / \ud A_{\rm l}$ & - & - & - & - & - \\
  & \multicolumn{2}{ l } {Sapwood turnover} & $k_{\rm ss} \, A_{\rm ss}$ & - & - & - & - & - \\ \hline
  \\
  \multicolumn{9}{ l } {\textbf{Predicted effect on plant growth rate} (small plant {\sepp} large plant)} \\
  & \multicolumn{2}{ l } {in mass} \\
  & & \tabitem{Absolute} & $\ud M_{\rm a} / \ud t$ & $\uparrow$\sepp$\downarrow$ & $\downarrow${\sepp}$\downarrow$ &  $-${\sepp}$\uparrow$ & $\uparrow${\sepp}$-$ & $\uparrow${\sepp}$\uparrow$ \\
  & & \tabitem{Relative} & $\ud M_{\rm a} / (\ud t . M_{\rm a} )$ & $\uparrow${\sepp}$\downarrow$ & $\downarrow${\sepp}$\downarrow$ &  $-${\sepp}$\uparrow$ & $\downarrow${\sepp}$-$ & $\uparrow${\sepp}$\uparrow$ \\
  & \multicolumn{2}{ l } {in height} \\
  & & \tabitem{Absolute} & $\ud H / \ud t$ & $\uparrow${\sepp}$\downarrow$ & $\downarrow${\sepp}$\downarrow$ & $-${\sepp}$\uparrow$ & $\uparrow${\sepp}$-$ & $\uparrow${\sepp}$\uparrow$ \\
  & & \tabitem{Relative} & $\ud H / (\ud t . H)$ & $\uparrow${\sepp}$\downarrow$ & $\downarrow${\sepp}$\downarrow$ & $-${\sepp}$\uparrow$ & $\downarrow${\sepp}$-$ & $\uparrow${\sepp}$\uparrow$ \\
  & \multicolumn{2}{ l } {in diameter} \\
  & & \tabitem{Absolute} & $\ud D / \ud t$ & $\uparrow${\sepp}$\downarrow$ & $\downarrow${\sepp}$\downarrow$ & $-${\sepp}$\uparrow$ & $\uparrow${\sepp}$-$ & $\uparrow${\sepp}$\uparrow$ \\
  & & \tabitem{Relative} & $\ud D / (\ud t . D)$ & $\uparrow${\sepp}$\downarrow$ & $\downarrow${\sepp}$\downarrow$ & $-${\sepp}$\uparrow$ & $\downarrow${\sepp}$-$ & $\uparrow${\sepp}$\uparrow$ \\
\hline
  \end{tabular}
  }
\label{tab:trade-offs}
\end{table}

\clearpage
\begin{table}[h!]
\centering
\caption{Check-list of criteria for meta-analysis in ecology developed by \citet{Koricheva:2014ku}.}
{\footnotesize
\vspace{0.5cm}
\begin{tabular}{p{0.2cm}p{4cm}p{0.8cm}p{7cm}}
  \hline
  & Quality criteria & Check & How and where is it addressed?\\
  \hline
  1.& Reporting full details of bibliographic searches & yes  & Complete details of bibliographic searches (i.e. electronic reference data bases that were searched, keywords, language, and search dates) are available in the ``Material and Methods''.\\
  2.& Reporting inclusion or exclusion criteria & yes & Inclusion criteria are specified in the ``Material and Methods''. A full list of all articles evaluated is available in the Appendix \ref{app:literature-list}.\\
  3.& Weighting effect sizes by study precision & yes & Effects sizes were weighted by number of species. See ``Material and Methods'' for details.\\
 4.& Specifying meta-analytical model & yes & We use a mixed-effect model with a study ID as a random variable, and stage as a fixed effect. See ``Material and Methods'' for details. Results are shown in Fig. \ref{fig:fig3} \& \ref{fig:fig4}.\\
 5.& Quantifying heterogeneity in effect sizes &  yes & We estimate the extent of heterogeneity for each trait using the $I^{2}$ statistic. See Appendix \ref{app:supp_info_analyses} for details.\\
 6.& Exploring causes of heterogeneity & yes &  We explore the effect of stage, and the year of publication (see points 4 and 10 in this list).\\
 7.& Multifactorial analysis of explanatory variable  & yes & We check whether several other explanatory moderators (i.e. growth measurement, experiment type or growth form) are consistently the strongest predictors of the data. We do not test the multiple variables in a hierarchical fashion, because of the limited size of our dataset. See in Figs \ref{fig:figA6}.\\
  8.& Testing for publication bias &  yes & We used several methods (i.e. funnel plots, the ``trim and fill'' method, and Rosenberg's fail-safe number). See Appendix \ref{app:supp_info_analyses} for details.\\
  9.& Sensitivity analysis &  yes &See ``Material and Methods'' for details, results in Figs \ref{fig:fig4} \& \ref{fig:figA5}.\\
  10.& Exploring temporal changes in effect size &  yes & See Appendix \ref{app:supp_info_analyses} for details.\\
  11.& Controlling for phylogeny &  no &  It was not possible to control by the phylogeny, since the number of species used to measured each coefficient of correlation ranged from 2 to 300 species.\\
  12.& Specifying the software used & yes & See sections in ``Material and Methods'' and ``Availability of code and data''.\\
 13.& Providing reference list of primary studies included in the analysis &  yes & All articles from which data were extracted are cited in the paper.\\
  14.& Providing the data set used for meta-analysis  & yes & A copy of the data set is available. See ``Availability of code and data'' for details.\\
   \hline
\end{tabular}
}
\label{tab:meta}
\end{table}

\end{document}
