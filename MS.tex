\documentclass[a4paper]{article}\usepackage[]{graphicx}\usepackage[]{color}
\usepackage{alltt}
\usepackage{amssymb,amsmath}
\usepackage{palatino}
\usepackage{framed}
\usepackage{parskip}
\usepackage{graphicx}
\usepackage{fixltx2e}
\usepackage[default,osfigures,scale=0.95]{opensans}
\usepackage{geometry}
\usepackage{longtable}
\usepackage{authblk}
\usepackage[colorinlistoftodos]{todonotes}
\geometry{verbose,tmargin=2cm,bmargin=2cm,lmargin=2cm,rmargin=2.5cm}
\linespread{2}
\usepackage{color}




\usepackage[title,titletoc,toc]{appendix}
  

\usepackage{etoolbox}
\newbool{MyRefNumbers}
\booltrue{MyRefNumbers} 

\usepackage{natbib}

\bibliographystyle{ecol_let}
\setcitestyle{authoryear,open={(},close={)}}
\newcommand{\bibstar}{* }
 

\begin{document}


\clearpage

\title{Plant species traits and growth rates: meta-analysis shows correlations change with plant size as predicted}

\author[1]{Anais Gibert*}
\author[1]{Emma F. Gray}
\author[1]{Mark Westoby}
\author[1]{Ian J. Wright}
\author[1]{Daniel S. Falster}
\affil[1]{Dept of Biological Sciences, Macquarie University, Sydney, NSW 2109, Australia}




\maketitle

\textit{Running title (36/45 characters)} Trait influence on growth along plant size

\textit{Corresponding author} Anais Gibert, anais.gibert@gmail.com, Dept of Biological Sciences, Macquarie University, Sydney, NSW 2109, Australia, tel: +61 2 9850 8194


\clearpage

\section*{Abstract (195/350 words)}\label{abstract}


\begin{enumerate}

\item
Growth rates (GR) have often been seen as central elements of plant ecological strategies, and as linked to species traits. Yet, the literature is inconsistent about empirical correlations between functional traits and plant growth rate, casting doubt on the capacity of traits to predict growth.

\item 
Traits should influence growth in a way that depend on individual plants size. We outline mechanisms and hypotheses based on new theory, and test these predictions in tree species for five traits using a meta-analysis of 112 studies (> 500 correlations). As an approximation of plants size, we analysed the shift in the correlations between growth and traits across plant stages.

\item
Results were consistent with predictions. Specific leaf area was correlated with GR in small but not large plants. Correlations of GR with wood density and assimilation rate were not affected by size. Maximum height and seed mass were correlated with GR only in one plant size category.

\item 
Synthesis: We show that traits-GR correlations change in a predictable way as a function of plant size. Our understanding of plant strategies should shift away from attributing slow vs fast growth to species throughout life, in favour of attributing growth trajectories.

 
\end{enumerate}


\textit{Keywords (8/10)} asymptotic maximum height, maximum photosynthetic rate, plant ontogenetic stage, plant functional traits, plant growth strategy, seed mass, specific leaf area, wood density 


\section*{Introduction}\label{introduction}
Plant "ecological strategy" refers to how species face the challenge they all share: to acquire sufficient water, nutrients and light for individual growth and to live long enough to replace themselves with offspring. An idea that species strategy differences might involve fast vs slow growth goes back at least to Grime and Hunt \citeyearpar{Grime:1975gr}, who measured a wide spread of relative growth rates for seedlings during early exponential growth under favourable conditions. Seedling potential relative growth rate (RGR) has continued to be used as a major strategy indicator within the CSR (Competitor - Stress tolerator - Ruderal) classification (e.g. \citealt{grime1979plant, Grime:1997wm}). Meanwhile partitioning of seedling RGR into its components showed that specific leaf area (SLA = 1/leaf mass per area LMA) was typically the strongest source of variation between species \citep{Poorter:1989tx}. From the 1990s a new style of "trait ecology" arose whereby measurable traits including SLA were used directly as strategy axes \citep{Westoby:2002ft}. This approach via measurable traits made broad comparisons possible across continents and latitudes and thousands of species. Since then, traits have been seen as constructive approaches to understanding plant growth strategies \citep{Grime:1977kc,Chapin:1980gz}, community composition \citep{Lavorel:2002ff,Shipley:2006ie}, global vegetation dynamics \citep{Scheiter:2013ed} and ecosystem processes \citep{Lavorel:2002ff}.

Currently, two main spectra of variation are widely cited as underpinning differences between species in vegetative growth rates. The 'leaf economic spectrum' \citep{Wright:2004jb} and the 'wood economic spectrum' \citep{Chave:2009iy} reflect tissue construction costs (for specific leaf area and for wood volume respectively) trading off with tissue turnover rates or mortality risks. It has been widely expected that species with low tissue construction cost will typically have fast growth rates at least in favourable physical environments (e.g. \citealt{MullerLandau:2004dc,Wright:2004jb,Poorter:2008iu,Chave:2009iy,Larjavaara:2010bn,Iida:2012jb,Paine:2015df}). Yet, a growing number of empirical studies indicate the correlation between traits and growth is not as consistent as that. Consider SLA as an illustration. Species with high SLA can deploy more canopy area for a given dry mass investment in leaves \citep{Poorter:1999wd, Reich:1992wm}. In addition, SLA is correlated with high photosynthetic rates and high nitrogen (N) or phosphorus concentrations in the leaf \citep{Wright:2004jb}. For seedling relative growth rates under favourable conditions, high SLA has repeatedly been found to be a strong predictor regardless of vegetation type or growth form \citep{Lambers:1992bj,Reich:1992wm,Grime:1997wm,Poorter:1999wd,Wright:1999ds}. Yet for adult plants, this correlation appears weaker or absent \citep{coomes_comparison_1998,Poorter:2008iu,Aiba:2009ft,Easdale:2009gv,Wright:2010tp}. The growing discrepancy between theoretical expectation and empirical results has led some to question whether traits are even useful for understanding growth \citep{Wright:2010tp, Paine:2015df}.

Meanwhile the possibility has been raised that trait-growth correlations may change systematically with size \citep{Enquist:2007ek,Falster:2011ii, Ruger:2012jv, Iida:2014ep, Iida:2014hq}. Recent theory suggests that the absence of a correlation between SLA and growth at larger sizes may be expected, when potential influences of traits on plant function are considered \citep{Falster:2011ii, falster:2013}. At larger sizes the low-cost deployment benefits of high SLA may be offset or even outweighted by the high costs of rapid leaf turnover \citep{Falster:2011ii, falster:2013}. 
Recently, some empirical studies examined the relationships between functional traits and demographic rates across a wide range of tree sizes in specific experimental site (see \citealt{Iida:2014ep, Iida:2014hq}) and suggested a variation with plant size. However, it is still an opening question whether the influence of traits on growth change with plant size in a predictable and consistent way across different environemtal context.

In this study, we review published literature reporting on relationships between plant growth and five prominent traits (specific leaf area = SLA, wood or stem tissue density = WD, maximum leaf photosynthetic capacity per unit area = Aarea,  maximum height= Hmax, and seed mass= SM) in tree or woody species in order to assess the following questions. 

\begin{enumerate}
  \item How general is the tendency for correlations between traits and growth rate to vary with plant size?
  \item Do the observed relationships conform to predictions (outlined below) from a recently developed mechanistic model \citep{Falster:2011ii, falster:2013}?
  \item
\end{enumerate}

While most studies on traits have not been concerned with effects of plant stage or size, they have reported trait-growth correlations for identifiable size/stage classes. Thus, we gathered correlation coefficients between traits and growth from studies that gave explicit information about the size or stage of the plants. Meta-analytic approach offers several advantage, in particular it allows us to generalize about the influence of traits on growth across different ecological context.

\section*{Theoretical expectations}\label{theory}

Recent theoretical work \citep{Falster:2011ii} allows hypotheses to be formulated about relationships between traits and growth, based on a mathematical decomposition of plant growth rate. Specific predictions arise from considering the influence of traits on elements of these equations and how these effects may vary with size (see Appendix\ref{sec:growth} for mathematical explanations).

\textbf{H1: The relationship between specific leaf area (SLA) and growth rate is expected to shift from positive in seedlings to non-significant or negative in adult plants}

Plants with high SLA can deploy more leaf area per unit of mass invested. However, high-SLA species also have shorter leaf lifespan and faster leaf turnover rate \citep{Wright:2004jb}. Whether plant growth increases with SLA depends on the relative magnitude of the leaf construction and leaf turnover effects of SLA (Table 1). During early life, while leaf comprises a large part of the plant, decreasing leaf construction cost has a positive effect on growth rate. As plants increase in size and leaf mass fraction decreases, the benefit of cheaper leaf construction diminishes; thus the cost of replacing leaf turnover becomes more important. 

\textbf{H2: The relationship between wood or stem density (WD) and growth rate is expected to be negative across all plant sizes}

Low WD decreases the cost of building stem tissue and thus the cost of deploying additional leaf area (Table 1), which should result in faster growth. Although cheaper wood carries a risk of increased mortality, this cost is not reflected in growth outcomes. Thus decreasing WD is predicted to increase growth across the entire size range. 

\textbf{H3: The relationship between asymptotic maximum height (Hmax) and growth rate is expected to be absent in seedlings but to strengthen with increasing plant size.}

As plants approach maximum height they increasingly divert energy away from growth and into reproduction (see \citealt{Thomas:1996do, Falster:2011ii, Wenk:2014jz}). For seedlings, where all individuals are investing solely in growth, we expect no influence of Hmax on growth. As plants become larger, species with smaller Hmax are expected to be allocating more to reproduction and less to growth, because at any given size they are closer to their Hmax (Table 1). 

\textbf{H4: The relationship between seed mass (SM) and relative growth rate is negative in seedlings, but weakens with increasing size.}

The main effect of increased seed size is to produce a larger overall seedling. Generally relative growth rate decreases with size, and as such larger seedlings are predicted to have a lower relative growth rate during the earliest phases of growth \citet{Turnbull:2012ew}. However, among larger plants we expect the effect of this initial size to be minimal, and thus the correlation between seed size and growth rate should disappear.

\textbf{H5: The relationship between maximum photosynthetic rate (Aarea) and growth rate is expected to be positive across all plant sizes, at least under high-light conditions}

We expect greater photosynthetic capacity to translate into faster growth rates by increasing the rate of biomass production (Table 1), at least at high light where Aarea is measured. It is uncertain how much this might translate to faster growth in low light.

\textbf{Effect on absolute (AbsGR) and relative growth rate (RGR)}  

The correlation between trait and growth is predicted to revert according to the growth measurement used for both SLA and seed mass (see Appendix 1\ref{sec:growth}). Tree growth in diameter, height or mass can be expressed in either absolute (AbsGR) or relative terms (RGR). Here, the predicted relationship between growth and either SLA or seed mass in seedlings differs if absolute growth rate in mass is used instead of $RGR_{mass}$ (i.e. becomes negative for SLA, positive for SM). 


\section*{Materials and methods}\label{material-and-methods}

We focused on an initial list of 20 traits (Fig. \ref{Fig1}). These traits have been selected because they capture different aspects of plant species strategies, and are commonly addressed in the functional traits litterature. Specific leaf area (SLA), leaf area, leaf nitrogen and phosphorus concentration, net (mass) assimilation rates (NAR), and assimilation rate of CO$_2$ (Aarea) are traits involved in species leaf economic strategy \citep{Wright:2004jb,Wright:2010tp}. Seed mass (SM), and maximum plant height (Hmax) capture information about reproductive strategy \citep{Falster:2005bw,Moles:2006ft}. Wood density (WD) and sapwood area per leaf area (SA/LA), vessel size and density, and stem specific conductance are all involved in species wood economic strategy \citep{Chave:2009iy}. Here WD and SA/LA are also considered as indicators of stem construction cost \citep{Falster:2011ii}, whereas SLA and leaf thickness capture crucial information about leaf construction cost \citep{Wright:2004jb}. Finally, the leaf area ratio and leaf mass ratio are indicators of resource allocation to leaf \citep{Lambers:1992bj}.

We aimed to identify studies that measured the correlation between a functional trait and a growth measurement across species. We searched for relevant studies using a three step procedure. First, we identified traits for which a substantial number of studies would be available using a literature snapshot in Web of Science. While Web of Science guarantees a relatively stable and reproducible search environment, with clearly defined lists of indexed journals, its searches are restricted to metadata, such as title, keywords and abstracts \citep{Beckmann:2012hn}. This limitation does not apply to Google Scholar, where, theoretically, all text that is electronically available on the Internet will be indexed. Thus, our second step was to systematically read documents returned by Google scholar for a restricted set of traits. Finally, we identified and recorded other possible data sources using our own knowledge as researchers in the field (this included adding data from an as-yet unpublished study by Gleason et al). The complete list of studies where we inspected the full text is given in Appendix \ref{app:supp_info_files} in Supplementary material.

\subsection*{Literature snapshot}\label{literature-snapshot}

The Web of Science database (Thomson Reuters) was searched between March and June 2014 using a range of Boolean search terms in English. Search terms were as follows: (''NAME'' AND (''growth rate'' OR ''relative growth rate'')) AND (''tree*'' OR ''woody*''), ''NAME'' was one of trait names listed in TableA1. According to this snapshot, 62\% of the literature for the 20 traits (or 82\% for the 5 traits for which we have clear predictions) is captured by searching only on SLA and WD.

\subsection*{Systematic search}\label{systematic-search}

Google Scholar database was searched using the range of Boolean search terms for SLA and WD. Search terms were as follows: ((''SLA'' OR ''LMA'' OR ''specific leaf area'' OR ''leaf mass area'' OR ''wood density'' OR ''WD'' OR ''wood specific gravity'' OR ''stem density'' OR ''stem specific density'') AND (''growth rate'' OR ''relative growth rate'')) AND (tree* OR woody*). The keyword search in Google Scholar returns approx. 18900 papers. For comparison, with the same keyword search Web of Science returns 541 papers. Thus, in Google Scholar we displayed papers by relevance and examined the first 750 items. 
Each study had to meet some basic criteria to be included in our analysis. We restricted our meta-analysis to documents reporting: i) the relationship between growth and trait as a correlation coefficient, ii) the sample size and the number of species used to perform the correlation, and iii) the size/stage of plants. Where available we also recorded the geographic coordinates, number of sites, experiment type (greenhouse, field experiment, forest, shade-house etc.), growth measure (relative or absolute growth rate measured using plant diameter, height or mass), vegetation type, life form (tree, woody or mixed with herbaceous species etc.), biological scale (interspecific, intraspecific or intragenus), and experimental treatments (e.g.~fertilization, shade, salt stress). Ultimately data was included from 112 different studies (starred references in bibliography). The articles laid aside because they did not meet our criteria are listed in Appendix \ref{app:supp_info_files} in Supplementary material.

The number of species used to perform the correlation, the experimental conditions, and the concordance between stage used to measure trait and growth are clear criteria for assessing the quality of the data for addressing our hypotheses. Therefore, we focused on correlation across species, and we stratified the data set into two subsets: one including exclusively studies under unstressed conditions ('ideal' dataset), and the other set including studies under all conditions ('complete' dataset). We excluded from the 'ideal' dataset studies where growth and traits were measured on plants from different stages, unless a particular stage is by definition required to measure a trait such as for seed mass (i.e. seed stage), or Hmax (i.e. adult stage). We weighted our data by number of species rather than the sample size for all statistical analyses (see Statistical analyses section). Finally, to avoid publication bias due to selective reporting of significant results within published articles (e.g.~particular case of a 'failure to publish bias' see \citealt{Jennions:2013ta}), we asked the authors for additional information when they reported a non-significant correlation between growth and trait without reporting an r value.


\subsection*{Plant size categories}\label{plant-size-categories}

We acknowledge that it would be better to address plant size as a continous variable, but the limited number of correlations available in the literature along plant size constrains us to approximate plant size by plant stage. The authors of the 112 articles reported in our meta-analyses, categorized plants into four differents stages (i.e. seedling, juvenile, sapling, adult). They also reported plant size informations in diameter, height or age (generally the authors give the size of the smallest and the biggest plants species used in their analyses). 

While the stage categories have been established according to the expertise and the appreciation of the authors, by compiling data across multiple studies and across different growth form (tree, woody, etc), there is a risk of assigning plants of similar size to different categories. Thus, we tested how consistent were the range values in size/age attributed to each stage across the 112 studies, and across the different growth form (tree, woody and across growth form, (Fig. \ref{Fig2}). Suprisingly, the threshold in size/age used by the authors to defined a plant stage are quite steady across the 112 studies (Fig. \ref{Fig2}). In particular, there was a clear break between seedlings and saplings in both age (at 1 yr) and height (at 0.5 m); there was also a clear break between saplings and adults in diameter (at 10 cm). The distinctions between sapling-adult in age (around 5 yrs) and height (around 2 m), and between seedling-sapling in diameter (around 1cm) were less pronounced. We used simple allometric relationships from the Biomass And Allometry Database (BAAD) \citep{Falster:2015} to ensure that these size/stage cutoffs match across the different size measurements (i.e. diameter, height and age, Fig. \ref{FigA1}). For instance, we showed according to BAAD that a sapling with a diameter < 10 cm would have a height < 1 m. Thus, according to the size/stage threshold established across 112 studies, the plant will be assigned to the sapling stage whatever we focused on its diameter or height (see Fig. \ref{FigA1} c).

There were an ambiguity on the plant stage attribution for less than 5\% of the studies (i.e. 5 studies on 112). We re-red carefully each of these studies, looking at any particular reason addressed by authors that could explained these discrepancies. Interestingly, these discrepancies were not due to a differences between growth form; all the ambiguous cases corresponds to studies focusing exclusively on tree species (Fig. \ref{FigA1}). In fact, we identified two main explanations. First, the term "juvenile" was vague enough to cover either seedling or sapling stages according to the authors. Second, the reproductive state of a plant is not strictly correlated to plant size; some plants with a large stature are not assessed as adult because the authors selected them to be non reproductive (See ID 50, 52 and 105 in Fig. \ref{FigA1}). Thus, we attributed a study to a new stage category when 95\% of the recorded range of sizes for that study lay within a new category. And, we attributed a study to the adult stage when the authors identified the plants as reproductive regardless the plant size or age. The size/stage threshold defined across the 112 articles, allowed us to attribute straightforwardly a stage category to studies for which no stage have been distinctly mentionned by the authors (5\% of the data, noted as 'mix' in Fig. \ref{Fig2}). Finally, we run the analyses with both the old and the standardized categories to ensure that the different stage attribution will not disturb our results. 

Results using the categories established by the authors are similar than the one using the standardized categories and are available in Fig. \ref{FigA5} model.1. In the main text, we report the results using the standardized categories. In the rest of the mansucript, we preferred the term "juvenile" to "seedling", because for a portion of the scientific community "seedling" also refers to a particular experimental condition (a greenhouse experiment).


\subsection*{Effect size calculation}\label{effect-size-calculation}

We used the cross-species Pearson product-moment correlation coefficient r between growth rate and species trait as our measure of effect size, weighted by the number of species contributing. When studies did not report r values but another coefficient such as coefficient of determination (R2), Spearman's rho rank correlation or Kendall's tau rank correlation, we converted these to correlation coefficients  \citep{Lajeunesse:2013tm} (see code in Appendix \ref{app:supp_info_files} for details). When the magnitude of Pearson's correlation coefficient approaches +/- 1, its distribution becomes skewed. Accordingly we transformed the correlation r into a metric with desirable statistical properties using Fisher's z-transformation,

\[ z = \frac{1}{2} \ln \frac{1+r}{1-r} \]

which has a variance estimate

\[ v_z= \frac{1}{n-3} \]

n being the number of species.

\subsection*{Statistical analyses}\label{statistical-analyses}

We tested stage effect on traits for which > 15 correlation coefficients across at least two stages were recorded. We used a mixed-effect model of meta-analysis, in which it is assumed that studies within a stage share a common mean effect but that there is also random variation among studies in a stage, in addition to sampling variation. This random effect allows us to account for the heterogeneity among studies (e.g.~experimental type, vegetation type, error, etc.). For each trait, we established if stage could explain variance in the coefficient correlation by comparing models with the stage variable to the null model using log-likelihood ratios \citep{Zuur:2009cfa}. If the log-likelihood test returned a significant p-value, the stage model fitted the data better than the null model. Estimates of the effect size for each stage were considered significant if the 95\% confidence intervals did not overlap with zero. In the case of multiple comparisons within a study, we used the method proposed by \citet{Borenstein:2009um}: we calculated the mean correlation coefficient by trait and by study, and re-calculated the effect size. Following recent recommendations for the adoption of a checklist of quality criteria for meta-analysis in ecology (\citealt{Koricheva:2014ku}), we conducted several analyses to assess the reliability of our meta-analysis. Table 2 reported how and where (i.e in the text, appendix and/or figures) we addressed the 14 methodological quality criteria listed by \citet{Koricheva:2014ku}. 


\subsection*{Availability of code and data}\label{code}

All analyses were conducted with R software \citep{Ralanguageanden:2014wf}. The code and data underpinning this paper is available at github.com/AnaisGibert/Growth\_trait\_metaanalysis. Archival copies are also included in Appendix \ref{app:supp_info_files} in Supplementary material.

\section*{Results}\label{results}

\subsection*{Main effects}

For both SLA and SM, the correlation between growth rate and the trait changed significantly with plant size in the manner expected. Model fits improved when stage was incorporated (see log likelihood ratio tests, Fig. \ref{Fig3} a and d; average values for the coefficients r ($r_{\textrm{mean}}$) in Fig. \ref{FigA4}). The correlation between growth and SLA shifted from strongly positive at juvenile stage $r_{\textrm{mean}}=0.6 \pm 0.08$, Mean +SD) to not significantly different from zero at adult stage ($r_{\textrm{mean}}= -0.01 \pm 0.07$, CI 95\% overlapping zero in Fig. \ref{Fig3} a). The correlation between growth and SM shifted from negative at juvenile stage ($r_{\textrm{mean}}= -0.59 \pm 0.07$) to not significantly different from zero at adult stage ($r_{\textrm{mean}}= -0.09 \pm 0.02$, , CI 95\% overlapping zero in Fig. \ref{Fig3} d). For both traits, correlation coefficients recorded at sapling stage were intermediate between juvenile and adult stages. 

For WD, Hmax and Aarea, correlations with growth were also reasonably consistent with expectations, though with limitations arising from very unequal numbers of observations between growth stages. For Hmax, the expectation was for a correlation with growth to be absent for small plants but present at intermediate and larger sizes, the mechanism being increased relative allocation to reproduction as Hmax is approached. Indeed Hmax was positively correlated with growth at adult stage ($r_{\textrm{mean}}= 0.38 \pm 0.05$, n=18). At sapling and juvenile stages, the correlation between Hmax and growth was not significantly different from that observed at adult stage (model fit not improved when stage was incorporated, see log likelihood ratio tests in Fig. \ref{Fig3} c), but also was consistent with no correlation (Fig \ref{Fig3} c, confidence intervals overlapping zero), due to much smaller numbers of replicate studies. Similarly for WD and for Aarea, results were consistent with negative (WD) and positive (Aarea) correlations with growth across all plant sizes (Fig. \ref{Fig3} b and e, model fit not improved when stage was incorporated). However, at the same time confidence intervals of r often spanned zero especially where there were small numbers of studies for particular growth stages. 

\subsection*{Coverage of dataset}

Given our basic conclusion that reported correlations responded to plant size in a manner consistent with expectations from theory, we assess now the coverage of the available data, to determine if biased sampling could have erroneously generated the above results. The results reported were confined to correlations across species, and under non-stressed conditions (n=176). A "complete" dataset (n=216) also included correlations measured under stressed conditions, and spanning different plant stages between the growth and trait measurements. This "complete" dataset generated similar effect sizes to those reported above from the "ideal" dataset, and led to similar conclusions (Fig. \ref{Fig3} grey points). Differences concerned often the confidence intervals around particular mean effect sizes in Fig. \ref{Fig3}. These differences in CI were not easily attributed to any single cause. For instance, the correlation between growth and WD at juvenile stage shifted from negative to non significant (i.e CI95\% included zero) due to 2 positive correlation coefficients from \citet{Augspurger:1984ct};  these correlations unlike others used wood density values from adult plants together with growth from juveniles. For Aarea, the comparison between complete and ideal data showed no signal of an effect of stress (here mainly low light) on the correlation with growth at seedling stage; the correlation was still positive between Aarea and growth even when 7 correlations measured under low light conditions were added. Finally, our results were unlikely to be biased by dominance of one or a few studies that measured a large number of responses, since when multiple responses were condensed into a single average within a trait and within a study, outcomes were very similar (Fig. \ref{FigA5}).

We were pleasantly surprised to find that correlations had been reported for at least two different stages across 18 traits out of the 20 targeted (Fig. \ref{Fig1}). We had thought it likely that for some traits correlations would be reported only for one size class, or only mixed across different sizes. These limitations applied only for LAR and NARmass, where studies appeared confined to juvenile stages. However, among the 20 traits studied we had enough data to test the effect of stage on the correlation between growth and traits only for the five for which we had predictions. 

We found 112 studies published between between 1983 and 2014 that met our requirements. Climate zones spanned from temperate regions to  tropics (map in Fig. \ref{FigA2}). Different experiment types (laboratory, field, plantation or greenhouse), growth measurements and growth form (tree, woody or across growth form) were included. Number of species used for each correlation ranged from 2 to 300 species with average 31. Given this variety, it was not surprising to find wide ranges of r reported for all traits. For instance, r varied from -0.98 to 0.99 for SLA, and from -0.74 to 0.74 for WD (Fig. \ref{FigA3}). The growth form, the experiment types and the growth measurement were not markedly best predictors of the data (results shown in the Fig. \ref{FigA5}). 

We met all the quality criteria required for meta-analysis in ecology (Table 2). We showed that the impact of publication bias is probably trivial in our meta-analyses (Appendix D, Fig. \ref{FigA7}). The extend of the heterogenity was high in our meta-analysis, and substantially explained by the plant stage for SLA and Seedmass as expected (Appendix D). Finally, temporal change in effect size did not jeopardizing the stability of our conclusions for all the traits (Appendix D, Fig. \ref{FigA9}).


\section*{Discussion}\label{discussion}

The meta-analysis reported here demonstrates that the influence of traits on growth rates changes as plants grow from small to large, for several key traits. While this point has been made recently in smaller scale studies, it is the first time that mechanistic hypotheses are confirmed across a large number of studies already available. Meta-analysis allowed us to generalize and help to reach a concensus about trait influences on growth. Thus, plant growth rates change with size through ontogeny \citep{Condit:1993hd, Clark:1999ed, Herault:2011dd} - \textit{they follow a trajectory of growth} - and species traits should be understood as influencing this trajectory and consequently rearranging the ranking of growth rates with size, rather than as rendering species fast-growing or slow-growing throughout life. 

\subsection*{Traits influence on growth along plant size}

Further, for five prominent traits, the patterns observed are consistent with specific predictions from a mechanistic growth model. This indicates provisionally that the mechanisms through which traits affect growth are correctly understood in the model. For specific leaf area (SLA) and for seed mass (SM) the meta-analysis confirms quite strongly the predicted influence of plant size on the trait-growth correlation. For wood density (WD), assimilation rate of CO2 (Aarea) and maximum height (Hmax) results were consistent with predictions, but coverage was limited and additional studies would be helpful for some trait-growth-stage relationships. 

It has previously been noted that the correlation of SLA with growth disappears among saplings and larger trees (\citealt{Wright:2010tp} for Barro Colorado Island forest, Panama), and this meta-analysis generalises that result. Several authors have suggested that the disappearance of SLA's influence on growth might be due to a disproportionate accumulation of leaf area by low SLA species \citep{Reich:1992wm,Poorter:2008iu,Wright:2010tp}, yet it remain uncear whether such a mechanism works as intended. More importantly, the assumed empirical relationship between SLA and total leaf area does not hold true across large number of species (Duursma and Falster, in review).In the model described here, the mechanism is slightly different: high-SLA species incur faster leaf and other tissues turnovers, the cost of which outweighs the benefits of cheap leaf construction at larger sizes.

For seedlings our meta-analyses showed as expected a negative relationship between seed mass and relative growth rate (in mass or height or volume) that disappeared across plant stage. This negative relationship arises because seedlings from larger seeds are larger to begin with, as extensively discussed in the literature \citep[reviewed by][]{Turnbull:2012ew}. It applies only to RGR not to absolute growth rate, but for seedlings growth is most often reported as RGR (Appendix Fig. \ref{FigA8} d).

For WD the meta-analysis showed negative correlation with growth across all plant stages, as previously reported from individual studies \citep{Wright:2010tp,Ruger:2012jv}. It should be noted however that there are few tests of this proposition at seedling stage. In our model (Text Box 1 \ref{sec:growth}) cheaper stem construction is responsible for the predicted negative correlation. Lower xylem conductance and associated lower photosynthetic potential have also been suggested \citep[reviewed by][]{Chave:2009iy}. 

For the correlation between Aarea and growth, to our knowledge no study has previously assessed this in relation to plant size. In the growth model (Text Box 1 \ref{sec:growth}), a consistent correlation irrespective of size arises from a direct effect of Aarea on plant growth rate. In juvenile stage, where most of the data were, we noted no signal of an effect of low light on the positive correlation between Aarea and growth rate.

Meta-analysis showed a positive correlation between Hmax and growth for adult stage, but not for saplings and juveniles. The trend for tall stature species to grow faster at a given tree size than small stature species has been noted across several tree size classes and forest types (e.g. \citealt{Poorter:2008iu,Wright:2010tp,Herault:2011dd,Ruger:2012jv,Iida:2014ep}). In \citet{Wright:2010tp}, Hmax was considered as the weakest predictor of growth rates of saplings; however it was the strongest predictor of growth rates of large trees. \citet{Poorter:2008iu} speculated that the Hmax-RGR relationship among large trees might arise from including small tree species with maximum size not much larger than the  10cm dbh lower threshold used to define large trees. In our growth equations, Hmax affects growth via the fraction of energy diverted to reproduction. At a given tree size, species closer to their Hmax are expected to be allocating relatively less to vegetative growth. However this applies only at sizes where some of the species have begun to reproduce, and should not affect sapling and juvenile growth rates. 

\subsection*{No signal of cross-correlations among different traits}

It is notable that the expected correlations between traits and growth do not appear to be much overridden by cross-correlations among different traits. A recurring issue in trait ecology is how many of the observed correlations arise directly from mechanisms, versus how many from secondary cross-correlation. For example, several papers have suggested that correlations between Hmax and growth rate may arise via cross-correlation with other traits such as WD \citep{Thomas:1996do, Poorter:2006vb, Wright:2010tp}. Similarly, effects of SM on growth have been suspected to be mediated via other traits \citep{Shipley:1990js,Westoby:2002ft,Poorter:2006vb}. Secondary cross-correlation might also potentially have the effect of suppressing a correlation that would otherwise be present. While such effects may still be observed within individual studies, across the range of studies reported we found that the available data were consistent with predictions arising directly from mechanisms. 

\subsection*{Future research directions}

Light and water micro-environments can affect growth values, and might modify correlations with traits in relation to plant size. For example, instead of a positive relationship that disappeared with plant size, \citet{Iida:2014ep} found SLA was negatively correlated with RGR at small size and positively correlated with RGR at larger sizes. Similarly, instead of a negative correlation between WD and growth across all plant stages, \citep{Iida:2014ep,Iida:2014hq} found a negative correlation at small size that disappeared when trees were larger. \citet{Iida:2014hq} explained the switch in directionality by changes in light and water condition up the vertical profile within a forest: larger trees reached a position in the canopy where they were released from light limitations, other resources such as water may have become limiting, and the performance of trees may then have been regulated by different traits. Note that in \citet{Iida:2014hq}, the shift in the correlation between growth and WD occurred for trees larger than 22-26 cm dbh. This points to one of the possible limitations in our meta-analysis, where trees with height above 5 m were pooled as "adult". 

Clearly there is much scope for strengthening our understanding of correlations between traits and growth through (1) adding further empirical studies for traits and size-classes with weaker representation, (2) accounting more satisfactorily for variables such as light, and (3) expressing plant sizes on continuous scales of mass or height rather than through the approximate and partly subjective categories unavoidably adopted here. Additional data are specifically required in small plants (juvenile and sapling) for WD and Hmax, and in large plants (adult and sapling) for Aarea and more broadly for all traits requiring a measure of whole plant biomass (e.g. LMR, LAR, NAR). Nevertheless the existing published evidence, even with all its limitations, is consistent with predictions from a simple growth model, and this suggests that species traits can shed light on their growth trajectories, and hence on the changing growth rate rankings across species as individuals become larger. 

\section*{Acknowledgments}\label{Acknowledgment}

We thank Andrea Stephens for helpful comments on the meta-analyses, and Sean Gleason et al for pre-publication access to their data. Charles Warren and Jordi Martìnez-Vilalta kindly provided additional information about their published results. This work was funded by the Australian Research Council through a fellowship to Westoby and a discovery grant to Falster.  

\clearpage
\linespread{1}

\nocite{*}
 
\bibliography{output/refs}\label{references}


\clearpage
\section*{Figures}

\begin{figure}[h!]
\centering
\includegraphics{output/Fig1.pdf}
\caption{Number of correlations between trait and growth for each plant stage recorded in our dataset. Shown here are 'raw dataset' (recorded both across and within species, under unstressed and stressed conditions). Bar colors: red for adult, orange for sapling, grey for juvenile. Aarea (or Amass) = rate of $CO_{2}$ assimilation per unit leaf area (or leaf mass) = maximum photosynthetic rate, Hmax= asymptotic maximum height,  Ks =stem hydraulic conductance, LA = leaf area, LMR = leaf mass ratio, NARarea (or NARmass) = net mass assimilation rate per unit of leaf area (or leaf mass), Nmass (or Narea) = nitrogen content per unit of leaf mass (or leaf area), Pmass = phosphorus content per unit of leaf mass, SA/LA = sapwood area per leaf area, SLA = specific leaf area, thickness = leaf thickness, Vessel size = stem conduit size, vessel density = stem conduit density, WD = wood density}
\label{Fig1}
\end{figure}

\begin{figure}[h!]
\centering
\includegraphics{output/Fig2.pdf}
\caption{Range of plant size in \textbf{a)} age, \textbf{b)} height and \textbf{c)} diameter and the plant stage assigned by the authors of the articles used in the meta-analyses. Each segment represents the plant size range (max and min values) of the species used to measure the correlation between growth and traits in each article. ID is the reference number of the article in our dataset. The colors represents the stage information given by the authors (green: seedling, blue: juvenile, orange: sapling, red: adult, grey: unspecified by the authors), the vertical line represents the most likely limits between stage categories, the horizontal line styles corresponds to the species growth form used in each study (solid line: exclusively tree, dashed line: woody species , dotted line: tree mixed with another growth form). Shown here are the "raw dataset" (recorded both across and within species, and under unstressed and stressed conditions).}
\label{Fig2}
\end{figure}

\begin{figure}[h!]
\centering
\includegraphics{output/Fig3.pdf}
\caption{The relationship between growth rates and SLA or seed mass changes with plant stage. Plots show the effect size (+ 95\%CI) of the correlation across species between growth rate and five functional traits for plant stage comparisons. \textbf{a)} Specific leaf area (SLA), \textbf{b)} Wood density (WD), \textbf{c)} Asymptotic height (Hmax) \textbf{d)} Seed mass and \textbf{e)} Assimilation rate of CO2 (Aarea). Effect size is a standardized measure of the deviation of correlation coefficient from zero (Fisher's z-transformed). Effects are significant if confidence intervals (95\%CI) for a given stage do not overlap with zero. Shown here are 'across species dataset' (recorded only across species, and under both unstressed and stressed conditions). Black point: unstressed conditions, grey points: all the data. 'n' indicates number of correlations contributing to each effect. Likelihood Ratio Tests (LRT, for unstressed conditions only) compared maximum-likelihood fits between a pooled model and one separated by stage. A significant p-value means the stage model fitted the data better than the null model. 
}
\label{Fig3}
\end{figure}

\clearpage
\section*{Table}
\setcounter{table}{0}
\begin{table}[h!]
\centering
\caption{\textbf{Hypothesised effects of traits on key elements of plant function determining growth rate.} Arrows indicate the effect an increase in trait value would have on the main elements of growth eqs. \ref{eq:eq0}-\ref{eq:eq4} in Text box 1:  Leaf deployment per mass ($\frac{\textrm{d}a_\textrm{l}}{\textrm{d}m_\textrm{t}}$), Allocation to vegetative growth ($\frac{\textrm{d}m_\textrm{t}}{\textrm{d}b}$), Net live biomass production ($\frac{\textrm{d}b}{\textrm{d}t}$), Sapwood turnover ($k_s a_s$). SLA: Specific leaf area, WD: Wood density, Hmax: Asymptotic maximum height, SM: seed mass, Aarea: Maximum photosynthetic rate. For further details, see Text Box 1.}
\vspace{1cm}
  \begin{tabular}{l p{3cm} p{2cm}p{4cm} p{2cm}}
  & Leaf deployment \newline per mass
  & Allocation to vegetative growth
  & Net live biomass \newline production
  & Sapwood turnover\\
  & & & &\\\hline
  & & & & \\
Increase in : & & & & \\ 
  SLA &$\nearrow$ &$\_$ & $\searrow$ \newline (via $\nearrow$ of tissue turnover) &$\_$ \\
  WD & $\searrow$ &$\_$  &$\_$ &$\_$ \\
  Hmax &$\_$ &$\nearrow$ &$\_$ &$\_$ \\
  SM &$\_$&$\_$&$\_$& $\_$\\ 
  Aarea &$\_$ &$\_$ & $\nearrow$ &$\_$ \\
\hline
  \end{tabular}
  \end{table}
\label{tab:trade-offs}


\clearpage
\begin{table}[h!]
\centering
\caption{\textbf{Checklist of quality criteria for meta-analysis in ecology} developed by Koricheva and Gurevitch (2014) }
\label{Table 2}
\vspace{0.5cm}
\begin{tabular}{p{3cm} p{1cm} p{10cm} p{3cm}}
  \hline
  Quality criteria & Check & Explanations & Addressed in \\
  \hline
  1. Reporting full details of bibliographic searches & yes	& Google Scholar database was searched using the range of Boolean search terms ((''SLA'' OR ''LMA'' OR ''specific leaf area'' OR ''leaf    mass area'' OR ''wood density'' OR ''WD'' OR ''wood specific gravity'' OR ''stem density'' OR ''stem specific density'') AND (''growth rate'' OR ''relative growth rate'')) AND (tree* OR woody*), in english. & "Systematic search", Table A1 \\
  2. Reporting inclusion/exclusion criteria & yes & Studies included report i) correlation coefficient between growth and trait, ii) the sample size/ number of species, and iii) the size/stage of plants. Studies excluded report correlation within species	& "Systematic search" \\
  3. Weighting effect sizes by study precision & yes & Effects sizes were weighted by number of species. The question addressed in our study concerns the correlation between growth and trait across species, thus the number of species is a better proxy of a study quality than sample size. &	 "Systematic search" and  "Effect size calculation" \\
 4. Specifying meta-analytical model &	yes	& Mixed-effect model with a study ID as a random variation, and stage as a fixed effect, m1 ~lmer (stage + (1|ID)) &	"Statistical analyses", Fig3 \\
 5. Quantifying heterogeneity in effect sizes &	yes &	We estimated the extend of the heterogeneity using the statistics $I^{2}$ (Santos and Nakagawa, 2012) for each trait. &	Appendix D \\
 6. Exploring causes of heterogeneity &	yes &	We explored two causes of variability: the stage and the year of publication & Appendix D, Fig3 \\
 7. Multifactorial analysis of explanatory variable	& yes & We focused on the "plant stage" as the main exploratory variable. Yet, the stage variable may be confunded with other explanatory moderators (i.e. growth measurement, experiment type or growth form). Thus, we checked if no one of them would not consistenly be a strongest predictor of the data. We did not test the multiple variables in a hierarchical fashion, because of the limited size of our dataset. & FigA4, FigA6  \\
  8. Testing for publication bias &	yes	& We used funnel plots for a visual assessement. We  used the "trim and fill" method to estimate the number of studies missing (Duval and Tweedie 2000). We also estimated the Rosenberg’s fail-safe number (Rosenberg 2005). 	&  Appendix D, FigA7\\
  9. Sensitivity analysis &	yes &	We tested how robust are our results to the decision made in the analysis. We calculated effect size using 3 differents dataset: i) an  "ideal" dataset, focusing on unstressed conditions exclusively, ii) a "complete dataset" reporting all conditions, and iii) a "reduced dataset" where multiple responses were condensed into a single average within a trait and within a study. &	Fig3, FigA5 \\
  10.Exploring temporal changes in effect size &	yes	& There may be a confounding effect between the publication year and the stage. Thus, we explored the publication effect on the effect size within each stage for traits where a stage effect had been highlighted (SLA and Seedmass). & Appendix D, FigA9 \\
  11. Controlling for phylogeny &	no &	It was not possible to control by the phylogeny since the number of species used to measured each coeffcient of correlation ranged from 2 to 300 species. &	\\
  12. Specifying the software used &	yes	& R softward, code available on Github. & "Availability of code and data" \\
 13. Providing reference list of primary studies included in the analysis &	yes &	Available on Github. The Bibtex file "meta-analyses.bib" corresponds to all the articles included in the meta-analyses, "read.bib" corresponds to all the articles (included and excluded) inspected for the meta-analyses.	& "Availability of code and data" \\
  14. Providing the data set used for meta-analysis  &	yes &	A copy of the complete data set is available on Github.	& "Availability of code and data" \\
   \hline
\end{tabular}
\end{table}


\clearpage
\section*{Text Box 1} \label{sec:growth}

\subsection*{Derivation of hypotheses from growth equations} 

We adapted growth equations from \cite{Falster:2011ii} and \cite{falster:2013} to generate hypotheses on both absolute and relative growth rates in mass (eqn \ref{eq:eq1}), height (eqn \ref{eq:eq2}) and diameter (eqn \ref{eq:eq4}). 

The rate at which \textbf{live biomass} $b$ is produced by a plant is given by:

\begin{equation}\label{eq:eq0}
\underbrace{\color{green}\frac{db}{dt}}_{\text{net live biomass production}} = \underbrace{y}_{\text{yield}}  \big(\underbrace{a_{l} A_{\textrm{area}}}_{\text{photosynthesis}} - \underbrace{\sum_{i=l,s,r,f} m_{i} r_{i}}_{\text{respiration}}\big) - \underbrace{ \sum_{i=l,s,r,f} m_{i} k_{i}}_{\text{turnover}},
\end{equation}

where $y$ is the ratio of carbon fixed in mass per carbon assimilated,  and  $A_{\textrm{area}}$ is the assimilation rate of CO$_{2}$ per leaf area. 

Here, and in the following equations, $m_i$ denotes the \textbf{live mass on the plant}, $r_i$ the mass-specific \textbf{respiration rate}, and $k_i$ the \textbf{turnover rate} of different plant tissues; the subscripts $i=l,s,r,t,f$ referring to leaves, sapwood and bark, roots, total of these vegetative tissues, and to the reproductive tissues (fruits, flowers etc.) (note that the live biomass $b= m_t +m_f$). Similarly $a_i$ denotes \textbf{areas}, of leaves ($i=l$) and of cross-sections of total stem and sapwood ($i= st,ss$) respectively.


The \textbf{growth rate in mass}, being the increment in vegetative live mass ($m_t$), is then: 

\begin{equation}\label{eq:eq1}
\underbrace{{\color{magenta}\frac{dm_t}{dt}}}_{\text{growth rate in mass}} = \underbrace{\color{blue}\frac{dm_t}{db}}_{\text{allocation to vegetative growth}}  \times \underbrace{\color{green}\frac{db}{dt}}_{\text{net live biomass production}},
\end{equation}

where allocation to growth is the fraction of net live biomass increase that goes to vegetative growth rather than to reproductive tissue production. 

The \textbf{growth rate in height} can then be expressed as:

\begin{equation}\label{eq:eq2}
\underbrace{\frac{dH}{dt}}_{\text{growth rate in height}} = \underbrace{\frac{dH}{da_{l}}}_{\text{architecture layout}} \times \underbrace{{\color{orange}\frac{da_{l}}{dm_{t}}}}_{\text{leaf deployment per mass}} \times {\color{blue}\frac{dm_t}{db}} \times {\color{green}\frac{db}{dt}}. 
\end{equation}

Similarly \textbf{growth rate in basal area} can be expressed as:

\begin{equation}\label{eq:eq3}
\underbrace{{\color{cyan}\frac{da_{st}}{dt}}}_{\text{basal area growth rate}} = \underbrace{\frac{da_{ss}}{da_{l}}}_{\text{sapwood area per leaf area}} \times {\color{orange}\frac{da_{l}}{dm_{t}}} \times  {\color{blue}\frac{dm_t}{db}} \times {\color{green}\frac{db}{dt}} +  \underbrace{ k_{s} a_s }_{\text{sapwood turnover}}.
\end{equation}

The \textbf{growth rate in diameter} is then given by the geometric relationship between diameter and area applied to stem:
\begin{equation}\label{eq:eq4}
\underbrace{\frac{dD}{dt}}_{\text{growth rate in diameter}} = \sqrt[]{\frac{\Pi}{a_{st}}} \times {\color{cyan}\frac{da_{st}}{dt}}.
\end{equation}

For the case of seedlings, eqs. \ref{eq:eq0}-\ref{eq:eq1} can be simplified considerably. Turnover of all tissues and allocation to reproduction have not yet begun. It is also common to assume all respiration rates are proportional to leaf area. Biomass growth rate in seedling then becomes a linear function of leaf area:

\begin{equation}\label{eq:eq1_seedlings}
{\color{magenta}\frac{dm_t}{dt}}  \approx  a_{l} \, y \, (A_{\textrm{area}} - r_{l}).
\end{equation}

The \textbf{relative growth rate in mass} $(\textrm{RGR}_{\textrm{mass}})$ in seedlings can be obtained by dividing both sides of this simplified eq \ref{eq:eq1_seedlings} by the total mass of living tissues $m_{t}$. The outcome is equivalent to the classical partitioning of RGR \citep{Lambers:1992bj, Cornelissen:1998ta}:
\begin{equation}\label{eq:eq1_seedlings_RGR}
\textrm{RGR}_{\textrm{mass}}  \approx \textrm{NAR} \times  \textrm{SLA} \times  \textrm{LMF},
\end{equation}
 where net assimilation rate NAR=$y \, (A_{\textrm{area}} - r)$, SLA = $a_{l}/ m_{l}$ and leaf mass fraction LMF = $m_{l}/ m_{t}$.

  
The predicted effects of different traits on growth can then be understood by considering the influence of trait variation on elements of equations  \ref{eq:eq0}-\ref{eq:eq1_seedlings_RGR} (see Table \ref{tab:trade-offs}) and how these may interact with plant size.


\textbf{Specific leaf area (SLA):} For seedlings (eq. \ref{eq:eq1_seedlings_RGR}), cheaper leaf construction (high SLA) should translate directly into faster relative growth rate in mass. For larger plants (eq. \ref{eq:eq1}-\ref{eq:eq4}), SLA is expected to affect growth rate positively through more cost-efficient deployment of leaf area (via $da_l/dm_t$; the same effect as in seedlings), but also negatively via increased leaf turnover (eqn 1; Table \ref{tab:trade-offs}). The effect of SLA on $da_l/dm_t$ decreases as plant size increases and increasing amounts of the plant are wood. Consequently the effect of turnover costs in slowing growth comes to dominate at larger sizes. 

\textbf{Wood density (WD):} Cheaper stem construction (low WD or stem density) increases the marginal rate of leaf deployment per total mass increment ($da_l/dm_t$), and consequently increases growth rate in leaf area, height and diameter (eqs. \ref{eq:eq2}-\ref{eq:eq4}; Table \ref{tab:trade-offs}). Disadvantages associated with cheap stem construction take the form of higher mortality and so are not reflected in growth equations. Under this scenario, wood density is negatively correlated to growth rate across all plant stages. Because the relative expenditure on stem compared to leaf is increases as plants grow larger, the relationship between WD and growth may also strengthen in larger plants. 

\textbf{Maximum height (Hmax):} Hmax affects growth via the fraction of energy allocated to vegetative growth versus to seed production ($dm_t/db$ in eq. \ref{eq:eq1}; Table \ref{tab:trade-offs}). For juveniles and seedlings this fraction is 1 for all species and so no effect is expected. But once some species begin to produce seed, then species closer to their Hmax value are expected to be allocating more to seed production and less to vegetative growth. Hence when comparing across plants at a given size, species with smaller Hmax are expected to show slower vegetative growth rates.

\textbf{Seed mass:} The $\textrm{RGR}_{\textrm{mass}}$ slows as plants grow larger, and large-seeded species begin larger as seedlings. For larger plants, seed mass is not expected to have any distinct influence on growth rates  (Table \ref{tab:trade-offs}).

\textbf{Aarea:} The effects of benefits (higher assimilation) and costs (higher respiration) of increased Area occur within the same element of eq. \ref{eq:eq0}, and assimilation must outweigh respiration for there to be growth, so the net effect will be positive (Table \ref{tab:trade-offs}).   

Note that the predicted relationship between growth and either SLA or seed mass in seedlings differs if absolute growth rate in mass is used instead of $RGR_{mass}$ (i.e. becomes negative for SLA, positive for SM). In our study, all data recorded for these trait-stage combinations have been measured on a RGR base for mass (see Fig S9).




\clearpage
\begin{appendices}\label{sec:appendices}
\renewcommand{\thefigure}{S\arabic{figure}}
\renewcommand{\thetable}{S\arabic{table}}

\setcounter{figure}{0}
\setcounter{table}{0}

\section{Supporting files}\label{app:supp_info_files}

The following files are provided:

"code.zip": the data and analysis code used to generate the results in this paper. A copy of the complete data set is also available from the Dryad Digital Repository.


\clearpage
\section{Supporting tables}\label{app:supp_info_tables}

\linespread{1}

\begin{table}[h!]
\centering
\caption{\textbf{List of the twenty traits targeted}. Abbreviation,
definition and search term used in the meta-analysis} 
\label{TableA1}
\vspace{0.5cm}
\begin{tabular}{p{3cm}p{3cm}p{8cm}}
  \hline
Trait & Definition & Search terms \\
  \hline
Aarea & Photosynthetic rate per area & "photosynthetic rate per area" OR "rate of CO2 assimilation per area" OR "Aarea" OR "Amax" OR "photosynthetic capacity per area" \\
  Amass & Photosynthetic rate per mass & "photosynthetic rate per mass" OR "rate of CO2 assimilation per mass" OR "Amass" OR "Amax" \\
  Hmax & Maximum height & "potential height" OR "maximum height" OR "Hmax" \\
  Ks & Stem specific conductance & "Ks" OR "stem specific conductance" OR "stem hydraulic conductance" OR "sapwood conductance" \\
  LA & Leaf lamina area & "leaf length" OR "leaf lamina area" OR "leaf width" OR "leaf size" \\
  LAR & Leaf area ratio & "leaf area ratio" \\
  LMR & Leaf mass ratio & "leaf mass ratio" OR "leaf mass fraction" \\
  NARarea & Net assimilation rate per area &  "net assimilation rate per area" OR "rate of dry mass increament per area" OR "NARarea" OR "NAR" \\
  Narea & Leaf nitrogen content per area & "Narea" OR "nitrogen per area" OR "N/area" OR "leaf nitrogen content per area" OR "LNCarea" OR "leaf nitrogen content" \\
  NARmass & Net assimilation rate per mass &  "net assimilation rate per mass" OR "rate of dry mass increament per mass" OR "NARmass" OR "NAR" \\
  Nmass & Leaf nitrogen content per mass & "Nmass" OR "nitrogen per mass" OR "N/mass" OR "leaf nitrogen content per mass" OR "LNCmass" OR "leaf nitrogen content" \\
  Parea & Leaf phosphorus content per area & "Parea" OR "leaf P" OR  "leaf phosphorus content per area"  OR "leaf phosphorus content" \\
  Pmass & Leaf phosphorus content per mass & "Pmass" OR "leaf P" OR  "leaf phosphorus content per mass" OR "leaf phosphorus content" \\
  SA/LA & Sapwood area per leaf area & "sapwood area per leaf area" OR "huber value" OR "leaf area per sapwood area" \\
  Seed mass & Seed mass & "seed mass" OR "seed size" OR "seed volume" \\
  SLA & Specific Leaf Area & "leaf mass per area" OR "specific leaf area" OR "leaf construction cost" \\
  Thickness & Leaf thickness & "leaf thickness"  OR "leaf tissue density" \\
  Vessel density & Vessel density & "vessel density" OR "stem conduit density" \\
  Vessel size & Vessel area or diameter & "conduit area" OR "stem conduit area" OR "vessel diameter" OR "vessel size" OR "conduit size"   \\
  WD & Wood density & "wood density" OR"WD" OR "wood specific gravity" OR "stem density" OR "stem specific density" OR "SSD" \\
   \hline
\end{tabular}
\end{table}


\clearpage
\section{Supporting figures}\label{app:supp_info_figures}


\begin{figure}[h!]
\centering
\includegraphics{output/FigA1.pdf}
\caption{Allometric relationships between a) age and diameter (m), b) age and height (m), and c) diameter and height in trees according to the BAAD database. Green and red lines are the cutoffs between stage categories used in the literature. Green lines corresponds to limits between juveniles and saplings, red lines to limits between saplings and adults.}
\label{FigA1}
\end{figure}


\begin{figure}[h!]
\centering
\includegraphics{output/FigA2.pdf}
\caption{Geographical distribution of site locations from studies performed in the field used in the meta-analysis. Shown here are the "raw dataset" (recorded both across and within species, and under unstressed and stressed conditions).}
\label{FigA2}
\end{figure}


\begin{figure}[h!]
\centering
\includegraphics{output/FigA3.pdf}
\caption{Range of correlations (r) between
growth and trait reported in the literature (112 articles). Plots shown the coefficients of correlation (r)
between the plant growth and \textbf{a)} Specific leaf area (SLA, n=
111), \textbf{b)} Wood density (WD, n=64), \textbf{c)} Asymptotic
maximum height (Hmax, n=23), \textbf{d)} Seed mass (Seedmass, n=36 and
\textbf{e)} Rate of CO2 assimilation (Aarea, n=24) reported from 112
articles. The x abcissa represents the articles ID sorted by the average coefficient value r reported.
Shown here are the "raw dataset" (recorded both across and within species
and under unstressed and stressed conditions). Each bubble represents a r value, the diameter of the bubble
is proportional to its weights in the meta-analysis (i.e. number of species used). The colors
correspond to the stage for which the r values have been recorded: red
points corresponds adult stage, orange points to saplings, and grey
points to juveniles.}
\label{FigA3}
\end{figure}

\begin{figure}[h!]
\centering
\includegraphics{output/FigA4.pdf}
\end{figure}

\begin{figure}[h!]
\centering
\includegraphics{output/FigA4b.pdf}
\caption{\textbf{Mean coefficient correlation r (+SD) between growth rate and traits, established from 112 articles.} The mean coefficients are weighted by sample size (i.e number of species used to perform the correlation), and calculated both across all plant stages and for each stage. Each panel corresponds to a functional trait: \textbf{a)} Specific leaf area (SLA), \textbf{b)} Wood density (WD), \textbf{c)} Net assimilation rate (NARarea), \textbf{d)} Seed mass, \textbf{e)} Asymptotic height (Hmax) and \textbf{f)} Assimilation rate of CO2 (Aarea). Shown here are the "unstressed dataset" (recorded across species and under unstressed conditions).}
\label{FigA4}
\end{figure}

\begin{figure}[h!]
\centering
\includegraphics{output/FigA5.pdf}
\caption{\textbf{Effect size (+ 95\%CI) of plant stage on the correlations between growth rate and five functional traits across species, with a data set averaging multiple comparisons by trait and by study.} Effects size are calculated from a reduced dataset; in the case of multiple comparisons within a study, we calculated the mean correlation coefficient by trait and by study. a) Specific leaf area (SLA), b) Wood density (WD), c) Net assimilation rate (NARarea), d) Seed mass, e) Asymptotic height (Hmax) and f) Assimilation rate of CO2 (Aarea). Effect size is a standardized measure of the magnitude of the relationship between a particular stage and the correlation coefficient z (Fisher's z-transformed). Effects are significant if confidence intervals (+CI95\%) do not overlap with zero. LRT: likelihood Ratio Test compared the fit between null model and stage model, both models are fitted by Maximum Likelihood. Black point: ideal data (under unstressed conditions), grey points: all the data. 'n' indicate the number of independent comparisons for each effect.}
\label{FigA5}
\end{figure}


\begin{figure}[h!]
\centering
\includegraphics{output/FigA6a.pdf}
\end{figure}

\begin{figure}[h!]
\centering
\includegraphics{output/FigA6b.pdf}
\end{figure}

\begin{figure}[h!]
\centering
\includegraphics{output/FigA6c.pdf}
\caption{\textbf{Effects size on the correlation between growth and traits of plant stage (mod1), vegetation type (mod2), growth measurement (mod3) and experiment type (mod4).} The effect size (z + 95\%CI) is reported for each class, and models are fitted by restricted max likelihood (REML). n= number of correlation r reported. Abbreviation: across veg= more than 2 vegetation types, boreal\&temp= boreal and temperate deciduous forest, med= Mediterranean,  temp\&med = temperate deciduous and Mediterranean forests, temp= temperate deciduous, temp rain= temperate rain, trop rain= tropical rain, trop seas= tropical seasonal forests, GR= growth rate and RGR= relative growth rate, measured in diameter (D), height (H), mass(M), cross section area (CSA). AIC: Akaike information criterion corresponds to the quality of each model, and allow a comparison between models. Smallest is the AIC better is the model. Shown here are "ideal dataset" (recorded at interspecific and under unstressed condition) and "raw dataset" (recorded at both interspecific and intraspecific level and under unstressed and stressed conditions).}
\label{FigA6}
\end{figure}

\begin{figure}[h!]
\centering
\includegraphics{output/FigA7.pdf}
\caption{\textbf{Funnel plots with filled in data based on the trim and fill method, showing transformed correlation coefficient z between growth and the traits a) Specific leaf area (SLA), b) Wood density (WD), c) Asymptotic height (Hmax), d) Seed mass and e) Assimilation rate of CO2 (Aarea)}. Shown here are ideal dataset (recorded across species and under unstressed conditions).  Note that in these plots the y axis is the 1/number of species included in the correlation, not the replication for the correlation. The number of species is sometimes as low as 1, and these are within-species correlations. The number of replicates used for the correlation coefficient was often much greater than the number of species. Nevertheless, the characteristic shape of funnel plots is observed, with correlations involving few species showing much wider variation even though within-species replication was sometimes high. The open points correspond to the estimated missing studies. The number of missing values is 9 for SLA,1 for WD, 0 for Hmax, 6 for Seedmass and 0 for Aarea}
\label{FigA7}
\end{figure}

\begin{figure}[h!]
\centering
\includegraphics{output/FigA8.pdf}
\caption{\textbf{Growth measurement used by traits and by size.} Shown here are complete dataset (recorded across species and under both unstressed and stressed conditions). The colors correspond to the stage for which the r values have been recorded: red bars corresponds adult stage, orange bars to saplings, and grey bars to juveniles.}
\label{FigA8}
\end{figure}

\begin{figure}[h!]
\centering
\includegraphics{output/FigA9.pdf}
\caption{\textbf{Temporal changes in coefficient of correlation between traits and growth. a) Specific leaf area (SLA), b) Wood density (WD), c) Asymptotic height (Hmax), d) Seed mass and e) Assimilation rate of CO2 (Aarea)} Shown here are ideal dataset (recorded across species and under  unstressed conditions). The colors correspond to the stage for which the r values have been recorded: red points corresponds adult stage, orange points to saplings, and grey points to juveniles.}
\label{FigA9}
\end{figure}

\clearpage
\section{Supporting Analyses}\label{app:supp_info_analyses}
\subsection{Quantifying and exploring the cause of heterogeneity in effect sizes}

\textbf{Goal:} In our meta-analysis, studies reported differed in design as well as in vegetation types, biome or species studied. Such diversity is commonly referred as methodological heterogeneity, and may or may not be responsible for observed discrepancies in the results of the studies. The extent of heterogeneity might influence the results of the meta-analysis, and then induces some difficulty in drawing overall conclusions (See \citealt{Higgins:2002iq}). It is therefore important to be able to quantify the extent of heterogeneity among a collection of studies. 

\textbf{Methods:} A common way of addressing the extent of heterogeneity is the statistic $I^{2}$ (\citealt{Santos:2012gt}, originally defined by \citealt{Higgins:2002iq}). $I^{2}$ estimates the consistency of the results obtained from published studies used in our meta-analysis; it describes the percentage of variability in point estimates that is due to heterogeneity rather than sampling error. It can be interpreted as a ratio (0\% < $I^{2}$ < 100\%) not affected by the number of studies or the metric of the effect size in the analysis. It also has the advantage of being analogous to indices used for regression (where R2 is the proportion of the total variance that can be explained by the covariates) a common statistics in the trait literature.

An $I^{2}$ near zero indicates that almost all of the dispersion will be attributed to random error, and any attempt to explain the variance is an attempt to explain something that is (by definition) random. By contrast, as $I^{2}$ moves away from zero we know that some of the variance is real and can potentially be explained by subgroup analysis or meta-regression. 
No universal rule could cover definition for ‘low, ‘moderate’ or ‘high heterogeneity. \citealt{Higgins:2003hz} suggested some benchmarks for $I^{2}$ based on the survey of meta-analyses of clinical trials. They suggested that values on the order of 25\%, 50\% and 75\% might be considered as "low", "moderate" and "high", respectively. These suggestions are tentative; the interpretation of heterogeneity in a systematic review will depend critically on the size and direction of treatment effects, as well as on considerations of methodological diversity in the studies (see \citealt{Borenstein:2009um}).
Here, we estimated $I^{2}$ as following equation from \citealt{Higgins:2002iq}., then we calculated how part of the heterogeneity may be due to the influence of moderators (stage and publication year) and finally we asked if the residual heterogeneity is statistically significant. In R, we used the package metafor (code available on guithub).


\textbf{Results:} We observed high levels of heterogeneity for all traits (SLA :$I^{2}$ 85.4\% CI: 80.9 to 92.1, WD :$I^{2}$ 61.8\% CI: 42.8 to 84.1, Hmax: $I^{2}$ 81.2\% CI: 69.8 to 93.4, Seedmass: $I^{2}$ 91.2\%, 85.9 to 96.1, Aarea: $I^{2}$ 68.3\% CI: 33.7 to 90.1) which suggests that the correlations coefficient r between growth and traits are not uniform across the range of studies investigated. This result suggested that a large part of the variance is not explained by sample error.

For SLA and Seedmass, 50.1\% and 20\% respectively of the total amount of heterogeneity were accounted by including the stage as a moderator. For the other traits, the stage explained few or no part of the heterogeneity observed across studies (<5\%). These results were consistent with our expectations about a shift in the relationship between growth and traits along plant size for SLA, Seed mass and the absence of effect for WD and Aarea (see Theoretical expectation section). The result of Hmax was more ambigus since we expected the relationship between asymptotic maximum height (Hmax) and growth rate to be absent in seedlings but to strengthen with increasing plant size (at adult stage). As discussed in the result section of the article, the smaller numbers of replicate studies for sapling and seedling stage limited the generality of our conclusion for this trait. 

The test for residual heterogeneity was significant for all traits, possibly indicating that others moderators not considered in the model were influencing the correlation between growth and trait. This result is not surprising for the trait literature. While the trait and growth measurements are standardized \citep{Cornelissen:2003gw}, studies differed largely in their design as well as in vegetation types, biome or species studied. Indeed, a goal in the trait literature is to establish general pattern about correlation between growth and traits and to validate them across a large variation of conditions. Here, we did not explore the influence of other moderators, since our goal was to test clear hypotheses about the influence of plant size/stage, not to establish the best predictor of the correlation between growth and trait. Finally, since between study variation is significant, as indicated by heterogeneity analyses, using the random effect model was more appropriate in our meta-analysis.

\subsection{Testing for publication bias}
\textbf{Goal:} Studies with significant results are more likely to be published. As a consequence, the studies in a meta-analysis may overestimate the true effect size because they are based on a biased sample of the target population of studies. A publication bias occurs when the probability of publication depends on the statistical significance of the effect. Here we tested if there is an evidence of any bias, and if the effect size observed for each trait is an artefact of this biais.

\textbf{Method:} We first checked the dataset for publication bias with a visual assessment of funnel plots. A publication bias toward significant results creates an asymmetric funnel with small studies with non-significant effects missing from the mouth of the funnel on the side opposite to the true effect. While funnel plot are recommended to aid interpretation, there is a high risk failing to detect actual publication bias (see (\citealt{Koricheva:2013tz}) by using exclusively a visual assessment. In addition, funnel plots are not effective when the number of studies is small (<30) such as for Aarea and Seed mass. 
We also used the ‘trim and fill’ method to estimate the number of studies missing from our meta-analyse. This method adjusts for funnel plot asymmetry \citep{Duval:2000dg}. 
Finally, we estimated the number of studies needed to overturn a result, using the Rosenberg’s fail-safe number \citep{Rosenberg:2005hk}. The fail-safe N is not based on funnel plot asymmetry. The Rosenberg method calculates the number of studies averaging null results that would have to be added to the given set of observed outcomes to reduce significance level (p-value) of the weighted average effect \citep{Rosenberg:2005hk}. A significant meta-analytic result is robust if the fail-safe N is greater than 5k+10, where k is the number or studies already in the meta-analysis \citep{Rosenthal:1979do}.
In R, we used the package metafor (code available on guithub).

\textbf{Result:} As expected the correlation coefficient varied less widely as sample size increased (funnel plots in Fig. \ref{FigA9}). Funnel plots exhibited a typical funnel shape for all five functional traits analyzed, showing no evidence for publication bias (Fig. \ref{FigA9}). 
The ‘trim and fill’ method detected some missing values in our data. The estimated number of missing value ranges from 0 to 9 depending on the trait considered (Fig. \ref{FigA9}). The overall estimates measured with the fill in data changed from 0.5 ±0.07 to 0.4 ±0.07 for SLA and from -0.48±0.11 to -0.32±0.12 for the seed mass, suggesting a lower correlation coefficient for SLA, and Seedmass. Even if the estimated correlation were closest from zero with the missing studied filled in, the results still indicate that the effect is statistically significant.
The Rosenberg’s Fail-safe N are 7086, 1823, 990, 574, 291 for SLA, WD, Hmax, Seedmass and A area, suggesting that there would need to be over a large number of studies with a mean correlation coefficient of 0 added to the analysis before the cumulative correlation would become non significant. N is greater than 5k+10 for all traits.
All together these result suggest that the impact of bias is probably trivial in our meta-analyses. 

\subsection{Exploring temporal changes in effect size}
\textbf{Goal:} Our meta-analysis combines data from studies published between 1983 and 2014. It has been shown that the magnitude of the effect size may changes over time due to a publication bias (i.e. a lag to publish negative results), a change in the methodology or biological changes in the magnitude of the effect (see \citealt{Koricheva:2013hy}). Therefore, detecting such temporal changes may be important to the interpretation of the results of a meta-analysis. 

\textbf{Method:} In the trait literature the year of publication may be confounded with a methodological change. Indeed, in the early 90’s studies mainly focused on seedlings, whereas today a broader range of plant size/stage are studied (Fig. \ref{FigA9}). As a consequence, we used publication year as a moderator in our analyses \citep{Zvereva:2008jm}p across and within stages for all traits. 
 
\textbf{Results:} There was a change in the effect size with publication year for SLA (pvalue<0.0001, estimates: -0.04 ±0.009), Seed mass (pvalue =0.011, estimates: 0.041 ±0.016) and Aarea (pvalue=0.0102, estimates: -0.0453± 0.0176). The Fig. \ref{FigA9} showed that for these 3 traits the publication years may be confounded with plant stage; the correlation coefficients have been reported from 1993, but studies started to report data on adult stage only 10 years later (around 2000-2007). 

For SLA and Seed mass, a publication year effect was not observed within seedling and sapling stages, but was identified for the adult stage (SLA pvalue=0.003, Seedmass pvalue=0.06). For both SLA and Seedmass, the magnitude of the effect size decreased with publication year (Fig. \ref{FigA9} a and d), suggesting we overestimated the magnitude of the effect size for these two traits.  In reality, the main effect size obtained would be more close to zero or more negative for SLA and more positive for Seedmass. These results are consistent with our hypotheses; we expected a negative or non-significant effect of SLA on growth in adult stage, and a non-significant or positive effect of Seed mass on plant growth (see theoretical expectation).

For Aarea, publication year explained a part of the variation across stage but not within stage, and the magnitude of the effect size decreased with publication year. This pattern is classic in meta-analyses; \citealt{Koricheva:2013hy} showed that a majority of studies reported a decrease rather than an increase in the magnitude of effect size with publication year.  Yet, here this effect did not lead to a loss of the statistical significance of the main effect size or to a change of sign. While overestimated, the main effect size for the correlation between growth and Aarea stay positive  (Fig. \ref{FigA9} e).

The temporal change were confounded with a plant stage effect in our meta-analyse, but they did not jeopardizing the stability of our conclusions. The conclusions of a meta-analyses conducted in different year may not differ from the one reported here.


\end{appendices}
\end{document}